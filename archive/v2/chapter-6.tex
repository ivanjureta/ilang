\chapter{How does an innovation language relate to decision-making?}
\label{c6}
\chaptermark{How does an innovation language relate to decision-making?}

\abstract{An innovation language is created and changes during an innovation process. I argue in this chapter that definitions which we thus create reflect the decision-making done during innovation. I argue that definitions record choices we made, as well as embed constraints on our future decision-making. A key idea in this chapter is that new terms we define need to rely on old terms, those whose definitions we do not change during the innovation process; this relationship between new and old terms tells us how the outcomes of the innovation process relate to what was there already.}

\section{How can it relate to decision-making?}
\label{c6:s1}
I mentioned, in earlier Chapters, that a definition in an innovation language relates a new term to old. It does so in the following sense: when you introduce a new term, to explain to others what you mean by it, you provide a definition; for them to understand this definition, or any of it, you need to explain the meaning of the term using existing terms, those which you assume they understand similarly to you. So any term you put in an innovation language, and this is independent from the plasticity of its definition, you will invariably explain its meaning using at least some words which are not in the innovation language, but are part of the base language.

This Chapter looks at how the definition of a new term relates to terms which are not part of the innovation language, but bring with them their own definitions, those which were not shaped during a given innovation process.

These ''old'' terms are only old in the sense that their definitions were created before and/or outside a given innovation process. Someone, sometime, somewhere had to make decisions on how to define these terms, and therefore, their definitions embed these decisions. Definitions reflect the choices we make, about how we explain that which we want the term to mean, or refer to.

To ''embed decisions'', then, means that a term's definition is a product of decisions of those who made it. And so, if someone else defined the term T, and you mention T in your definition of some new term S you are introducing in your innovation language, your definition of S will embed the choices from the definition of T; indeed, someone else cannot understand your S unless they understand T.

\section{How new ideas use old terms?}
\label{c6:s2}
What is the relationship between new words and old words in an innovation language? How do old terms build up definitions of new ones?

Consider an example: How do you explain a satellite to a child? Here are two definitions that you could rely on.

\begin{svgraybox}
Satellite (OED definition): ''An artificial body placed in orbit round the earth or moon or another planet in order to collect information or for communication.'' \cite{def-satellite}
\end{svgraybox}

\begin{svgraybox}
Satellite (Wikipedia definition): ''In the context of spaceflight, a satellite is an artificial object which has been intentionally placed into orbit. Such objects are sometimes called artificial satellites to distinguish them from natural satellites such as Earth's Moon.'' \cite{wikipedia-satellite}
\end{svgraybox}

If the child already has an idea of artificial bodies, orbit, earth, planet, of what collecting information and communication are about in relation to these notions, then the first definition might be a good start. It is not an explanation, but we will come back later to the differences between definitions and explanations.

How about this, which NASA says is for children ages 10+?

\begin{svgraybox}
''A satellite is a moon, planet or machine that orbits a planet or star. For example, Earth is a satellite because it orbits the sun. Likewise, the moon is a satellite because it orbits Earth. Usually, the word ''satellite'' refers to a machine that is launched into space and moves around Earth or another body in space.'' \cite{nasa-satellite-58}
\end{svgraybox}

Or the following version, from NASA too, for a younger audience.

\begin{svgraybox}
''A satellite is an object that moves around a larger object. Earth is a satellite because it moves around the sun. The moon is a satellite because it moves around Earth. Earth and the moon are called ''natural'' satellites. But usually when someone says ''satellite,'' they are talking about a ''man-made'' satellite. Man-made satellites are machines made by people. These machines are launched into space and orbit Earth or another body in space.'' \cite{nasa-satellite-k4}
\end{svgraybox}

If you had to choose one, which one would it be? Are either of the first two good candidates?

You can think of these questions by looking at definitions backwards, from the definiens which defines or explains, to the definiendum, i.e., the term being defined.

Take the Wikipedia Satellite definition again. Notice it mentions the term ''context'', a complicated term. Context is an abstraction that bears multiple precise definitions, which are likely to vary across specific disciplines; i.e., context has a different definition in different technical languages. For example, Wikipedia gives this, for ''context'' in computer science:

\begin{svgraybox}
Context (Wikipedia definition, computer science): ''In computer science, a task context is the minimal set of data used by a task (which may be a process or thread) that must be saved to allow a task to be interrupted, and later continued from the same point.'' \cite{wikipedia-context-computing}
\end{svgraybox}

This is certainly not a unique or conventional understanding of the notion of context in computer science altogether. There are specific, and substantially different definitions of context in, e.g., knowledge representation, requirements engineering, mobile computing, and so on. Here is a definition from mobile computing research.

\begin{svgraybox}
Context (mobile computing): ''Context is any information that can be used to characterize the situation of an entity. An entity is a person, place, or object that is considered relevant to the interaction between a user and an application, including the user and applications themselves.'' \cite{abowd1999towards}
\end{svgraybox}

In short, context has a definition that varies between technical languages. In philosophy, the relationship between context and knowledge is an ongoing and longstanding debate.\cite{Rysiew2016}

Oxford English Dictionaries give the following generalist definition.

\begin{svgraybox}
Context (OED definition): ''The circumstances that form the setting for an event, statement, or idea, and in terms of which it can be fully understood.'' \cite{def-context}
\end{svgraybox}

Being twisted in this way, ''context'' is not a term to use lightly, when explaining ''satellite'' to children.

The important takeaway here is not that it is hard to explain a satellite to a child, but this: how you define a term depends on two things, namely
what you assume the user of that definition already knows, and
what they may want to do, once they learn the definition or explanation you give them.

You may, sometimes in error, assume that the child does not know enough for you to use the Oxford English Dictionary definition. If you also think the child needs it simply out of curiosity, then perhaps either of the two NASA definitions will do.

But what if an astrophysicist asks an engineer for a definition of ''satellite'', not out of curiosity, but because they are working together on the design of a new satellite, and they are trying to align their ideas. None of the definitions above might work; perhaps their definition would speak of gravitationally bound objects in orbits, or use other technical terms more appropriate to their background knowledge and the task at hand.

This gets us to the second and third takeaways
\begin{itemize}
	\item it makes little sense to think about one term and its definition in isolation from everything else, and
	\item a technical language leans on natural language to some extent, even if only by using terms from natural language in definitions of technical terms.
\end{itemize}

The following illustrates how the Satellite definition from the Oxford English Dictionary presumably leans on other natural language terms. Some are emphasized below.

\begin{svgraybox}
Satellite (OED definition, emphasis added): ''An \textit{artificial body} placed in \textit{orbit} round the \textit{earth} or \textit{moon} or another \textit{planet} in order to \textit{collect information} or for \textit{communication}.'' \cite{def-satellite}
\end{svgraybox}

But that definition of Satellite is taken from the Oxford English Dictionary, so by design, it leans on other words from there, i.e., on how these other words are defined there.

Let's go to a different example now, where some words in the definiens are not in the Oxford English Dictionary at the time of writing.

\begin{svgraybox}
Biosafety (UNDP definition): ''The prevention of large-scale loss of biological integrity, focusing both on ecological and human health. Set of measures or actions addressing the safety aspects related to the application of biotechnologies and to the release into the environment of transgenic plants and organisms, particularly microorganisms, that could negatively affect plant genetic resources, plant, animal or human health, or the environment.'' \cite{undp-glossary}
\end{svgraybox}

This definition of Biosafety comes from the United Nations Development Programme (UNDP), in a glossary for documentation on sustainable development. Below, terms set in bold are not defined in the Oxford English Dictionary, and so for all practical purposes here, not part of natural language. At the same time, these terms are not defined in that UNDP glossary; terms set in italics are not part of natural language, but are defined in the glossary. The rest of the terms are part of natural language.

\begin{svgraybox}
Biosafety (UNDP definition): ''The prevention of \textit{large-scale loss} of \textit{biological integrity}, focusing both on ecological and human health. Set of measures or actions addressing the safety aspects related to the application of biotechnologies and to the release into the environment of transgenic plants and organisms, particularly microorganisms, that could negatively affect plant genetic resources, plant, animal or human health, or the environment.'' \cite{undp-glossary}
\end{svgraybox}

To summarize, in the Biosafety definition above, there are two terms outside natural language and outside the glossary itself: ''large-scale loss'' and ''biological integrity'', and only ''Biosafety'' is itself defined in the glossary, no other term from the definiens is.

Is that a good definition?

This is hard to say, as ''good definition'' itself is an open topic, one that spans most of this book - recall the various perspectives discussed in chapter \ref{c4}. It is clear, however, that there are several things to think about here:
''Biosafety'' is defined independently from the rest of the glossary,
''Large-scale loss'' is undefined within the topic of the glossary, i.e., the glossary does not help reduce the vagueness of that term in any way, and
''Biological integrity'' is undefined in the glossary, and since it is not part of natural language, the authors in the glossary must have leaned on a technical language, but did not point to it in the glossary.

If these look like quibbles, they are not. Here is a sample of the activity around getting to and agreeing on a technical definition of ''biological integrity'':

\begin{quote}
''[United States Environmental Protection Agency] convened a symposium [...] on the integrity of water soon after passage of PL 92-500 [Federal Water Pollution Control Act], but no clear definition of biotic integrity emerged. Many authors advocated the use of a holistic perspective. Karr and Dudley [...] argued that the ''integrity'' objective encompasses all factors affecting the ecosystem and developed a now widely quoted definition of biological integrity as the ability to support and maintain 'a balanced, integrated, adaptive community of organisms having a species composition, diversity, and functional organization comparable to that of natural habitat of the region.' A more recent paper defined ecological health (an umbrella goal, the maintenance of which motivates virtually all environmental legislation) as follows: 'a biological system...can be considered healthy when its inherent potential is realized, its condition is stable, its capacity for self-repair when perturbed is preserved, and minimal external support for management is needed'[...].'' \cite{karr1991biological}
\end{quote}

Disagreement is common when specialists work on definitions. The problem is not disagreement, but disconnect, which happens if the discussion involves parallel monologues, each pushing its own definition, and no dialogue where there is convergence to an agreement on the definition of each debated term.

Going back to takeaways, these examples illustrate how a definition, even of a new technical term, depends on natural language and therefore of old terms and their readings accepted as part of natural language, however unclear they may be. As long as your technical language leans on natural language, which is not bad in itself, you are building something that is intended to be more precise, accurate, and clear from pieces which may not be that at all. In the worst case, you are simply changing where vagueness, ambiguity, and other deficiencies are coming from.

New ideas need new words. At the same time, new words are anchored in old ones.



\section{How definitions embed choices?}
\label{c6:s3}
How do definitions in an innovation language reflect our problem-solving and decision-making?

A technical language is a record of past decisions. This is a simple idea with significant consequences.

Let's start with a simple example, from the consulting contract case introduced in chapter \ref{c5}. We had the following definition.

\begin{svgraybox}
\textbf{Service:} The collaboration of one specific Consultant and one specific Client, and under the rules and guidelines set out by the Service Contract to which they have both agreed.
\end{svgraybox}

Consider these alternative definitions. Differences from the original one above are emphasized.

\begin{svgraybox}
\textbf{Service1:} The collaboration of one or more specific Consultants and one specific Client, and under the rules and guidelines set out by the Service Contract to which they have both agreed.
\end{svgraybox}

Service1 allows any number of Consultants to work for a Client under the same Service Contract. This may be better or worse than the original, but in either case, it can only be so if we decided that it should be so. In other words, the original definition excludes this possibility, because we decided to exclude that possibility. The main reason was that it would require coordination between Consultants, and that could have an impact on the speed at which we get to an agreed Service Contract.

Service2 below, is another alternative.

\begin{svgraybox}
\textbf{Service2:} The collaboration of one specific Consultant and one specific Client.
\end{svgraybox}

Service2 is less specific than either Service and Service1. You can see it as a less mature variant of each of these. It fails to mention the contract as the framework for collaboration. By doing that, it fails to say how we go from negotiation to collaboration, the transition which happens because there is agreement on a contract.

The point here, is that definitions, especially about new things, and especially if these new things are abstractions, ideas, are not given to us. We make them, and as a result, we define.

To clarify that given/made distinction, here is the Oxford English Dictionary definition of giraffe.

\begin{quote}
Giraffe (OED definition): ''A large African mammal with a very long neck and forelegs, having a coat patterned with brown patches separated by lighter lines. It is the tallest living animal.'' \cite{def-giraffe}
\end{quote}

This definition conveys what one generally sees, when looking at the animal called giraffe. Giraffes, in other words, are given -- it is not man-made, but a product of evolutionary processes. Alternatively, if we used a definition that lists a different set of properties, as below, we are still in the same case: a giraffe has these properties not because we made them so, i.e., the definition is not a reflection of our design of giraffes, but of our observation of giraffes.

\begin{quote}
Giraffe (AWF definition):
\begin{itemize}
\item ''Scientific name: Giraffa camelopardalis
\item Weight: Males: 1,930 kilograms (4,254 pounds) Females: 1,180 kilograms (2,601 pounds)
\item Size: 5.7 meters tall from the ground to their horns (18.7 feet)
\item Life span: Average 10 to 15 years in the wild; recorded a maximum of 30 years
\item Habitat: Dense forest to open plains
\item Diet: Herbivorous
\item Gestation: Between 13 and 15 months
\item Predators: Humans, lions, leopards, hyenas, crocodiles'' \cite{awf-giraffe}.
\end{itemize}
\end{quote}

But things are very different for our ''Service'' term above, since it refers to an abstraction, one we fully designed -- a team chose its properties and from there followed its definition.

There are also in-between cases where we are trying to define a complex process that has and is being observed from various perspectives. 

''Globalization'' is an interesting case, and below are four definitions that the United Nations Economic and Social Council cites in an official document \cite{unesc-terminology}.

\begin{quote}
Globalization (UN definition): ''Globalization is increased global integration and interdependence. It has a multidimensional character: economic, political, social and cultural. It is characterized by unprecedentedly rapid flows of goods and services: private capital; circulation of ideas and tendencies; and the emergence of new social and political movements.''
\end{quote}

\begin{quote}
Globalization (IMF definition): ''The process through which an increasingly free flow of ideas, people, goods, services and capital leads to the integration of economies and societies. Major factors in the spread of globalization have been increased trade liberalization and advances in communication technology.''
\end{quote}

\begin{quote}
Globalization (Hirst \& Peters definition): ''Globalization describes the growth in international exchange and interdependence. With growing flows of trade and capital investment, there is the possibility of moving beyond an international economy (where 'the principle entities are national economies') to a 'stronger' version of the globalized economy in which 'distinct national economies are subsumed and rearticulated into the system by international processes and transactions'.''
\end{quote}

\begin{quote}
Globalization (World Bank definition): ''Globalization can be defined as universalization. In this use, 'global' is used in the sense of being 'worldwide' and 'globalization' is 'the process of spreading various objects and experiences to people at all corners of the earth'. A classic example of this would be the spread of computing, television etc.''
\end{quote}

The World Bank definition looks more abstract than the others, in that it defines globalization as a type of universalization (which I leave undefined here), while the other three definitions talk of integration and interdependence.

Below is another definition of ''globalization'', as it appears in the United Nations UNTERM, ''a multilingual terminology database maintained jointly by the main duty stations and regional commissions of the United Nations system.''

\begin{quote}
Globalization (UNTERM definition): ''Expansion of global linkages, organization of social life on global scale and growth of global consciousness, hence consolidation of world society. Most particularly (per Friedman), the term is used to refer to the seemingly ''inexorable integration of markets, nation-states and technologies to a degree never witnessed before -- in a way that is enabling individuals, corporations and nation-states to reach around the world farther, faster, deeper and cheaper than ever before.'' This phenomenon also involves the spread of the gospel and forms of market capitalism to virtually every country in the world.'' \cite{unterm-globalization}
\end{quote}

Is any of these, or is there, at all, the definition of globalization?

It is a complex process, which we do not know how to measure (or there are different ways, but we cannot agree on one). There is disagreement on when, where and how it started. We do not know what cues to look for, which data is representative of it. We do not know if it has reached a peak, what reaching a peak means anyway, and we have hardly any solid ideas on how globalization might end.

\begin{quote}
''Today few doubt the reality of globalization, yet no one seems to know with any certainty what makes globalization real. So while there is no agreement about what globalization is, the entire discourse on globaliz- ation is founded on a quite solid agreement that globalization is. Behind the current and confusing debates about its ultimate causes and con- sequences, we find a wide yet largely tacit acceptance of the factuality of globalization as such, as a process of change taking place 'out there': even otherwise constructivistically minded scholars tend to regard globaliz- ation as an undeniable and inescapable part of contemporary experience.'' \cite{bartelson2000three}
\end{quote}

Which of these would you choose?

This question depends on what, if anything you want to do about, with, or in relation to the phenomenon that is defined.

More specifically, a choice between definitions depends on the extent to which their differences make a difference in your decision-making and problem-solving.

Going back to our small-scale consulting contract case, the choice of one of those three definitions given earlier (Service, Service1, and Service2) does make a difference. ''Service'' requires a different negotiation process than ''Service1'', and both can involve different processes than ''Service2''.

Definitions indeed record choices. If they define what is given and (relatively) easily observable, such as a giraffe, then definitions are more a record of choices of properties to pay attention to, among all potential (and perhaps unknown) properties of giraffes. If, instead, they define abstractions which we fully imagined and designed, then the definition reflects our choices of properties to give to these abstractions. And in many cases, we will be creating abstractions to communicate about potentially complicated combinations of phenomena, as we saw for globalization, where properties reflect, again, what we see through observation, but perhaps too, some wishful thinking, what we want these phenomena to be.

A definition is the bridge of the new to the old, we piece together the new from old notions. The new, by being there, by being defined, becomes a tool or vehicle for change, for moving not only ideas from old to new, but also learning to make new choices and act differently than before. 

Conceptualizations of ''globalization'', its definitions, work the same way, embodying observations, choices, mashing them up with expectations, making it a vehicle, not a simple reflection of what seems to be a shared experience. Bertelson's summary, of his discussion of the concept of ''globalization'' nicely sums up the multiple roles a definition can have, especially for complex phenomena, which lend themselves to analysis from different angles, and through tools from various bodies of knowledge.

As he puts it, the term ''globalization'' links the old and the new (label [A] in the quote below), denotes the interest in the phenomenon itself [B], tells us about what can be [C] and when it will be [E], and points to a vague reality [D], i.e., does not refer to something tangible or otherwise easily accessible, for which you and I could rapidly come to same or similar enough observations.

\begin{quote}
''Through its various connotations, the concept of globalization functions as [A: a mediating link between the modern world with its crusty social ontology and the brave new world that remains inaccessible and unintelligible not only to the subscribers to that ontology but also to the believers in global change as well]. Not only is globalization a moving target for social inquiry, but it also [B: signifies the movement of that inquiry itself]. The concept of globalization thus makes modern political experience meaningful while simultaneously releasing political expectation from the strictures that the very meaningfulness of this experience has imposed on political imagination. It does so through a double gesture: by [C: projecting expectations onto the global while simultaneously making those expectations constitutive of globality as a point of reference and convergence]. In this respect, the logic of the concept of globalization resembles that of the concepts of civilization and revolution as they were shaped before and during the French Revolution: [D: these concepts also lacked stable referents, but functioned as vehicles of social change by signifying change in its purest, most necessary and irreversible form: change as the condition of possible objects and possible identities in a possible future]. And like these concepts, globalization does not represent a mere prognosis for the future, but a prophecy in quest for self-fulfillment. So far from being here to stay, the metaphors of globalization will perhaps die when the concept has fulfilled its destabilizing task, that is, [E: when globalization has become something that goes without saying and therefore no longer stands in need of being spoken about].'' \cite{bartelson2000three}
\end{quote}


\section{How definitions make choices?}
\label{c6:s4}
How can definitions make it easy do do some things, and hard to do others? How do they impose constraints on future choices?

If you commit to a definition, meaning you choose one definition of a term over others, then you also take additional commitments. It is yet another case of buying one and getting more than you asked for. How much of this more do you know? Do you want this additional baggage? These are the two central questions.

Law is an interesting domain when thinking about definitions. This is for two reasons: one, it insists on using terms in a precise and consistent way, and two, how these terms are used has substantial effects on people's lives.

Consider the hard case, of ''terrorism''. How would you define it?

Here is how terrorism is defined in the United Kingdom Terrorism Act. The version below is from 2018. Most text is the original definition, and it was included in law in 2000; text labeled [A] was added to the original definition in 2006, and the updated definition was published in the Terrorism Act 2006; text labeled [B] was added in 2008, and published in Counter-Terrorism Act 2008.

\begin{quote}
''Terrorism: interpretation.
(1) In this Act 'terrorism' means the use or threat of action where 
(a) the action falls within subsection (2), 
(b) the use or threat is designed to influence the government [A: or an international governmental organisation] or to intimidate the public or a section of the public, and 
(c) the use or threat is made for the purpose of advancing a political, religious [B:, racial] or ideological cause. 
(2) Action falls within this subsection if it 
(a) involves serious violence against a person, 
(b) involves serious damage to property, 
(c) endangers a person's life, other than that of the person committing the action,
(d) creates a serious risk to the health or safety of the public or a section of the public, or 
(e) is designed seriously to interfere with or seriously to disrupt an electronic system. 
(3) The use or threat of action falling within subsection (2) which involves the use of firearms or explosives is terrorism whether or not subsection (1)(b) is satisfied. \\
(4)In this section \\
(a) 'action' includes action outside the United Kingdom, 
(b) a reference to any person or to property is a reference to any person, or to property, wherever situated, 
(c) a reference to the public includes a reference to the public of a country other than the United Kingdom, and 
(d) 'the government' means the government of the United Kingdom, of a Part of the United Kingdom or of a country other than the United Kingdom. 
(5) In this Act a reference to action taken for the purposes of terrorism includes a reference to action taken for the benefit of a proscribed organisation.'' \cite{uk-terrorism-act}
\end{quote}

This is a complicated, but carefully made definition of a complex and important phenomenon. It is also a poorly understood phenomenon. Any definition, therefore, however attentively made, will have limitations.

One of the possible consequences of applying this definition is that some acts, which normally do not look like cases of terrorism, still could be categorized as terrorism.

\begin{quote}
"Civil disobedience, public protest and industrial action are among the activities that could fall within the definition. These types of activities should be excluded from any definition of terrorism. [...] Unlike the definition in the Australian Criminal Code (and its State equivalents), the New Zealand Terrorism Suppression Act, the Canadian Criminal Code and the 2003 South African Anti-Terrorism Bill, the United Kingdom definition does not contain an exception in favour of advocacy, protest or industrial action. The legislation simply requires the purported terrorist to have committed an act (such as endangering a person's life, or seriously damaging property), and to have committed that act in furtherance of a political, religious or ideological cause with the aim of influencing the government or intimidating the public (or a section thereof). This encompasses groups whose methods are generally non-violent and who do not aim to intimidate or to coerce the government or the public. For example, a long-running nurse's industrial dispute where staffing levels in public hospitals have been seriously reduced could 'create a serious risk to the health or safety of the public', within the meaning of s 1(2)(d) (as could the industrial actions of other essential services, such as fire officers, police, and so forth). If the strike were directed towards convincing the government to increase pay and conditions in public hospitals then this could also satisfy both the 'political cause' and the 'influencing government' requirements, in s1. Similarly, a mass student protest against the deregulation of university fees by the British Government could also fall within the definition of terrorism." \cite{golder2004terrorism}
\end{quote}

These are interesting potential consequence, regardless of intentions that led to the definition.

Going back to less complicated topics, and more in the context of innovation languages, your choice of one definition over another has consequences on the structure of the problem space, of the solution space, and on the innovation process. This means that how you define something has an impact on what problem you are going to be solving -- that is the problem space part -- the potential ways you will be looking at to solve it -- the solution space part -- and on how you go through the formulation of the problem and its resolution -- the innovation process part. In other words, it can affect really all that goes on during innovation.

In chapter \ref{c5}, I introduced the case where we were designing a process for contract negotiation. One of the issues that was highlighted early on, was that such contract negotiation tends to take an unpredictable amount of time. They can last long, and if they last too long, they may simply not lead to the contract at all. So one of the aims that was taken for granted, was that the process should shorten the time to contract, or more broadly, to the decision to sign or not.

How do these early assumptions shape the problem space? Consider these three potential definitions for the contract negotiation process.

\begin{svgraybox}
\textbf{Contract Negotiation 1:} Process in which parties agree on terms of collaboration.

\noindent\textbf{Contract Negotiation 2:} Process in which a party that needs to have assignments performed communicates with potential parties that can execute these assignments, until she has either agreed with one on terms for collaboration towards the execution of these assignments, or has agreed with none of the considered parties.

\noindent\textbf{Contract Negotiation 3:} Process in which parties agree on terms of collaboration within two calendar weeks from the time the asking party requested a proposal from the supplying party.
\end{svgraybox}

The first definition is the least constraining. It leaves a lot open. Nothing is said about how many parties can be involved, how much time they can take, what they can collaborate on. The second definition is clear about the number of parties that can be involved, and how the process ends. Only the third definition mentions the duration of the process, and is precise about it, but is open regarding the number of participants.

If the innovation team adopts the third definition, they still need to decide on the number of participants, among others. But can they allow any number of participants? Does allowing fewer participants make two weeks a long time for negotiations? Is two weeks too short, when there are many participants? Does the number of participants influence duration? Is it the other way around, does the duration limit the number of participants? Should the process differ in maximal duration if it has many participants? Which numbers would require which durations?

If the team does indeed adopt the third definition, and makes it immutable, then this affects substantially the possible designs of the negotiation process. And it does so in ways which may not be easy to see as early on in the innovation process. We did, in fact adopt a definition which included a time limit. Because of this, we had to restrict communication between the parties. If we left it open, letting them freely communicate, we believed that they would not converge to a go/no go decision quickly enough. This, in turn, required that we provide template contracts, where it was clear which exact parameters could be negotiated. And that, then, required that we restrict communication between parties only to the negotiation of values of those parameters. Yet another unanticipated consequence of this, is that it can work only for highly structured contracts, where these negotiable parameters are few and well known. For example, a monthly retainer amount and a price per hour, together with the list of tasks. That is a far reaching consequence, because it affects the size of the potential market for the process, and the software that was to be made to support that process.

This does not mean that a more open definition (one with fewer constraints) is better. If the team takes seriously the first definition instead, then it may ignore the relationship of duration to outcome, and more specifically that the duration of negotiations influences negotiation success.

All these issues seem obvious as you read this text, but they are hard to see and analyze when you are involved in the innovation process. It is often only in retrospect that their merits and drawbacks become clear.

In that respect, you can see a definition as a bundle of predictions. In the case I mentioned, these are that two weeks are a good duration for contract negotiation, that the number of participants needs to be two or more, and so on. These are predictions about what will make the outcomes of innovation useful.

An analysis to do, that helps understand the implications of a definition on the problem and solution spaces, is to identify choices in the definition, and identify at least one other option for each. I did it below on the three definitions.

\begin{svgraybox}
\textbf{Contract Negotiation 1:} Process in which (one, two, more parties) (agree, disagree, cancel negotiations) on terms of (delegation, collaboration).
\end{svgraybox}

I identified three choices in the first definition. One is the number of parties, with four options (either one of the three, and all three as the fourth). The second is the outcome of negotiations, with four options as well. The final is the object of negotiations, with two options. You could have gotten other choices and other options for each. What the analysis highlights, is that the first definition commits to one out of 4x4x3 possible combinations; the problem space is restricted to the chosen options in each choice, and any solution will have to solve a problem in that part of the problem space.

Consider now the second definition.

\begin{svgraybox}
\textbf{Contract Negotiation 2:} Process in which a party that needs to have (one, more assignments) performed communicates with (one, more than one potential parties) that can execute (all, subset of these assignments), until she has either agreed with (none, one, more than one party) on terms for collaboration towards the execution of (all, subset of these assignments).
\end{svgraybox}

Above, I highlighted five choices, and a problem space of 3x3x3x3x3 combinations. You could easily enlarge this problem space, by allowing participants to make their own decisions. That would look like this.

\begin{svgraybox}
\textbf{Contract Negotiation 4:} Process in which a party that needs to have (one, more assignments, party's own choice of number of assignments) performed communicates with (one, more than one potential parties, the number of parties decided for that contract) that can execute (all, subset of these assignments, a minimum set by the asking party), until she has either agreed with (none, one, more than one party, number chosen by the asking party) on terms for collaboration towards the execution of (all, subset of these assignments, minimal set required for the asking party to accept the contract).
\end{svgraybox}

And this is only for a rough definition of a single term. You can imagine how this becomes complicated to think about, if we start looking at choices in two, three, or more realistically, innovation languages with a few dozen, and sometimes hundreds of terms. Even if you dedicated effort to the analysis of this kind, to each definition alone, you would still have to do this for interactions of choices which are in definitions of different terms, but depend on each other.

Same analysis is now applied to the third definition.

\begin{svgraybox}
\textbf{Contract Negotiation 3:} Process in which (two, more than two, number set by asking party) parties agree on terms of collaboration within (one, two, custom number chosen by asking party) calendar weeks (from the time the asking party requested a proposal from the supplying party, from the time the first supplying party responded to the request for proposal, either of these as requested by asking party).
\end{svgraybox}

Again, notice how a seemingly simple definition can cause headache, as the number of choices and options in each start to multiply.

Let's take a step back. What goes on in this analysis? I said I was identifying choices. In abstract terms, these are assignments of values to variables. I took easy examples in these definitions, because it is straightforward to think about, e.g., the number of participants as a variable that can take different numerical values, positive integers. But this is less apparent in the third definition, for the choice of trigger that starts the negotiation timer, where I had the choice "two calendar weeks from the time the asking party requested a proposal from the supplying party".

In simple terms, choices are all items in a definition that you could change. A choice in a definition is something that can or could be different.

A choice is not necessarily easy to identify in the definition. It could be straightforward, like in the cases I had above, or much harder, as in the following example.

\begin{svgraybox}
Shippable: A Shippable Unit (a.k.a. "Shippable" only) satisfies all the following conditions:
\begin{itemize}
	\item It involves one or more software functionality;
	\item It is represented, in the release management process, by exactly one User Story;
	\item It is released to UAT Environment;
	\item Prior to its release to UAT Environment, it has been tested by Vibe to ensure that this software functionality operates in accordance to its Product Specification;
	\item Its Product Specification has been approved, prior to the commitment of development resources to its implementation, i.e., prior to its User Story being introduced for the first time in a Sprint.
\end{itemize}
\end{svgraybox}

This comes from a terminology made by a software engineering team that was adopting for its own work style a general purpose development method. You might recognize that they were adopting and adapting for themselves, a variant of the Scrum method for software development. A first choice, which is hard to see in this definition, is that a shippable should involve pieces of functionality; it was only when the team started dedicating more resources to analysis tasks, which do not produce functionality, but do produce designs of future functionality, that they recognized that outcomes of these analyses cannot be presented as shippables, which skewed data on team's performance, and caused problems when performance assessments started being done by a third party. This was, in fact not a conscious choice, because the team was already - when the definition was made and approved - involved for months in developing functionality which was well specified up front, and there was no analysis effort to do. That option was not considered, and when no options are considered, this part of the Shippable definition seems, at that time, as immutable, it is not a choice at all, even if this turned out to be a mistake.

Another issue is the last item, where the Product Specification needs to be approved up front. This turned out to be a problem, because if we had to follow it rigorously, it would delay a number of decisions to change minor aspects of the product, decisions which we ended up taking anyway, without having the approved specification. However, when the definition was being made and approved, it seemed like the only option.

It can, in short, be hard to see that something in a definition is a choice in the first place. But even in the hard example above, choices were still items, parts of the definition.

It can be harder still, as when the definition itself is a choice, and taking another option would lead you to throw away that definition altogether.

In the Shippables example above, the notion of a shippable is associated, as are the notions of User Stories, Sprints, and so on, with the Scrum method for software development. The choice of going along with that method, of aligning to it, is in fact a choice like any other. If we wanted a different method, there may not be a notion of Shippable there at all.

This highlights the fact that you need to worry about choices that led to the introduction of the term to the innovation language, then about choices captured in the definition itself, and the choices that are implicit, which you only learn through experience.

Troubles do not end there. I have a rather different understanding of options and choices when I am analyzing definitions that I create, as opposed to those that others made. Indeed, a term's definition is simply an outcome of the thinking and communication that led to it (including one's specific experience and expertise accumulated up to that point). If you did not participate in, or do that thinking, then your analysis of choices in a definition will be only on the definition itself, and the rest of the innovation language.

But even if I had to make a definition of Shippable, I can still know little about Scrum, or software development altogether. Your analysis, in other words, inevitably leans on the knowledge you have.

And even if I knew a lot about software engineering methods, it would be more appropriate to know specifics of that team, who will be using the definition, than if my knowledge was generic.

The bottom line is that the less you know and the more indirect that knowledge is, the more difficult it is to anticipate the effects of choices in the definitions you make, or are handed over.

I mentioned three parameters so far, which play into your ability to predict the effects of choices in a definition. They are:
\begin{itemize}
	\item If your knowledge is direct or indirect; it is direct if you have expertise and experience on phenomena at hand; it is not if they are new to you, and much of what you can think of these phenomena is, thus, speculation based on, for example, analogy;
	\item How extensive your knowledge on the phenomenon is. The less extensive it is, the more likely it is that many choices in the definition will remain implicit to you;
	\item If you are an author of the definition, or only its recipient, in which case you did not participate in whichever explicit choices were made when that definition was designed.
\end{itemize}

These three dimensions are orthogonal. Your level on one has no relationship to your level on another. With this in mind, your worst case is being a recipient of a definition, while having low indirect knowledge of the phenomenon that the definition is about.

But what if choices in one definition are linked to choices in others? That requires a different view of an innovation language, presented in the rest of this book.


% Chapter bibliography
\printbibliography