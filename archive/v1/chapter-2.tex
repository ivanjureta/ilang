\chapter{Why analyse and design innovation languages?}
\label{c2}
\chaptermark{Why analyse and design innovation languages?}

\abstract{Even if an innovation language is made in any innovation process, as argued in the first chapter, should we invest effort in analyzing and designing it? To do analysis and/or design, we need to document the new terms throughout the innovation process, along with their definitions and relationships of new terms to old terms, that is, those terms mentioned in the new definitions, but whose own definitions are not made or changed in the innovation process. In this chapter, I argue that there are benefits to doing both analysis and design of an innovation language, during the underlying innovation process; both analysis and design are done with the aim to increase clarity, accuracy, and precision of the new terms, and expected benefits for the innovation team are less misunderstanding and faster identification of agreement and disagreement on new terms, and thus, on the content of the innovation process.}

\section{What should an innovation language do?}
\label{c2:s1}
I argued, in chapter \ref{c1}, that when you do innovation, you also have to extend the base language you use; by "base language", I meant what we call a natural language (e.g., English, French, Swahili, etc.) together with applicable technical languages.\footnote{I avoided in chapter \ref{c1} the question of what a technical language may be. I use that term to refer to terms and definitions considered as part of an established body of knowledge in a specific domain. A synonym would be ''domain language'';. I consider controlled languages as one of many kinds of technical languages \cite{Kittredge2012}: "A controlled language (CL) is a restricted version of a natural language which has been engineered to meet a special purpose, most often that of writing technical documentation for non-native speakers of the document language. A typical CL uses a well-defined subset of a language's grammar and lexicon, but adds the terminology needed in a technical domain.'' See also \cite{Kuhn2014} for a survey of controlled languages.}

In other words, people in an innovation team add new terms and definitions to communicate about new ideas and things they make; in addition, they define the new terms, and thereby connect them to the base language. 

I also argued that innovation teams always do this. It does not matter if they are aware in any way of the concept of innovation language. They do it because they need it: they need a new language to communicate about the new ideas they are creating, so that they can work together.

With so much innovation around [4], it looks like many know how to create and use innovation languages. At the same time, just because there is so much innovation, it makes sense to want to understand it even better, and to improve how it is done. 

Wanting to understand innovation begs the question of how to analyze innovation languages, in addition to all we already study about innovation. The need to improve how innovation is done begs a different kind of question, one of how to design innovation languages.

Both analysis and design are biased towards a specific purpose they should serve; the one I argued for in Chapter 1 had speed and relevance at its core: How can I be precise, accurate, and clear about unstable content of innovation? And if I were, Would this helps us reach consensus faster about the content made, changed, and otherwise manipulated during innovation?

A good innovation language, roughly speaking, would be one which is designed to be precise, accurate, and clear, and because it is such, it would help us progress through an innovation process faster, and in a relevant way, that is, towards outcomes that everyone involved judges to be constructive. 

\section{Who does the analysis and designs of an innovation language?}
\label{c2:s2}
It is important to distinguish analysis from design of an innovation language, not only because they are different tasks, but because of who can do which. 

Analysis can be done on innovation languages that an observer records by observing a team at work; there is no need for the team to invest resources to record the innovation language, prepare it for analysis, and do analysis. In short, analysis of an innovation language can be done without changing (or with minimal changes to) a team's innovation process. 

In contrast, the design of an innovation language changes the team's innovation process. It requires everyone involved to be aware of what an innovation language is, how it is made, and what their role is in making and maintaining that innovation language. 

It is an easier decision to do analysis than design. If someone wants to observe, and if the team is open to it, then why not analyze the innovation language? If you agreed with my arguments in Chapter 1, then an innovation language is something that is made anyway, and since it is specific to the team, it is a phenomenon worth looking into if you want to understand how they do innovation.

\section{Why design an innovation language?}
\label{c2:s3}
Why would a team want to design their innovation language? Why would they want to document and maintain an innovation language? 

In this Chapter, I argue in favor of design, and suggest how to do it; here is the outline of the argument.
\begin{enumerate}
	\item{\textit{Speed matters in innovation}, because the faster the innovation process, the faster we can get to observe the outcomes of innovation, see what worked and what failed, then make changes to get more desirable outcomes.}
	\item{\textit{If speed matters to the innovation team, then they better have precise, accurate, and clear communication.} Why so? Regardless of the specifics of their innovation process, they need to work together, which in turn means that they need to identify what they agree and disagree on; if that can be sped up, then it should speed up their overall process. Precise, accurate, and clear communication, even of new ideas, should help identify faster what they agree and disagree on.}
	\item{\textit{But what about relevance of communication?} Isn't that what we are looking for? If so, aren't precision, accuracy, and clarity less important? I argue that one should prefer precision, accuracy, and clarity to relevance in the following sense: if we have some limited amount of time to prepare communication before delivering it, then we should prefer to spend time making it precise, accurate, and clear, rather than relevant. This is because relevance is hard to predict, we can only really say if something was relevant after we communicated it; the issue there is that relevance of my communication with you is based on my judgment of how my communication related, after I did it, to the actions you took. But this isn't the case for precision, accuracy, and clarity, since I can judge these against my own standards, not yours. As I defined them in Chapter 1, I am talking about precision, accuracy, and clarity of communication relative to my ideas, not yours, whereas relevance is judged relative to your actions.}
	\item{What does precise, accurate, and clear communication have to do with the design of an innovation language, and especially with documenting that innovation language? An \textit{explicit, documented innovation language} helps make communication precise, accurate, and clear. There are three reasons for this. One, to document a term and its definition, you have to think through both of them, thus making you clean up at the very least the misunderstandings you have about them. Two, a documented definition is easier to access than one in someone's mind; I do not need to ask you for it, you do not need to make yourself available to answer; instead, I can look for it wherever it is documented. Three, as your definition is documented and accessible to everyone involved, this puts more pressure on me, to say if I disagree with it.}
	\item{\textit{However, it takes effort to make an innovation language explicit.} More effort means more time, yet the innovation process needs to take less time. This is called the \textit{innovation language paradox}.}
	\item{Defining new terms and phrases takes the most effort in making and maintaining an innovation language. I argue that this effort makes sense only if it consists of making a special kind of definitions, called \textit{plastic definitions}.}
\end{enumerate}

\section{Why speed matters in innovation?}
\label{c2:s4}
There are two reasons to want to do innovation faster.

Innovation takes time, and time has a cost. Trivially, if we can do the same or better in less time, then why not?

But the second reason is more interesting and important. 

Innovation is not simply the production of new ideas and things. They also have to be useful, or more precisely, prove useful after they have been produced through innovation. Of all new ideas you had today about the future, perhaps only some of them are about something that will be useful in that future. Any innovation process must rely on predictions of what will be useful in the future. 

Future is unknown, and so, predictions can be wrong.\footnote{Although it is obvious to say that the future is unknown, this is so only superficially; an eloquent recent and elaborate treatment is N. N. Taleb's triptych \cite{taleb2005fooled1,taleb2007black1,taleb2012antifragile1}. Moreover, it is not the same to try to predict the near or distant future, regardless of what we decide is near and far -- minutes and hours, or days and years).}

What does prediction have to do with speed? If your innovation process takes less time, then it takes less time for you to see if the new is also useful. So you can work with predictions of the near future, rather than distant future; you wait less to see if and how innovation is useful.

It should be clear why speed matters: you want to go from the idea, to seeing it executed quickly, so that you can see if it worked; and if you do this faster, then you invested less to get feedback, and change ideas, and eventually change that which you have been designing. You want to have short iterations from idea to its market.\footnote{See the following small sample of a large body of work on the need for speed in innovation \cite{teece1992,teece1997,von2005democratizing,christensen2013innovator,
ries2011leanstartup,christensen2016}.}

\section{How do precision, accuracy, and clarity relate to speed?}
\label{c2:s5}
Is it speed at any cost, and in any way? Speed in innovation can be destructive. The fastest way to go through many new ideas is, perhaps, to reject each one.\cite{dougherty2008interpretive,dyer2009innovators,martins2003building} That makes no sense.

What would be the ideal way to increase speed? There would be no innovation process at all. You would have the right new idea, the one that produces the most useful change; thus, you would have the right prediction. You would need to somehow perfectly know the future. That also makes no sense. 

How to approximate that ideal? 

We can cut uncertainty up, and say that some things about the future can be known today and we are confident will remain so in the future (laws of physics being a trivial example), others could be known (but we cannot spare the cost to learn them, and/or we do not know how), and some unknown number remains ignored, i.e., Rumsfeld's ''unknown unknowns'' \cite{wikipedia-known-knowns}. The ignored ones are by definition not knowable today; we might only find out ex post that we in fact ignored them. We were not even aware of questions which we do not know the answers to. Those we could know, are answers to questions we are aware of now.

Leaving aside what we later discover that we ignored, the apparent way to look to increase what can be known, \textit{is to get input from others}. Since your new ideas are predictions about an uncertain future they share, they may be of help as long as what they provide is what you did not know already.

This only provides additional support to the argument made earlier, that \textit{communication about new ideas is critical in innovation}.

But how is this related to speed? It should be, since if outcomes of innovation require agreement inside the team, it should matter to reach that agreement faster. Even if agreement does not matter, joint work and delegation means that everyone involved has similar ideas about what is new, why it may be useful, and so on.

Therefore, some kinds of communication should be better than others. Specifically, if communication should help speed up innovation, then it should not be misleading or cause misunderstanding. Either of these cases create a need for additional communication, to better explain and align on who meant what. That takes more time, not less.

It looks better, then, if communication is \textit{precise, accurate, and clear}.


\section{How is relevance related to speed, precision, accuracy, clarity?}
\label{c2:s6}
Is precise, accurate, and clear communication always also relevant?

There is a long history of research on relevance of information \cite{sperber1986relevance,sperber1995relevance,mizzaro1997relevance}. A persistent idea is that relevant and irrelevant information differ as follows: the former influences action in some constructive way, and the latter does not. "Constructive" here means that, as you try to solve a problem or, perhaps equivalently, achieve a goal, the relevant information pushes you towards actions which move you closer to the solution or goal. By "pushes" I mean that it may be necessary, but not sufficient in your decision-making. In a setup where you collaborate with others to solve problems and achieve goals, as you usually would do in innovation, constructive ends up meaning that you are progressing towards consensus on the prediction, that the outcome of your effort will be useful in the future.

Nothing guarantees that precise, accurate, and clear communication is also always relevant. Even if you and I needed to work together, we could be leading two monologues, both precise, accurate, and clear, but mine about ideas which you see in no way related to your decision and action, and vice versa.

Let's take a simple example. If I should spend some time outdoors tomorrow at my present location, let's say its Frankfurt, then it is more relevant for me to know the weather forecast for Frankfurt, rather than the forecast at some remote location like Vancouver. You might be giving me a precise, accurate, and clear weather forecast for Vancouver; and I might be recognizing its precision, accuracy, and clarity, but if it makes no difference to me, if it has no effect on the options, criteria, and eventually my choice on how to act, then it is irrelevant to me at that time.

How can I then make sure that what I'm communicating, even if precise, accurate, and clear, is also relevant to you? 

For that, I need to make assumptions about what may be relevant to you, which means that I have to make assumptions about problems you are trying to solve, what you know and ignore about these problems, what matters to you, what preferences you have, and alternative solutions you may know and ignore, and much more. I need to think about where you are, when, with whom, how you perceive where you are, when, and with whom, why and how that matters to you, why something matters and other does not, and so on. I might also want to think, not only about what you may be observing, thinking, remembering, but also what you may have observed, thought, remembered in the past, as well as what you may expect of the future, and how you predict it to be. 

Ideally, I would work on facts, not assumptions. That is, I would be able to know in some perfect way your state of mind. That being elusive, the relevance of my communication critically depends on my predictions being right about your state of mind. 

This gets us to an important step in this argument. How much time do you spend speculating what might be relevant to me, before you communicate precisely, accurately, and clearly? How do you split the effort you are able to invest in making something relevant, and to make it precise, accurate, and clear? Is it 50/50? Do you put more into relevance? 

You can decide to communicate later, so that you have more time to speculate about relevance. Or, you can decide to communicate sooner, and spend less time worrying if and how it is relevant to me. 

This is where speed of innovation comes in, and connects precision, accuracy, and clarity to relevance: if I want to spend less time making assumptions about what is or is not relevant to you, then I better communicate precisely, accurately, and clearly, wait for you to decide and act, and then draw my own conclusions about if and how my communication was relevant to you. Rather than spend more on speculating about relevance, my argument is that precision, accuracy, and clarity should be preferred - they let us converge through interactions to agreement or disagreement.

In other words, if you want speed, then you should worry less about the relevance of your communication to me before you actually communicate, and worry more about precision, accuracy, and clarity. If you spend more time thinking about relevance, and therefore communicate later, you will find, in my actions and their outcomes, either evidence for relevance or irrelevance of what you communicated. If you find irrelevance, then you wasted the time spent focusing purely on relevance.

In short, relevance of what is being communicated matters to speed in innovation, but less than precision, accuracy, and clarity. 

This only makes sense in a corrective, rather than preventive approach: \textit{provide precise, accurate, and clear input, then correct if there is evidence for irrelevance}; instead of spending much time trying to anticipate what will or not be relevant, before communicating.

The corrective and preventive approaches distribute responsibility in different ways. The corrective approach asks you to be precise, accurate, and clear about your ideas in innovation, and it is the responsibility of others to do the same with theirs; because of that, you are also responsible to voice disagreement and irrelevance with others' ideas, rather than expect them to be responsible for making their communication relevant for you.

Let's continue the weather forecast example. Suppose we are talking over the phone, and you do not know that my current location is place Frankfurt. The corrective approach is for you to give me the forecast of Vancouver, even without checking if that is my location; the corrective part is that I should stop you and ask about the forecast for Frankfurt. The preventive approach would be, for example, that you spend time trying to identify my location, then choose which forecast you give me. 

This is not a minor point in the overall approach. \textit{It is not that relevance is, in some sense, irrelevant.} If it were, this constructive approach easily turns into a joke, for I would adhere to it also every time I was to throw at you a bunch of very precise, clear, and accurate ideas, but completely disconnected from the situation we are in, and the goals, constraints, values, and so on, which each of us may have and assume for the other. Relevance must be there, and the claim here is not that it should be disregarded altogether. The claim is only that given a minute more to spend, trying to improve communication during an ongoing collaboration we have, is better spent on precision, accuracy, and clarity, since you have more control over that, instead of spending it on relevance.

This aspect of the overall approach, an attention to precision, accuracy, and clarity, and a preference for it over improving relevance that's already there, also rests on a certain set of values. It assumes that \textit{we want speed in innovation}. It also assumes that you and I both have a voice in the innovation process.


\section{Why should an innovation language be explicit and documented?}
\label{c2:s7}
What does precise, accurate, and clear communication have to do with the design of an innovation language, and especially with documenting that innovation language? An explicit, documented innovation language helps make communication precise, accurate, and clear. 

There are three reasons for this. One, to document a term and its definition, you have to think through both of them, thus making you clean up at the very least the misunderstandings you have about them. Two, a documented definition is easier to access than one in someone's mind; I do not need to ask you for it, you do not need to make yourself available to answer; instead, I can look for it wherever it is documented. Three, as your definition is documented and accessible to everyone involved, this shifts responsibility to them, to voice disagreement with or suggest improvements to your definitions, that is, explanations of your ideas.


\section{How does all this result in a design approach?}
\label{c2:s8}
There is an approach to design innovation languages, which caters to the need for speed and relevance, and lets you use your preferred innovation process or method.

The approach is simple: as soon as you need to get others involved in the innovation process, to help you move you from your new ideas to something more convincing (for them and for you), then \textit{your team should define the key terms you all are using, in an explicit, clear, accurate, and precise way, and revise these definitions throughout their innovation process}. Doing so, the team is making its innovation language explicit, and documents it for access to everyone involved.

\textit{Explicit} means in a format accessible to others without requiring the attention of its author; for example, written in a document or database. 

\textit{Clear} means there are no multiple interpretations, especially conflicting; it means avoiding ambiguity, vagueness, synonymy. 

\textit{Accurate} means being true to, or exact with regards to the ideas that it should say something about. 

Finally, \textit{precise} means being specific, or close to, or focused on only the ideas to communicate about, rather than other things and ideas. 

The approach, in short, is that the team should make explicit and document their innovation language, maintain and revise it throughout the innovation process.


% Chapter bibliography
\printbibliography