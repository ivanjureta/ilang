\chapter{When innovating, which languages do we speak?}
\label{c1}
\chaptermark{When innovating, which languages do we speak?}

\abstract{What are innovation languages? Who creates them, and why? Why is it interesting to analyze and improve innovation languages? I argue in this chapter that teams tasked with innovation -- the invention of new ideas and things, and the transformation of these into something useful for others -- always create new terms in order to communicate about, and coordinate when working on new ideas. I define an innovation language as a combination of new terms, definitions of new terms, and relationships of these definitions to ''old'' or established concepts. I argue that the analysis of an innovation language is relevant if we are interested in how innovation is done and in how to improve innovation processes.}

\section{Say that again?}
\label{c1:s1}
\begin{quote}
We're gonna use the best pointing device in the world. We're gonna use a pointing device that we're all born with - we're born with ten of them. We're gonna use our fingers.
We're gonna touch this with our fingers. And we have invented a new technology called multi-touch, which is phenomenal.
It works like magic.'' \cite{Wright2015a}
\end{quote}

The quote is from Steve Jobs' 2007 MacWorld keynote. He was introducing the first generation iPhone. This was the official start of a product line which, more than a decade later continues to shape communication, work, and entertainment of hundreds of millions of people. 

At that time, when he was doing this keynote, only a small number had some notion of what he might have meant by "multi-touch", "swipe", ''home button'', ''you can touch your music''. Most of the audience was seeing the first iPhone for the first time. As he was demoing it, it became clearer what he meant by multi-touch, swiping, and so on. 

Why did he use these new terms? Why couldn't he use plain English, so to speak? 

The reason should be obvious: he was showing something new. And so, isn't it also evident that he had to have a new language to talk about it? It certainly seems right that new ideas and things merit new names, and as a result, we must extend whichever language we use with these new terms. 

What makes these new languages? What role do they play in innovation processes? How do we create them? How can we analyze them, to help us progress in innovation? The book focuses on these questions.

\section{How do we talk about new ideas?}
\label{c1-s2}
This book is motivated by a simple observation: \textit{when people do innovation, they create new words and phrases to talk about new ideas and things that they are creating}. 

While I was neither an observer, nor participant in innovation processes that led to the first iPhone models, I did have such roles in many others over the last decade in various companies. Each time, teams which did innovation, also made and spoke their own new language, pieced together from ordinary language, relevant technical languages, and new terms. These new terms were needed to communicate about new ideas created, changed, and thrown away during the team's innovation process. 

If we take that observation seriously, then it must be possible to learn something more about how innovation is done, and how it could be done, by looking more closely at the language that teams use to communicate about, and coordinate around new ideas. 

We should be able to understand and change innovation processes if we take a closer look at these innovation languages: extensions to ordinary and technical languages, that teams responsible for innovation create as the work through new ideas. By technical language, I mean a set of terms and definitions which may have been new in the past, but are now commonly used when dealing with specific kinds of problems and tasks, in the specialities of medicine, engineering, and many other fields of research and industry.

\section{How many innovation languages are there?}
\label{c1-s3}
An innovation language is an artefact of an innovation process, in the sense of being something made and changed throughout that process. 

It does not matter how exactly it was made: it could have been accidental or intentional, created in methodical or messy ways. But, my claim and premise is that \textit{it is made every time innovation is done, as long as innovation required collaboration}; it certainly did in all innovation processes I was involved in. 

You can test this yourself, that is, if people use new terms to talk about new ideas and things, and from there, if there are innovation languages. 

If you are doing innovation, ask someone from outside your team to join a typical team meeting, and ask them how much of what was said they understood. Was there any word or phrase that was new to them, and if so, are these not already in an established technical language? If not, they are part of your team's innovation language. Alternatively, were there old words and phrases, but which are used with a new meaning?

If you are not a participant, ask to sit in a few meetings of the innovation team. Ask yourself these same questions. There will be ideas you understood roughly just like everyone else, and much of that would be communicated through the base language, the common language such as English, French, or other, that you and they all speak. Then, there will be established technical terms and phrases, which rely on, and convey information and knowledge that both them and you (as well as others) have, or can have access to. But then, there will be new words and phrases, with their own definitions, which do not exist as such outside that innovation team - this is what makes a big part their innovation language (but not all of it, as I will explain later); if these add-ons, to the language they use anyway, is lacking, how could they possibly be using old language to talk about new ideas and things?

\section{Why are there innovation languages?}
\label{c1-s4}
An innovation language is made for \textit{two reasons} at least; one, it is a tool for communication between those involved, and therefore \textit{critical for their coordination}, and in turn, it must influence how innovation processes unfold; two, it reflects these people's intentions, assumptions, and knowledge about the new ideas and things they work on, and relationships with old ideas and things, \textit{telling us about what is, and is not new in the outcomes of innovation}.

To work together, we have to coordinate, and we can only coordinate through communication. This, in turn, forces us to create terms with which we can refer, or talk about, ideas and things. For most obvious things, and many ideas which are, so to speak common knowledge, we already have natural language and its broad terminology.

But \textit{how can we use old terms to refer to something new}? We can use them: either by redefining them, or else it will be confusing. In the former case, when we \textit{redefine}, old terms remain syntactically the same, but their meaning has changed, and so we are mistaken to think that we are still using the same language. Either way, coordination around new ideas and making new things together will inevitably lead those involved to create extensions of the underlying natural language (English, French, and so on), that they use for communication. This extension is called an innovation language, and its emergence, design, and analysis merit attention.

\section{Why are innovation languages interesting?}
\label{c1-s5}
We create innovation languages when we do innovation, and studying them can help us understand and improve how we do innovation. If this is self-evident to you, then you might want to skip the rest of this chapter.

But it is not a simple claim to make, as it touches on many old and much-discussed topics in philosophy, linguistics, and innovation. The rest of this chapter elaborates the argument, in the following steps.
\begin{enumerate}
	\item \textit{New ideas, just like old ones, are inaccessible to others}. I can access my ideas in some way (whichever the specific biological, psychological or other mechanisms involved), but I cannot access yours \textit{in the same way}; the way you can access yours is not the way for you to access mine.
	\item Since others' ideas are inaccessible, \textit{communication is about ideas, not of ideas}. I cannot be certain that you have the same ideas as I do, or that my are the same as yours.
	\item Communication being \textit{about} ideas, the only way to ascertain that your and my ideas are aligned, is that my actions signal to you my alignment, and that yours signal yours to me. \textit{Actions and outcomes of actions are accessible, and are evidence for alignment in ideas}.
	\item To align on ideas, we need to communicate about them in such ways that our communication leads to actions and outcomes which signal to us something about presence or absence of that alignment. Therefore, \textit{if we need to get aligned on new ideas, as we do need to do in innovation, we need new terms for these new ideas}.
	\item But since others' new ideas, just like their old ones, are inaccessible to you, you cannot simply and only add new terms to the natural language you are using in communication (e.g., English, French, etc.). If we were only to add new terms, their meaning would not be clear enough to others, even if my new terms are clear to me. We also need to establish reference between the new terms and these ideas in a way which others can access, which makes it necessary to produce definitions of these new terms, to make it clear what they refer to.
	\item By this point, we have agreed that coordination during innovation requires new terms \textit{and} their definitions. Those two are still \textit{not} an innovation language. This is because they are not standalone: definitions will use a mix of old and new terms, i.e., a definition leans on the underlying natural language. It follows that understanding the new terms requires not only their definitions, but also the relationships\textit{ of those definitions to established terms}. An innovation language will include, therefore: the set of new terms, the definitions of these new terms, and the relationships of these definitions to the terms in the underlying natural language.
	\item From all above, it follows that a team invested in innovation has to, and will create an innovation language. It will be one of many artefacts created during innovation. Since innovation involves uncertainty, and there is thus a possibility for it to be unsuccessful, we get to the problem which motivates this book, and in general motivates paying attention to innovation languages: \textit{How can we create, and analyze to improve, the innovation language so that it helps the innovation process?}
\end{enumerate}

\section{Are ideas accessible, and why does that matter?}
\label{c1-s6}
New ideas, like all thoughts, are by definition abstract - something "[existing] in thought or as an idea but not having a physical or concrete existence" \cite{def-abstract}.

Ideas are from the mind and in the mind. You do not "have" them because you take them from somewhere else, but because you make them. Despite what Plato was arguing, I never saw an idea outside the mind (maybe I did, just could not recognize it), only descriptions, representations, i.e., only some imperfect proxies of what someone was thinking, and I see them only because that person decided to do something in relation to the ideas she was having.

The point is that ideas are not representations of something, they are whatever they are, but not a reflection of something that exists independently of the individual who has them. They are not, at least in this book, Plato's "eternally existing pattern[s] of which individual things in any class are imperfect copies" \cite{def-idea,Kraut2017}. 

Importantly, if ideas come from one mind, they are "in" that single mind, whatever that "in" may mean exactly in the language of physics, chemistry, neurology, and so on. They are best known only to that mind. This is simply to underline that what I know, especially if it is something new I just came up with, is certainly better known to me, than it is to you. I mean "known to me" in the colloquial sense. There is a lot to say about the epistemology of these new ideas, but it is clear that they are not knowledge, i.e., not justified true belief.

For example, if I'm imagining now, how creatures on a distant planet may look like, even if we agree that the planet is HD 156668 b \cite{wikipedia-hd156668b}, you don't really know what I'm thinking. Unless we spoke about it, of course.


\section{Do we communicate ideas or about ideas?}
\label{c1-s7}
Even if we communicate, words and language are not for copying and pasting thoughts as-is between minds. Sperber puts this nicely on two separate occasions.

\begin{quote}
"Just as the human mind is not a blank slate on which culture would somehow imprint its content, the communication process is not a xerox machine copying contents from one mind to another. 

[Mechanisms] of human communication [are] much more complex and interesting than is generally assumed, and much less preservative and replicative and more constructive than one might think: understanding involves a lot of construction, and not just reconstruction, and very little by way of simple replication." \cite{Sperber2005}
\end{quote}

\begin{quote}"[Communication] is not a replication system. When I communicate to you, you don’t get in your mind a copy of my meaning. You’ll transform it into something else. You extract from it what’s relevant to you. It involves both understanding and misunderstanding. But even if you’re understanding me perfectly, your goal will not be to have a copy of what was in my mind, it will be to extract from it some thoughts of yours which will have been usefully informed by mine, but which will be relevant to you.'' \cite{Sperber2017}
\end{quote}

In short, you can't know my thoughts like I do. I can't know yours. And I cannot copy my ideas to you, nor can you copy yours to me. Thinking otherwise is odd. Could you, or anyone really claim to know exactly what Bruce Sprinsgsteen meant when he wrote the following part of his "Lost in the flood"?

\begin{quote}
"The ragamuffin gunner is returnin' home like a hungry runaway
He walks through town all alone--"He must be from the fort," he hears the high school girls say
His countryside's burnin' with wolfman fairies dressed in drag for homicide
The hit-and-run plead sanctuary, 'neath a holy stone they hide
They're breakin' beams and crosses with a spastic's reelin' perfection
Nuns run bald through Vatican halls, pregnant, pleadin' immaculate conception
And everybody's wrecked on Main Street from drinking unholy blood
Sticker smiles sweet as Gunner breathes deep, his ankles caked in mud
And I said, 'Hey, gunner man, that's qucksand, that's quicksand, that ain't mud
Have you thrown your senses to the war, or did you lose them in the flood?'" \cite{wikipedia-lost-in-the-flood}
\end{quote}

Your ideas about his ideas are typically called your interpretations of the meaning he intended. Aren't they speculations? You are trying to make sense, using what you think, feel, intend, and so on, at some time (now?) about what someone else was thinking, feeling, intending, a long time ago - sometime before or in 1973 for that song. Best of all, you do this only on the basis of signals that you get from communication with him. In the case of the song, you listen to it, and that's most of us have, unless you have a chance to talk with Bruce. 

This struggle with meaning should be obvious. Don't you know the feeling of trying to get someone to understand something? You often think you failed: they are not doing what you thought they would, and, you think, it must be that you were not clear and convincing.

For that song, the struggle with meaning is not a major issue \footnote{Many heard that song, but only a few go so far as to share their interpretations and claim to be right. For them, the stakes are higher \cite{songmeanings-lost-in-the-flood}.}. There is not much at stake.


\section{When does alignment matter in innovation?}
\label{c1-s8}
So far, I noted that ideas are personal and inaccessible. Communication is about ideas, not of ideas. In contrast, actions, including communication, and at least some of their outcomes are accessible to others. It is only through your interpretation of these actions and outcomes, that you judge if others did as if their ideas are aligned with yours.

Now, if I judge that some ideas only matter to me, and that actions that I expect they will lead me to, only affect me, and finally, I see no need to try to convince anyone else in them, then it is good enough that only I know them; you don't need to bother.\footnote{It is more complicated, since I might hold, and not communicate, ideas which might lead me to destructive actions.}

When does that alignment matter? In most cases, or at least any interesting case, something is at stake in our interactions with others.

In innovation, two things are at stake: how the process unfolds, and its outcomes. As soon as your new ideas require you to get help from others, your and their communication and actions start to matter. 

Suppose that I am having ideas about a new kind of bicycle, and I want you to help me make it, or at least up to a prototype. I'm having these ideas, about what the bicycle should look like, run like, behave like, I'm thinking of all sorts of features, properties, characteristics, qualities I want it to have, a booming set of intertwined thoughts. But I also do not know all of it; there's much I haven't considered, and I haven't made up my mind about each one I did. I'm still wondering about the steering, braking, colors, and so on. That is where you come in. I ask you for help. What's next for us? What is the immediate next thing we should do?

Before we start investing more of our time in the bicycle, probably even before you decide if you want to help, I need you to have an idea about the bicycle. But I cannot just copy and paste my ideas into your mind. I need to convey, to communicate about my ideas to you, to help shape your ideas about it, in ways which align them with mine. To be more precise, which will align your actions with those I think I need you to have, so that I will believe that you are helping me. In short, and abstracting all those complexities away, we need to start aligning. We need to get on the same page. 

Now, suppose we did get on the same page, hopes are up, we are working through the design of the bicycle. It turns out we need more people. They need to be on the same page too, at least about the things we decided by then, so that they can help us out.


\section{Why are new terms alone, not enough to convey new ideas?}
\label{c1-s9}
Through collaboration with you, my original ideas will change, yours too, and all of those will get changed as new people come on board, as they make their contribution in that innovation process. Remember, their ideas are theirs, and we cannot simply transfer, as-is, our ideas to them.

While it is certainly desirable to change ideas as we progress through innovation, we also want to make sure these ideas improve upon previous ones, whatever our exact criteria to judge improvement. In other words, we want learning to happen as we go along, and we want that learning to have a direction. This, in turn, requires  coordinated action.

That is where new terms come in. We need names for new ideas and things we make in the innovation process, so that we can communicate about them in order to coordinate actions. We need to refer to them.

There is a lot to say about the reference relation, one that supposedly exists between words, phrases, or statements, and ideas. At this point, all we need to agree on, is that reference is somehow close to the idea of aboutness, and so, that a statement refers to an idea is somehow similar enough to that statement being about the idea. Note, clearly, that I'm simply saying that reference is aboutness, which is not much better than where I started. In other words, perhaps, I'm avoiding to commit on the exact meaning of reference, or what exactly the reference relation may be. It just does not matter much at this point, my argument remains unchanged if you adhere to any of widely discussed notions of reference.\footnote{See, and pay special attention to Section 5 in \cite{Michaelson2019}.}

Therefore, \textit{we cannot only add names of new ideas and things to the natural language and technical languages we already use}. 

We must also \textit{define the new terms, i.e., explain what they refer to}. How can we otherwise explain them to others, to make them understand what these terms refer to? In other words, how can I explain, describe, or do whatever I need to do, to make sure you do not mistake what I am telling you, with something I did not want to say?


\section{Are new terms and their definitions enough?}
\label{c1-s10}
Having only the new terms and their definitions is not enough. A definition of a new term cannot only mention other new terms - it also mentions those with an established meaning, those that are part of the ordinary and technical languages used in the team. A definition of a new term leans on old and new terms.

Understanding new terms, then, not only requires the new terms and their definitions, but also being clear about the relationships between the new and established, or old terms. 

An innovation language is, then, made of 
\begin{itemize}
	\item{new terms,}
	\item{their definitions, and }
	\item{relationships of these definitions to terms in the underlying natural language and any technical language that may be relevant.}
\end{itemize}

\section{Why put more thought into innovation languages?}
\label{c1-s11}
We make many artefacts in an innovation process. There will be documents written, technical drawings made, messages exchanged, calls made and recorded, scale models made, and so on. We do this towards consensus on what needs to be done next, be it more design, prototyping, development, manufacturing, or other.

These artefacts act as a record, however partial and imperfect, of the actions we have taken, themselves the outcome of our joint decisions and design interactions, them, in turn, guided by the ideas we had and have, shaped by context, and communication and action of others. 

A substantial amount of thought and effort, both in academic research and perhaps more in business-oriented research and development, has gone into the design, analysis, and testing of the right structures of these artefacts. The question there is What template should an innovation team use, when solving this or that problem that regularly pops up as they do innovation? 

There is a lot of sense in searching for, designing, and testing such templates, be they for processes and tasks, or for keeping records of what was done, and for planning what to do next. 

The basic motive is that, if you learn the rules once, you don't need to learn them again every next project where they are used, for example, when you look at the next blueprint made according to the same rules. From everything that you need to learn to reach consensus, on a given project, at least you know the rules for reading and making some of the artefacts which record design decisions. You might choose to disagree about the content, but at least you know where to look for it, and how to read it. You know the rules that define the structure of the artefact.

When you study architecture, industrial design, civil engineering, software engineering, management, or one of many others, part of the effort goes into learning rules on how to read and produce such artefacts, so that they fit standardized or well-established practices. 

But even if you know the rules for making and reading the artefact, you still need to know how to read its contents. If you give me two architectural blueprints, one for a family home, another for a hospital, my interpretation of these will depend on the extent to which I was exposed to ideas about the former, and about the latter. It will, evidently, also depend on the knowledge which applies to both, say, something I might have studied as a student of architecture. But each project is different. 

The problem I am interested in, and which motivates looking more closely at innovation languages, happens before we make any artefacts, and persists while we do the design of the bicycle I mentioned earlier, and lives as long as we remain interested in improving the design. 

The problem happens, for the first time, as soon as I need your help, and need to explain to you what it is that I'm thinking to design with you. 

If you think about this in terms of artefacts made during design, blueprints, specifications, documentation, and so on, the problem is not to select or make the right structure for them, but what to fill them with - the content, substance that these artefacts need to contain. 

This problem is particularly interesting when doing design, or innovation more generally, because the ideas will be new to you.\footnote{Design and innovation are different kinds of processes, but I have never participated in design which did not involve some degree of invention, and which, if these outcomes proved useful, was not called innovation. For the record, Oxford English Dictionaries define the three as follows. To design is to "[decide] upon the look and functioning of (a building, garment, or other object), by making a detailed drawing of it" \cite{def-design}. To innovate is to "[make] changes in something established, especially by introducing new methods, ideas, or products" \cite{def-innovate}. To invent is to "[create] or design (something that has not existed before); be the originator of" \cite{def-invent}. It looks safe to say that design and invention go hand in hand, and if their outcomes are useful, they are in retrospect called innovation.} They would not be my references or allusions to something you and I have learned in the past, in roughly the same way, e.g., arithmetic, geometry, especially if it is precisely defined and the definition is widely shared. 

Instead, these ideas will be something that you need to learn from me about. At the same time, since these ideas will not be definite, polished, and complete, you may have yours, which come to change mine, as we collaborate during design, inventing and innovating together.

The problem can be rephrased as the following question, which is the central motive for being interested in innovation languages in the first place:

In innovation, when ideas are new, incomplete, unfinished, brittle, how can I be precise, accurate, and clear about them, so that I can convey them to you, that we can reach consensus on what innovation is about, and that you can then help me replace these ideas with better ones? 

Phrased otherwise, How can I be precise, accurate, and clear about unstable content of innovation? And if I were, Would this helps us reach consensus faster?

This is not the classical problem of what meaning is and is not, in general or of specific notions. It is not about how to convey any kind of ideas, in any situation. It is about new ideas, and they are different from stable ones. In many situations, for stable ideas, it is simply not place or time to debate them much, since they proved to work in the past. If they worked for you, and you have no effort to invest in trying to improve them, then they will not change. This is not to say that they will remain stable forever, but their stability is testament to their past relevance. 

For most children, when they start to learn natural numbers, it is not the point to debate if Peano axioms are good enough a definition; they can do plenty without bothering with it (or any simpler definition they were given). But innovation is a setting and a process, in which ideas need to be short-lived. They are by definition changeable, since the earliest ones are unlikely to be the best; not because they were perfect, but killed by premature criticism, but because they have come up at a time when the person who had them, simply knew less about the problems, situations, and people these ideas should relate to. 

This is also \textit{not} a question of how you arrange the innovation process. Whichever activities it has, however these may be arranged over time, they operate on the content, the substance of innovation, those ideas that each of us has, and which we need to confront, rethink, replace, until we have converged to a consensus that we should take the next step, of taking action which is no longer about the change of ideas, but about the bending of the world to these ideas, be it only by stopping the design of the bicycle I mentioned above, and arranging its manufacturing. 

To summarize, new ideas are necessary for invention, and therefore, for innovation. But new ideas are in individual minds, and cannot, as-is, be carried between them. So we cannot work together unless we communicate these ideas. What's more, we cannot communicate these ideas; we can communicate about them. If people need to collaborate in invention and innovation, their communication will lead each of them to keep or replace her own ideas. 

How can we do this faster while remaining relevant, this process of aligning, diverging, replacing, then aligning again, diverging again, replacing again, and so on? 

One part of the answer is about methods, processes, tasks, that is, how we do invention and innovation; another is how you structure the outputs processes, which is about the structure of artefacts you make. The third part, intertwined with the two others, is the content that the process manipulates, and that the artefacts structure and document, and what needs to be done about that content, for the sake of speed and relevance. 

And as soon as you think about that content, and because of the fact that it needs to be communicated, then you also need to look at the language which is used to express, convey, represent that content. I look in this book at content, and it is only because of arguing how that content needs to be, that I touch on the process and artefacts. 

Caution is advised, since the content of innovation is not tangible. It is not text, it is not drawings, it is not specifications, and it is not the innovation process; it is precisely the ideas of participants, each one's hidden away and inaccessible to others; they are the content, and the only way to do something about them is through material proxies, innovation languages being one kind of those.


% Chapter bibliography
\printbibliography