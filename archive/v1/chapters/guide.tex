\chapter{How to Make and Improve an Innovation Glossary}
\label{c:practical_guide}
%\chaptermark{}

%\abstract{}

%\section{}
%\label{}
\section{Basic Structure of a Glossary}
A glossary is made of definitions. A definition has two parts: 
\begin{itemize}
        \item \textit{definiendum} (plural is \textit{definienda}), also known as \textit{term}, which is a word or phrase that you use to name the thing or idea you are defining, and
        \item \textit{definiens} (\textit{definientia}) is an account about the things or ideas you are defining; it should ideally be such that, if you know the definiens, and you see an actual instance of the thing it defines, or you think the idea it defines, then you will know how to name it, that is, you will know that this is something to call by the term.
\end{itemize}

What the definition does, is state states the equivalence of these parts: if you and I agree on a definition, then we agree that, whenever either of us mentions the term, we also mean the definiens we agreed for it, not something else. 

To make a glossary, then, we have to make definitions.

\section{Value of Disagreement}
A definition exists within a community of people. There are definitions which will work for large communities; if you look at an ordinary encyclopedia, it will include definitions that are widely accepted.

The reason I mention this, is that the definition is \textit{local} to that community. Again, if the community is large, then there are not many interesting implications to saying that a definition is local. But the smaller the community, the harder it is to say that the definition they take for granted will be agreed on by others. 

A definition, then, comes from consensus within a community. In practical terms, this consensus on a definition is useful primarily because decisions and actions that people in that community take, should not contradict that definition.

Here's a blunt example: if you and I agree that ''public transportation'' is defined as ''buses, trains, subways, and other forms of transportation that charge set fares, run on fixed routes, and are available to the public'', then I shouldn't think that I was using public transportation when you drove me to work in your car.

Most people get in touch with some form of public transportation sooner or later, and independently from the experience itself, they encounter the term, and get an explanation for it at an early age. 

Agreement on a definition is part of the bigger picture, which is the \textit{coordination} within a community. We can work together if we have an agreement on words we use and the meaning we have for them. Even if nuances will inevitably vary -- I know less about public transportation than a driver of a public bus -- I know I should pay for using public transportation, and she knows I know I have to do it, for example. Even if her and I never needed to confirm our mutual agreement on a general definition of public transportation, our actions will, or will not be consistent with that, and many other definitions that are accepted, even if implicitly, in a community.

But even these general definitions came about either because there is a potential for misunderstanding, or there was actual misunderstanding, which mattered only so far as it led us to actions which somehow clashed with whatever may be the usual functioning of that community. 

Disagreement has that value of forcing us to make and change definitions, especially on things and ideas which frequently play a role in our individual and joint decision-making and action. 

There is no need for glossaries if people in a community agree, and coordinate in such a way that their individual actions neither clash with those of others, nor lead to outcomes which are undesirable to others. 

In short, a glossary is a tool that helps reach, or formalizes agreement. But it comes from potential or actual disagreement, which brings us to the interesting topic of glossaries in innovation processes.

\section{Disagreement and Innovation}
Let's say I told you that I have an idea for how to fly faster to the Moon. You ask me to elaborate, and I go on to explain that I've come up with a new engine which takes rainbows for fuel. ''But you cannot guarantee continuous fuel supply'' you say, ''and that's one of many problems you probably haven't thought of''. True; I'd have to run around chasing storms to get all the rainbow I need. I couldn't guarantee a regular departure schedule to my customers.

New ideas are rarely great on arrival. They need discussion, testing, refinement, revision. The push for any of this comes from disagreement. You are doubting I can make fuel from rainbows, so what shall I do? Either ignore you from now on, which means cease collaboration, or do something more about that, and thereby change what I was thinking about rainbows as fuel in the first place.











The acceptance of a definition is a matter of agreement. This is important here because the innovation I am interested in involves more than a single person. A team will be the smallest community which needs to agree on a definition. 

Agreement matters, because it is disagreement around a definition which makes us change it. If we have a new term, and a definition for it, then any change of that definition, and so of the community's shared reading of it, will hinge on agreement.




The definition describes and the definition describes that thing in such a way that, if you know the definition, you could unequivocally say if something you see should be called by the term. So if I have a definiton of the term ''elephant'', that definition should be such that, if I see an animal, I will know without a doubth if I should call it ''elephant'' or not.

A glossary is a set of terms. No two terms in it are the same, no single term can have more than one definition, and no single definition can apply to more than one term. Anything else will only be confusing.

An \textit{innovation glossary} is a special kind of glossary. In it, each term must be itself new, or in any case must have a definition which identifies \textit{new} indeas and, or things. Novelty of the term, that is, the fact that the name of things and ideas is new, is not that important. It is, however, important that the definitin points to new things and ideas. Take these two definitions of ''grizzly beat'':
\begin{itemize}
    \item ''The grizzly bear, also known as the North American brown bear, is a large population of the brown bear inhabiting North America.'' \cite{wiki-grizzly-bear}
    \item ''Software to make, which will be used to record sightings of grizzly bears.''
\end{itemize}
Same term, ''grizzly bear'', but two completely different definitions, which point you to different things. The second one refers to software which is new: it needs to be built. The first one cannot be in an innovation glossary.

In an innovation glossary, \textit{all} terms have definitions specific to the problem or topic you are working on, and all definitions are new. While names of terms can come from anywhere you chose, these terms will be part of an innovation glossary only if their definitions were newly made for the purpose of addressing the problem, or working on the topic you are interested in. This is why it is called an \textit{innovation} glossary -- definitions will have come from an innovation process you are involved in, where novelty requires new terms and definitions, or sticking new definitions on old terms. 

My experience has been that an innovation glossary is never made in one step, and is never done. Making it in a single step would mean that you write all new terms and their definitions, and that's it; in other words, you know it all up front, which also means that the glossary was not made as you progressed through your innovation process, but was documented at its end. As we'll see below, you start from one or two new terms, write a rough version of their definitions, this points you to other new terms to define, and you iterate as long as resources allow you. It stops inevitably, but not because the glossary was perfect; rather, the problem may be consdered solved, or we may abandon its resolution, or for any other reason that makes it irrelevant to invest in improving the innovation glossary.

An innovation glossary is made and improved as follows:
\begin{enumerate}
    \item Start an innovation glossary from scratch in the following two steps:
        \begin{enumerate}
                \item Identify one or more frequently used terms in your innovation process.
                \item For each frequently used term, find a general definition for it, check if that definition fits how it is used in your innovation process, and if usage differs from the definition, then the term requires a new definition and you should add the term to the innovation glossary.
        \end{enumerate}
    \item Improve the innovation glossary by repeating the following steps: 
        \begin{enumerate}
            \item For each term:
                \begin{enumerate} 
                    \item Verify if the term's definition fits how the term is used in the innovation process, and if not, update the definition.
                    \item Identify, in the term's definition, all other terms which are mentioned, and which are \textit{not} defined in the innovation glossary; for each of these terms, find a general definition, check if the term's usage in the innovation process fits its general definition, and if not, add the term to the innovation glossary and make a new definition for it.
                    \item Evaluate the confidence in the term's definition, and identify actions to take to improve it.
                \end{enumerate}
            \item Across all terms in the innovation glossary, prioritize actions which improve confidence, and decide which to invest it. As its result, change the term or terms it applied to.
        \end{enumerate}
\end{enumerate}

Let's break this down. The process has two big pieces. If you want to start from scratch, Step 1 suggests how to identify one or few first terms that will make up the innovation glossary. Step 2 then improves the glossary from there. Step 2.a says what to do with each individual term; Step 2.a.iii gives an evaluation of confidence, and gives ideas on how to improve it for a single term. Since there are several, perhaps many terms, there will be many potential actions to improve the confidence of them all, and thus we need to look across the entire innovation glossary to prioritize improvements to confidence.





% Chapter bibliography
\printbibliography