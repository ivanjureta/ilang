\documentclass[graybox,envcountchap,sectrefs]{svmono}

% choose options for [] as required from the list
% in the Reference Guide

%\usepackage{mathptmx}
%\usepackage{helvet}
%\usepackage{courier}
%
\usepackage{type1cm}         

\usepackage{makeidx}         % allows index generation
\usepackage{graphicx}        % standard LaTeX graphics tool
                             % when including figure files
\usepackage{multicol}        % used for the two-column index
\usepackage[bottom]{footmisc}% places footnotes at page bottom

\usepackage{newtxtext}       % 
\usepackage{newtxmath}       % selects Times Roman as basic font

% additional, on top of Springer's default packages
\usepackage{microtype}
\usepackage{hyphenat}
\usepackage[colorlinks=true, linkcolor=blue]{hyperref}
\usepackage{url}
%\usepackage{minted} % source code formatting
\usepackage{longtable}
\usepackage{booktabs}
\usepackage{lscape}
\usepackage{listings}
\lstset{
  basicstyle=\ttfamily,
  columns=fullflexible,
}
\usepackage[normalem]{ulem} % for strking through text
% 
\usepackage[refsection=chapter]{biblatex}
\addbibresource{book_references.bib}

\usepackage{verbatim}

% see the list of further useful packages
% in the Reference Guide

\makeindex             % used for the subject index
                       % please use the style svind.ist with
                       % your makeindex program

% custom commands and environments
\newenvironment{pycode}{\VerbatimEnvironment\begin{minted}[linenos,tabsize=2,breaklines]{python}}{\end{minted}}
\newcommand{\ncn}{NCN}
\newcommand{\ncnf}{New Concept Network}
\newcommand{\newterm}[1]{\begin{quote}\textsf{#1}\end{quote}}
\newcommand{\nterm}[1]{\textsf{#1}}
\newcommand{\newdef}[1]{Plastic Definition}
\newcommand{\ndparadox}[1]{\newdef{} Paradox}

%%%%%%%%%%%%%%%%%%%%%%%%%%%%%%%%%%%%%%%%%%%%%%%%%%%%%%
%%%%%%%%%%%%%%%%%%%%%%%%%%%%%%%%%%%%%%%%%%%%%%%%%%%%%%
%%%%%%%%%%%%%%%%%%%%%%%%%%%%%%%%%%%%%%%%%%%%%%%%%%%%%%
\begin{document}

%% TITLE PAGE
\author{Ivan Jureta}
\title{\ncnf s}
\subtitle{How to Rapidly Evolve New Ideas during Innovation}
\maketitle

%%%%%%%%%%%%%%%%%%%%%%%%%%%%%%%%%%%%%%%%%%%%%%%%%%%%%%
%%%%%%%%%%%%%%%%%%%%%%%%%%%%%%%%%%%%%%%%%%%%%%%%%%%%%%
%%%%%%%%%%%%%%%%%%%%%%%%%%%%%%%%%%%%%%%%%%%%%%%%%%%%%%
%%% FRONT MATTER
\frontmatter

%% DEDICATION
%
%%%%%%%%%%%%%%%%%%%%%%% dedic.tex %%%%%%%%%%%%%%%%%%%%%%%%%%%%%%%%%
%
% sample dedication
%
% Use this file as a template for your own input.
%
%%%%%%%%%%%%%%%%%%%%%%%% Springer %%%%%%%%%%%%%%%%%%%%%%%%%%

\begin{dedication}
Use the template \emph{dedic.tex} together with the Springer document class SVMono for monograph-type books or SVMult for contributed volumes to style a quotation or a dedication\index{dedication} at the very beginning of your book
\end{dedication}





\begin{dedication}
For Khione, who lead me to appreciate paradoxes.
\end{dedication}

%% FOREWORD - EMPTY
%%%%%%%%%%%%%%%%%%%%%%%foreword.tex%%%%%%%%%%%%%%%%%%%%%%%%%%%%%%%%%
% sample foreword
%
% Use this file as a template for your own input.
%
%%%%%%%%%%%%%%%%%%%%%%%% Springer %%%%%%%%%%%%%%%%%%%%%%%%%%

\foreword

%% Please have the foreword written here
Use the template \textit{foreword.tex} together with the document class SVMono (monograph-type books) or SVMult (edited books) to style your foreword\index{foreword}. 

The foreword covers introductory remarks preceding the text of a book that are written by a \textit{person other than the author or editor} of the book. If applicable, the foreword precedes the preface which is written by the author or editor of the book.


\vspace{\baselineskip}
\begin{flushright}\noindent
Place, month year\hfill {\it Firstname  Surname}\\
\end{flushright}




%% PREFACE - EMPTY
%%%%%%%%%%%%%%%%%%%%%%%preface.tex%%%%%%%%%%%%%%%%%%%%%%%%%%%%%%%%%%%%%%%%%
% sample preface
%
% Use this file as a template for your own input.
%
%%%%%%%%%%%%%%%%%%%%%%%% Springer %%%%%%%%%%%%%%%%%%%%%%%%%%

\preface

%% Please write your preface here
This book is motivated by a simple observation: when people innovate, they create new words and phrases to talk about new ideas and things that they are creating. They create their innovation language as a tool for communication and coordination. This makes an innovation language a key artifact of an innovation process. It is a reflection of people's intentions, assumptions, and knowledge about the new ideas and things they work on, and relationships with the old, telling us what is and is not new in the outcomes of innovation processes. 

However, innovation languages remain informal and undocumented, and are not subject to rigorous analysis, design and improvement. This creates substantial risks for innovation-driven businesses, whose future growth relies on generating and preserving intellectual property. 

Over the past six years, I have been leading and participating in high-risk innovation processes, in software product startups based in USA, UK, Denmark, Belgium, Israel, and Serbia, where we systematically recorded, documented, designed, and improved innovation languages, used them with everyone involved - investors, lawyers, product designers, product managers, software architects, software engineers, and non-technical staff - in all aspects of business, from corporate strategy and finance, marketing and sales, production, business operations, research and development, delivery, maintenance. Benefits went beyond facilitated communication and teamwork, for local and remote team members. The innovation language became a core asset for preserving, analyzing, improving, and documenting intellectual property, spanning business documentation, requirements and software specifications, marketing and sales material, as well as serving legal professionals who assisted the assessment and protection of intellectual property.

This book has three parts. The first part motivates the need for documenting, analyzing, and improving innovation languages, introduces key concepts needed to do so, shows samples of actual innovation languages, and discusses how and why these languages evolved. The second part of the book shows how innovation languages can be measured, and how measuring them guides their analysis and improvement; this part includes a step-by-step tutorial, on an actual innovation language, with Python code. The third part of the book focuses on the hard problem of evaluating the credibility in an innovation language, which is critical for its adoption and use throughout the lifecycle of the intellectual property it is made to support.



%This book is motivated by a simple observation: when people do innovation, they create new words and phrases to talk about new ideas and things that they are creating. They create their innovation language for two reasons at least; one, it is a tool for communication, and therefore is critical for their coordination, and in turn, it must influence how innovation processes unfold; two, it reflects these people's intentions, assumptions, and knowledge about the new ideas and things they work on, and relationships with old ideas and things, telling us about what is, and is not new in the outcomes of innovation. The book develops the notion of innovation language and proposes a practical approach to how to specify and analyze an innovation language. The proposal is illustrated with several case studies from actual innovation projects in industry; `and analyses are implemented in Python, with the source code included and commented.


%Use the template \emph{preface.tex} together with the document class SVMono (monograph-type books) or SVMult (edited books) to style your preface.
%
%A preface\index{preface} is a book's preliminary statement, usually written by the \textit{author or editor} of a work, which states its origin, scope, purpose, plan, and intended audience, and which sometimes includes afterthoughts and acknowledgments of assistance. 
%
%When written by a person other than the author, it is called a foreword. The preface or foreword is distinct from the introduction, which deals with the subject of the work.
%
%Customarily \textit{acknowledgments} are included as last part of the preface.
 

\vspace{\baselineskip}
\begin{flushright}\noindent
Belgrade, Copenhagen, London, Namur, New York City, Vancouver, Whistler\hfill {\it Ivan  Jureta}\\
May 2019\hfill {\it }\\
\end{flushright}



\preface
Innovation stands for various actions we take to create something new and useful. To prove novelty, we have to explain how the outcomes of all that effort -- the invention -- relates, and specifically differs, from all that's already available, so-called prior art. To prove usefullness, we have to produce evidence that it is being used by our target audience. 

To show \textit{both} novelty and usefulness, we have to define the invention. Its definition, as long as it precisely, accurately, and clearly identifies its properties, will help us identify comparable ideas, artifacts, products, services, and from there let us build an explanation of novelty. The invention's definition is crucial to generating evidence for (and against) usefulness: to build, deliver, and see if and how it is used, we must define it. 

How and when do you make a definition of an invention? A patent specification, an integral part of a patent application, is an example of an exhaustive definition of the invention. However, a patent specification is made after the ideas around the invention are stable, when the inventors are ready to submit a patent application. That moment is only the end of an innovation process, during which inventors came up with new ideas, researched prior art, prototyped (parts of) the invention to try it out with a sample of their target audience, collected feedback, changed their ideas, and performed many such iterations over and over, to build confidence that the invention will in fact become an innovation, once it goes to market. 

This book starts from a simple observation: during innovation, inventors have to describe new ideas in order to communicate about them, and they have to do this well before these ideas are stable enough to justify the effort of producing their exhaustive specifications, or detailed and structured definitions. These descriptions are necessary for coordination -- how else can we agree on what to prototype, make, deliver, and get feedback on?

If innovators have to produce descriptions of their new ideas throughout their innovation process, because they have to communicate and coordinate with others about them, and if we eventually want to have an exhaustive definition, or specification of the invention when ideas on it are mature enough, then we should consider the following question.

What if we wanted to have precise, accurate, and clear definitions of invention during the innovation process, from its earliest moments, and not only at its end? 

This question motivated my efforts when working with inventors over the past ten years, and eventually led to this book. I wanted to understand if producing precise, accurate and clear definitions throughout innovationimpedes innovation, or if it can be done in a way which is helpful. 

It is non-controversial to say that we want to innovate faster rather than slower. We want to rapidly go from early new ideas to more mature new ideas, since the faster we go to market, the earlier we will see the invention in all its glory (or see it fail). But first new ideas are rarely the same as last new ideas an innovation process: an innovation process will rarely stabilize the earliest new ideas; instead, there will be disagreement about the new ideas, learning about what works and what doesn't, ideas will be confronted with the behavior and expectations of a sample target audience. Innovation can involve many iterations, during which new ideas will give place to newer ones, that is, the invention itself will be changing. 

If change is the constant of innovation, then why invest an additional effort into producing precise, accurate, and clear definitions of ideas which we know will change, and can change very quickly? Why not go through the chaos of innovation with low quality descriptions, and wait for there to be enough confidence to be bothered with precision, accuracy, and clarity of the invention's definition?

I argue in this book that we should invest effort to produce precise, accurate, and clear definitions of new ideas during innovation, even if we reject them immediately after producing such defintions. In other words, I argue that innovation processes should embrace the paradox of wanting to be precise, accurate, and clear about unstable ideas. 

The reason to embrace the paradox turns out to be simple. During innovation, new ideas change through confrontation: innovators confront each other on how to change the invention to improve it, they confront the realities of the environment in which the invention is expected to be used, they confront expectations and existing behaviors of their target audience, and so on. In absence of confrontation, why change the earliest new ideas? Why have them in the first place? If confrontation is central to progress through new ideas in an innovation process, and if we want faster innovation, then we should generate confrontations more more rapidly. This is where definition comes in: if one is imprecise, vague, ambiguous about one's new ideas, then it is harder to find what to confront them on. Instead, if one is precise, accurate, and clear, then it is easier for others to identify what they disagree with. In other words, being precise, accurate and clear about new ideas in innovation, is an open invitation for disagreement, one which is easier to accept and act on for others. 

Over the past ten years, I have been leading and participating in innovation processes in companies in USA, UK, Denmark, Belgium, and Israel, where we invented new software products and services, and eventually helped build new organizations around them. We dedicated substantial effort to make precise, accurate, and clear definitions of new ideas from the very start of each innovation process, when new ideas were changing daily. 

These definitions were related, as each used terms from others. Definitions and their relationships formed what I call ''\ncnf'' in this book; as we will see, this is neither a terminology, nor an ontology, but can be a precursor to either. 

We recorded, documented, designed, and improved an \ncn\ in each innovation process. They were available to everyone involved: inventors, investors, lawyers, product designers, product managers, software architects, software engineers, and non-technical staff. It was relevant in all topics, from corporate strategy and finance, marketing and sales, production, business operations, research and development, delivery, maintenance. Benefits went beyond facilitated communication and teamwork, for local and remote team members. The NDN became a core asset for preserving, analyzing, improving, and documenting intellectual property, spanning business documentation, requirements and software specifications, marketing and sales material, as well as serving legal professionals who assisted the assessment and protection of intellectual property.

The book has three parts.
\begin{itemize}
  \item Part \ref{pt-1} is a tutorial on how to make an \ncn. I look back on five actual \ncn s made between 2010 and 2019, outline new ideas that they were trying to define, and show how and why each \ncn\ evolved. I close this part by outlining a general approach to creating and evolving an \ncn.
  \item Part \ref{pt-2} outlines my rationale for proposing \ncn s, and how \ncn s build on and differ from established work on definition in philosophy, terminology in lingistics, ontology and formal specification in computer science.
  \item Part \ref{pt-3} deals with the practical problem of making and using \ncn s which include dozens of definitions. Using basic natural language processing techniques, we can define various measures on an \ncn, and use measurements as indicators to change individual definitions, or the structure of the \ncn. 
\end{itemize}



% This book is motivated by a simple observation: when people innovate, they create new words and phrases to talk about new ideas and things that they are creating. They create their \ncnf as a tool for communication and coordination. This makes an \ncnf a key artifact of an innovation process. It is a reflection of people's intentions, assumptions, and knowledge about the new ideas and things they work on, and relationships with the old, telling us what is and is not new in the outcomes of innovation processes. 

% However, \ncnf s remain informal and undocumented, and are not subject to rigorous analysis, design and improvement. This creates substantial risks for innovation-driven businesses, whose future growth relies on generating and preserving intellectual property. 

% This book has three parts. The first part motivates the need for documenting, analyzing, and improving \ncnf s, introduces key concepts needed to do so, shows samples of actual \ncnf s, and discusses how and why these languages evolved. The second part of the book shows how \ncnf s can be measured, and how measuring them guides their analysis and improvement; this part includes a step-by-step tutorial, on an actual \ncnf, with Python code. The third part of the book focuses on the hard problem of evaluating the credibility in an \ncnf, which is critical for its adoption and use throughout the lifecycle of the intellectual property it is made to support.

% This book is motivated by a simple observation: when people do innovation, they create new words and phrases to talk about new ideas and things that they are creating. They create their \ncnf for two reasons at least; one, it is a tool for communication, and therefore is critical for their coordination, and in turn, it must influence how innovation processes unfold; two, it reflects these people's intentions, assumptions, and knowledge about the new ideas and things they work on, and relationships with old ideas and things, telling us about what is, and is not new in the outcomes of innovation. The book develops the notion of \ncnf and proposes a practical approach to how to specify and analyze an \ncnf. The proposal is illustrated with several case studies from actual innovation projects in industry; `and analyses are implemented in Python, with the source code included and commented.


%Use the template \emph{preface.tex} together with the document class SVMono (monograph-type books) or SVMult (edited books) to style your preface.
%
%A preface\index{preface} is a book's preliminary statement, usually written by the \textit{author or editor} of a work, which states its origin, scope, purpose, plan, and intended audience, and which sometimes includes afterthoughts and acknowledgments of assistance. 
%
%When written by a person other than the author, it is called a foreword. The preface or foreword is distinct from the introduction, which deals with the subject of the work.
%
%Customarily \textit{acknowledgments} are included as last part of the preface.
 

\vspace{\baselineskip}
\begin{flushright}\noindent
\hfill {\it Ivan  Jureta}\\
September 2019\hfill {\it }\\
\end{flushright}





%% ACKNOWLEDGMENT - EMPTY
%%%%%%%%%%%%%%%%%%%%%%%acknow.tex%%%%%%%%%%%%%%%%%%%%%%%%%%%%%%%%%%%%%%%%%
% sample acknowledgement chapter
%
% Use this file as a template for your own input.
%
%%%%%%%%%%%%%%%%%%%%%%%% Springer %%%%%%%%%%%%%%%%%%%%%%%%%%

\extrachap{Acknowledgements}

Use the template \emph{acknow.tex} together with the document class SVMono (monograph-type books) or SVMult (edited books) if you prefer to set your acknowledgement section as a separate chapter instead of including it as last part of your preface.



%% TABLE OF CONTENTS
\tableofcontents

%% ACRONYMS - EMPTY
%%%%%%%%%%%%%%%%%%%%%%%acronym.tex%%%%%%%%%%%%%%%%%%%%%%%%%%%%%%%%%%%%%%%%%
% sample list of acronyms
%
% Use this file as a template for your own input.
%
%%%%%%%%%%%%%%%%%%%%%%%% Springer %%%%%%%%%%%%%%%%%%%%%%%%%%

\extrachap{Acronyms}

Use the template \emph{acronym.tex} together with the document class SVMono (monograph-type books) or SVMult (edited books) to style your list(s) of abbreviations or symbols.

Lists of abbreviations\index{acronyms, list of}, symbols\index{symbols, list of} and the like are easily formatted with the help of the Springer-enhanced \verb|description| environment.

\begin{description}[CABR]
\item[ABC]{Spelled-out abbreviation and definition}
\item[BABI]{Spelled-out abbreviation and definition}
\item[CABR]{Spelled-out abbreviation and definition}
\end{description}

%%%%%%%%%%%%%%%%%%%%%%%%%%%%%%%%%%%%%%%%%%%%%%%%%%%%%%
%%%%%%%%%%%%%%%%%%%%%%%%%%%%%%%%%%%%%%%%%%%%%%%%%%%%%%
%%%%%%%%%%%%%%%%%%%%%%%%%%%%%%%%%%%%%%%%%%%%%%%%%%%%%%
%%% MAIN MATTER
\mainmatter

%%%%%%%%%%%%%%%%%%%%%%%%%%%%%%%%%%%%%%%%%%%%%%%%%%%%%%
% PART I 
%%%%%%%%%%%%%%%%%%%%%%%%%%%%%%%%%%%%%%%%%%%%%%%%%%%%%%
\begin{partbacktext}
\part{Design of \ncnf s}
\label{pt-1}
\end{partbacktext}

%%%%%%%%
\chapter{Appreciating Disagreement over New Ideas}\label{c:introduction}

%\abstract{}

\section{Loads or Shipments? Truckload or LTL?}
%\label{}
''Why is it a problem to have stops? Stops are common. We should be able to add them to a live load.'' He was insisting.
 
This made no sense to me.
 
''You mean a shipment, right? The load becomes a shipment once matched.'' I waited for his confirmation. It wasn't happening.

This got me thinking about what it means to add stops for loads too. What was this about live loads? We'd have to change how matching algorithms work. This already took months to research, design, redesign, and have everyone align on. My next meeting with design, engineering, and quality teams will not go well. I can't keep revising the short-term roadmap, nothing will get done.
 
''That's what I meant. Was it truckload only? We did say LTL too?'' He was one of the founders, and an important investor.

''No. We said truckload, and we agreed at the time that this was full truckload only. LTL is a different business altogether. You know it. You built a business in FTL before, and I don't think you did LTL. Different customer needs, suppliers, service, technology. We'd have to do new research. Do you want to wait for another year? Differentiators are different. Everything is different.'' I was Head of Product at the time, which meant that I was responsible for aligning everyone on what the product \textit{is}, what it \textit{could be}, and getting everyone to agree what it \textit{should} be. In this venture, the initial ideas came from investors, what one might call a ''product vision''. It was also on me to make sure the product satisfies everyone, from customers to engineers who make, release, maintain, and improve it. 

''Can we stick to truckload only for now? We know it’s a big opportunity, we're early, and it's complicated enough.'' I hoped this would stop him, or at least postpone this.

He was silent. I continued. ''So, there's no such thing as 'live load'. I know someone may be calling freight that's moveing a 'live load', but we aren't. Remember, the load is what the customer asks us to move, and it stays a load until it's matched to a carrier; at that point, it becomes a shipment. Loads and shipments are described in a different way, the information about the load is only some of the information we then need to have and keep a record of, about a shipment.'' 

It might have been the tenth or 20th time we had essentially the same issue; I lost count. It wasn't specific to the two of us. We had been working together for a while. There were no bad intentions. It was happening frequently and in our other teams. It was in conversations, brainstorming, planning. What we used to communicate didn't make much difference -- emails, chats, remote or live meetings. It was faster to resolve in live meetings, but that lasted only until the end of the meeting, or at most until the next one.

\section{Unpacking Disagreement}
\label{s:unpacking-disagreement}
At some times, we used the same names for different new ideas, and at others, we used different names for similar new ideas. We disagreed frequently. Even when we might have agreed quickly, we couldn't. If the same words stand for different ideas, and if these ideas are new (and therefore, do not come with an established definition), you are never sure if you agree or not. 

Disagreements we had over ''load'', ''shipment'', ''truckload'' and ''LTL'' were an insignificantly small sample of confrontations we had over four years when I was involved in the logistics venture. Innovation there was never-ending. As our customers changed their minds about what worked best for them, as we acquired new customers with new expectations conventions, constraints, practices, we kept coming up with new ideas internally for how to change our organization, products, services, systems, in response.

In such an environment, it is more useful to develop an appreciation for disagreement, than to prefer stability. This is not only to accept it as a frequent phenomenon, but also to learn to analyze it, so you can then better decide how to address it. 

Part of the problem with ''load'' and ''shipment'' was that we used same words for different new ideas. The cure for polysemy is obvious if we could pick one of a few available definitions for ''load'' and ''shipment'': we should review available ones and agree on one for each word. 

Disagreement over new concepts is more subtle, of course, for three reasons. 

Firstly, it is na\"{i}ve to expect to reach the agreement easily -- disagreement is not simply over which definition we will pick among a few, it is over the scope of the system, product, service to design, build, run, manage, and improve. There are significant ramifications of adopting a definition of a new concept; the definition affects where we want to go, how we will get there, and what resources we will need. If we defined ''load'' as being anyting fitting in a dry van, this would not remove the possibility of shipping smaller loads for different customers on the same truck (the same trailer), and would lead us to LTL. 

Secondly, disagreement over ''load'' may not be local to the definition of ''load''. What we agree for ''load'' may lead us to have to change our definition of ''shipment'', ''customer'', and others. New concepts depend on each other, in that the meaning of one will be tied to the meaning of others. If definitions ought to represent some of that meaning, then changing the definition of one new concepts will affect definitions of other concepts, where the former is mentioned. If the definition of X mentions Y, then changing the definition of Y may require us to change the definition of X.\footnote{Part \ref{pt-3} focuses on such relationships between definitions, and how we can use them.}

Thirdly, we were creating \textit{new} ideas, and the first version of a new idea is rarely the best.\footnote{Why am I talking about ''new ideas'', and not ''new concepts'' or ''new terms'' or ''neologisms''? This is because I will need one definition of ''concept'' and ''term'' later, when more precise relationships between concept, term, idea, neologism, and others, starts to matter.} It wasn't that we disagreed over \textit{general-purpose} or even \textit{established specialized definitions} of ''load'' and ''shipment'' -- we used these words in new ways, specific to inventions we were coming up with, within the local context of the innovation process we were involved in. Even if he had a specialized, industry-standard definition in mind for ''load'', it didn't matter, since I was looking for an idea of load which was new, and which fitted \textit{our} aims and \textit{our} constraints and the innovation \textit{we} wanted to get to.\footnote{It could be argued that it's bad practice to use words with established definitions to denote new ideas, and instead have a neologism; we will return to this later, via a detour in lexicography.} The problem that the novelty of an idea introduces, is that disagreement we have now is not going to be the only disagreement we will have: the new idea will go through many changes, which will be motivated by various disagreements over time. 

Disagreement over new ideas is a problem that intensifies over time and with more people. The more successful the venture became, the more this problem become pronounced, and costed more to solve. If communication leads to disagreement over meaning of words, how can you tell that the teams are in sync? How could you possibly assess and manage the risk of planning one thing, then being delivered another? 

Disagreement about who meant what, while working on new ideas, may seem a straightforward issue to solve. Let's get together and talk it through. But you first need to detect disagreement, then spend time solving it. You might detect it late, after damage is done. Handling it means more communication, not less. Could you have avoided this? 

When you know that there is a risk for this kind of disagreement to occur, how do you detect it? Moreover, how do you detect it early, when it involves fewer people, before more is invested, and may only have affected inconsequential decisions? How can you make detection and correction part of a routine, instead of just hoping it will all go well?

\section{Is Disagreement an Anomaly?}
This book has ''new'' in its title. The focus is on new concepts, those which are invented to fit specific purposes when we design and build new products, services, systems. 

The problem with disagreement about new concepts is quite different from disagreement established concepts. When we disagree  on established concepts, there is a reference that we can look up, to settle our differences and reach a common understanding. This could be a dictionary, an encyclopedia, a terminology accepted in a domain -- something that we can both accept, along with others, as an authoritative source. 

However, when we disagree on \textit{new concepts}, then there will be no authoritative source, someone other than the two of us, or a passive source -- a book, database, knowledge base, or otherwise -- which we can both go to. 

Instead, \textit{we have to create and define the new concept}. This is exactly what was done in the logistics venture, where we had a new and our own ''load'' and ''shipment'' concepts, among many others.

The same happened in other businesses I was involved in during the last decade: I was in teams which were tasked with inventing, creating, testing, delivering, and running new products, services, systems which targeted specific opportunities and problems in various industries. We were coming up with new concepts, and had to make specific definitions for them -- part of it was so that we can agree internally on what to do with and about them, the other part being that we have to be clear how our innovation differed from what was already available.

Disagreement over established concepts and disagreement over new concepts are two different kinds of anomalies. The former signals the need to point everyone to the authoritative reference, which provides the agreed-upon concept. The latter begs a different question: Is disagreement a signal that the concept in question should change? And if so, how do we change it so as to avoid disagreement later? 

The key point is that disagreement over established concepts signals an anomaly, something to detect and correct without changing the concept, while disagreement over new concepts is part of their formation, that is, is a step in the creation of such concepts, and in their maturing up to the moment when they become accepted by, and thereby established in a community. At that point, there is an authoritative source, an accepted definition, and disagreement is an anomaly.


\section{Did Disagreement come only from Synonymy and Ambiguity of Nouns?}
If two different words stand for the same idea, they are synonyms. A single word is ambiguous if it can stand for different ideas. 

The examples I used so far are all nouns; \textit{truckload}, \textit{FTL} abbreviates \textit{full truckload}, \textit{LTL} is for \textit{less than truckload}, and there were \textit{loads} and \textit{shipments}. It is quite obvious that a noun can be ambiguous, or that two (or more) of them can be synonyms. 

Same applies to verbs. One important verb in that logistics venture was ''to match''. We wanted to build an online marketplace for freight transportation; if I grossly oversimplify, the marketplace is software which is used both by those who have freight and need to have it transported, and they are the demand side, and there is the supply side, those whose business is to transport freight. What a marketplace should do, is to match supply with demand, which is to say, make sure that as many businesses as possible from the demand side find someone on the supply side to move their freight.

When we started thinking about the market, we had to say how exactly we want demand to match to supply. ''Matching'' became a word we used hundreds of times daily,  for months. It meant very different things to different people on the various teams. Very early on, when only six of us were involved in designing this market and its marketplace, I was responsible for proposing and analyzing different ways that ''matching'' could work. At the very first meeting on this, I remember laying out a few dozen ways matching could be accomplished, each with its own pros and cons; and this was actually a small subset of what can be done. There is a field in academic economics research focused on so-called \textit{market design}\footnote{''Economists have lately been called upon not only to analyze markets, but to design them. Market design involves a responsibility for detail, a need to deal with all of a market's complications, not just its principle features. Designers therefore cannot work only with the simple conceptual models used for theoretical insights into the general working of markets. Instead, market design calls for an engineering approach.'' \cite{roth2002economist} Ours was exactly an engineering approach, informed by economic theory (especially Alvin Roth's work \cite{roth2002economist}, including that with Marilda Sotomayor \cite{roth1992two} and Axel Ockenfels \cite{roth2002last}) and business practices in the logistics industry.} which deals, among other, exactly with this kind of problem.

Is ''matching'' an ambiguous verb? It isn't in general: in daily informal, general-purpose usage, it is probably not. Another way to think of it, is that it is not ambiguous enough to cause trouble, so you don't think much about all it could possibly mean. But when we had to design the marketplace, the meaning it had for the people involved was critical, since we were all wanting to build the marketplace to work exactly according to one single idea of matching.\footnote{Keep in mind that there was no place where we could look for an established idea of matching, of how matching should work in our specific marketplace. That is, it's not that we were failing to agree on some given definition of ''to match'', but we had to make a new definition for it, and agree.} If one of us thought it should work in one way, another in another way, and this disagreement remained opaque when each of us used the verb ''to match'', then any agreement we thought we had was creating risks as we continued designing and building that marketplace.

What about adjectives? In their case, issues can come from vagueness. Gradable adjectives are vague; big, small, tall, fast, easy, high, low, and so on, are all vague. Such adjectives imply an ill-defined scale, whose units have no standard, universal definition, and there is no generally agreed upon threshold which cuts that scale up: I might think this car is big, but you could say it isn't, and we could both have good reasons to stand by what we thought. Is a car big if it is longer than some specified length? Is it heavy because it goes over some specified weight? There are no such specifications in general; have a look at an encyclopedia, Wikipedia included, if you disagree -- that's where it would normally show up. Does it?

One of our goals was to make matching ''transparent'' for our users. We spent a long time trying to agree what that \textit{should} mean. Should it mean that we show the budget available on the demand side to the supply side? Does it mean showing the name of the supplier to the customer? Notice the complexity behind just these two questions: they touch directly the business model, and more specifically how the operator of this marketplace -- that was us -- make money from managing it.  

We were disagreeing with each other in the logistics venture, sometimes substantially, on nouns, verbs, and adjectives. As we build sentences over these, problems of the pieces we put together wouldn't go away. In those sentences, adverbs, pronouns, determiners, prepositions, and conjunctions wouldn't -- just by being added in a mix -- remove ambiguity, synonymy, or vaguness.


\section{What is Wrong with Ambiguity, Synonymy, and Vagueness?}
There is nothing wrong \textit{per se} with ambiguity, synonymy, and vagueness. They cause risks, and if you accept to live with those risks, then there's no need to worry. In fact, I live with many such risks daily; when my daugther says she'll return her toys in place soon, I might wonder when exactly, i.e., she's being vague. But her not returning toys in place is not much of an issue; so while there is a non-null probability she will not do it, any negative consequences of this are, at least for me, negligeable, and I'm simply ignoring that risk; I am overthikning this one here, but the point should be clear. Other, also quite frequent risks, are not negligable; if there's a delay on a flight I should take, it is not the same if that delay remains unspecified by the airline, or if it is clearly communicated, i.e., it is not the same to see a message that says ''flight delayed'' and ''flight delayed by 2 hours'' -- each implies different decision options and different criteria; the former takes me to speak with ground staff, and what I'd like there is to have an idea of the delay, while the second may lead me to ask for a rebooking of my next flight on the same day.

% If I disagree with an investor over the meaning of some frequently used words (be they nouns, verbs, or adjectives), or even phrases, I might end up guiding teams to deliver something that is disconnected from that investors were asking for. There, the risk is significant, because it is very different to design software for handling truckload shipments or LTL shipments. 

Here is one example where we failed to appreciate the risk and suffered eventually. So-called ''onboarding'' of carriers is about how to bring (and again, I'm oversimplifying) trucking companies to the supply side of the marketplace. The software was initially designed and delivered with a simple carrier onboarding process, the assumption being that the shorter it takes to the carrier, the better. It turned out that this was too optimistic; fraudulent carriers passed the process alongside trustworthy ones, and this caused issues for our customers and us. It took us months to regain trust with some of our pilot customers, and the launch of the software on the market also had to wait for repairs to relationships and code to be done. Besides the time of investors, management, engineering, and other teams, it was also a hit to morale, an aspect that does not lend itself easily to quantification.

Disagreement over the meaning of ''scalable'' is another example. Investors kept insisting on having a scalable marketplace; we ended up designing and delivering software which could, based on simulations at least, scale to support all transactions in logistics in the North American market. But that was an unrealistic scale, one which no-one in this market could ever achieve (if only because of anti-trust regulation). Supporting that volume of transactions required a complex system, which was costly to change. 

Ambiguity, synonymy, and vegueness create risks which we should be aware of, and even better, which we should proactively identify, estimate, and manage. 


\section{Standardization or Progress?}
\label{s:useful-disagreement}
We can live with risks that come from ambiguity (polysemy), synonymy, and vagueness. Should we? 

If you are proposing a new term and its definition, then it also makes sense to want to be precise, accurate, and clear about the meaning you are trying to give it with that definition. This seems a straightforward motive --  you simply want to make sure \textit{you} are understood as \textit{you} intended, regardless of what comes next, that is, if you will meet agreement or confrontation within the community you are proposing that term to.

Efficiency in communication seems a straightforward motive for precision, accuract, clarity of a definition. 
%It is a major reason for investing in the development of terminologies.

There is a more subtle question, however: What exactly are you trying to accomplish when proposing a definition of a new term? Are you proposing the new term's definition for standardization as-is? By standardization, I mean that the definition gets the status as being the right one for use in the community, and that any other interpretation of the term becomes an error. Or are you proposing it for discussion, before and independently from standardization? 

There was a rather technical debate in ecology research, on the definition of the term ''ecosystem engineer''. It is interesting here not for its contributions to ecology, but for other reasons. One, it was a proposal of a new term and its definition. Two, the proposal led to a debate, which went through a few iterations. Three, the proposed definition was picked up in a different debate, on whether such proposals help or harm the development of technical, specialized knowledge in a discipline. Four, it was not clear if the authors of the new term wanted to standardize their definition immediately, and that specific point was also subsequently debated. Overall, it is an example that touches on many topics discussed so far.

In 1994, Clive G. Jones, John H. Lawton and Moshe Shachak, three scientists working on ecosystem change, proposed the term ''ecosystem engineer''. They offered the following definition.

\begin{quote}
  ''Ecosystem engineers are organisms that directly or indirectly modulate the availability of resources (other than themselves) to other species, by causing physical state changes in biotic or abiotic materials. In so doing they modify, maintain and/or create habitats. 
  
  Autogenic engineers change the environment via their own physical structures i.e., their living and dead tissues. Allogenic engineers change the environment by transforming living or non-living materials from one physical state to another, via mechanical or other means.'' \cite{jones1994organisms}
\end{quote}
Beavers, according to them, qualify as ecosystem engieers:
\begin{quote}
  ''That is they are allogenic engineers, taking materials in the environment (in this case trees, but in the more general case it can be any living or non-living material) and turning them (engineering them) from physical state 1 (living trees) into physical state 2 (dead trees in a beaver dam). This act of engineering then createsa pond, and it is the pond which has profound effects on a whole series of resource-flows used by other organisms. The critical step in this process is the transformation of trees from state 1 (living) to state 2 (a dam). This transformation modulates the supply of other resources, particularly water, but also sediments, nutrients etc.'' 
\end{quote} 

In a later article \cite{flecker1996ecosystem}, Alexander Flecker preseted research on how a fish, \textit{Prichilodus mariae}, modifies its environment by ingesting sediment, and this in turn changes algal and invertebrate assemblages. He argued that \textit{Prichilodus} is an ecosystem engineer. 

Commenting on that work from Flecker, Mary Power argued that ''ecosystem engineer'' is a value-ladden term. For her, it subsumes intent \cite{power1997estimating}. Since intent to modify habitat is hard to attribute to \textit{Prichilodus}, she aruged that this was a case of ''habitat modification'' instead. \textit{Prichilodus mariae} isn't an ecosystem engineer for her, and the definition of ''ecosystem engineer'' is failing to be clear on the need for intent to be present.

Jones, Lawton and Shachak replied to Power, arguing against her interpretation of their definition for ''ecosystem engineer''.

\begin{quote}
''First, our definition of ecosystem engineering is not value-laden, and does not imply intent. Whether or not humans perceive intent when an organism (such as a beaver or a human) physically modifies the environment is not a scientific issue. It becomes the subject of scientific enquiry if questions are asked about feedbacks to the engineer. 

Engineering does not require a feedback, even though this often occurs. We made the explicit distinction between engineering that affects the fitness of the engineer ('extended phenotype engineering'; e.g. the effect of beaver dam on beaver) -- which is what we construe Power meant by purpose -- and 'accidental engineering' (e.g. a cow hoof print that probably has no feedback effects on the cow).

Second, the term 'habitat modification', if used to mean either a process or an outcome, is a much broader term, than ecosystem engineering by organisms. For example, habitats are often modified by abiotic forces. We see little point in being less precise when precise terms exist. It is also worth noting that our definition considers engineering to be a process, and thus any habitat modification / maintenance / creation / destruction that occurs is an outcome.'' \cite{jones1997ecosystem}
\end{quote}

Two things stand out in the debate so far. Power raised an issue with the applicability, or scope of the definition, by arguing that the fish in question does not qualify as an instance of ecosystem engineer. In addition, her argument comes her understanding of the term ''engineer'', in which purposeful intent is a must. The defense from Jones, Lawton and Shachak is that they have a different understanding of the term ''engineering'', and so of ''engineer''. A legitimate question is if this disagreement and debate would have been any different, if  Jones, Lawton and Shachak also gave their definition of ''engineering'' alongside that of ''ecosystem engineer''. This touches on the very important question on the relationship between definitions of established terms, and definitions of new ones, and what to do about it when proposing a definition of a new term. If you are proposing a definition for some new term, and that definition mentions other terms A, B, and C which are not new, but have established definitions already, do you give their definitions alongside the new one, or do something else? I will return to this later. 

Jones, Lawton and Shachak also remarked the following:
\begin{quote}
''Last, when we did 'coin the term' ecosystem engineering, adding new 'buzzwords' to the ecological lexicon was certainly not our primary purpose. Many of the interesting questions about the effects of species in ecosystems that were discussed by Power, were also the focus of our papers. Asking and answering interesting and important ecological questions is the primary purpose of our science. However, since many areas of ecology do not yet use unambiguous formal language (unlike the formulas of mathematicians, physicists and chemists), we must pay particular attention to terminology. After all, we cannot have scientifically meaningful dialogue unless we first agree on the definition of what we are studying.''
\end{quote}

It is still not clear at this point if Jones, Lawton and Shachak were open to revising the definition they proposed. They did not do so in subsequent publications, at least not in response to Mary Power's critique. Absence of clarity on this point, as well as their rejection of Power's critique come back in a different article, where this exchange is used as an example of how to stiffle innovation and growth of knowledge, specifically in ecology research.

Karen Hodges wrote the following in reaction to Jones, Lawton and Shachak's claim that ecology needs to pay particular attention to terminology.

\begin{quote}
''Polysemy and synonymy may stimulate rapid growth in a field, vague terms are not necessarily problematic, and creating rigid definitions and standardized terminology too early may stunt the growth of a field. [...] Strongly demarcated definitions and classificatory decisions can [...] have serious negative effects on a discipline by constraining inquiry.'' \cite{hodges2008defining}
\end{quote}

It is surprising to assume that to grow knowledge, you should stimulate polysemy and synonymy. Proposing a definition early, and making it precise, accurate, and clear, will stiffle the development of new knowledge only if this definition comes bundled with the constraint that it cannot change, that it should be the standard. It is only if its authors refuse to remain open to criticism, or simply ignore it, that \textit{they} will stiffle innovation. The definition alone cannot do this. The benefit of making it precise, accurate, and clear, is that it will be easier to debate. This is not a tradeoff between precision and innovation, it is between premature standardization and opennes to change the definition in light of new arguments and information. 

The example, especially the reply from Hodges, suggest that it is important to be clear about what you want to accomplish when proposing a definition: Are you offering it as final, and thus want to see it accepted as-is, or is it to advance debate towards a more stable definition? In either case, being precise, accurate, and clear should help, not harm.


\section{Constructive Disagreement}\label{s:constructive-disagreement}
The important conclusion of this Chapter is that defining new ideas involves disagreement. But should we then appreciate all disagreement equally? 

The argument I developed in this Chapter is that we should prefer disagreement which arises over definitions of new ideas, whereby such definitions are precise, accurate, and clear. We should prefer it over disagreement which merely signals lack of such qualities in definitions, and where resolution of ambiguity, synonymy, or vagueness would add nothing to the maturity of the new ideas in question. The practical difficulty with this preference, is that it is hard to detect what disagreement came from. We could debate what would have happened if the definition were phrased otherwise, but it wasn't; this kind of speculation takes yet more time, and has unclear and unpredictable benefits.

If we do want to keep that preference, then the definitions of new ideas should be precise, accurate, and clear, \textit{and} any definition should be offered as tentative, open for discussion and destruction, rather than hostile to change. If you take that seriously, then definitions of new ideas aim different things than those of established ideas: \textit{definitions of new ideas invite thir own destruction, while definitions of established ideas invite acceptance and refuse dissent}. 

\section{Background to Cases: High-Risk, High-Return Technology Startup Companies}\label{s:cases-background}
I started doing fieldwork in teams tasked with innovation in spring of 2007. Up to 2019, almost all teams I worked with were innovating in products and services where software played a major part. Almost all were also startups, and by that I mean companies which all had the following four characteristics: 
\begin{itemize}
    \item They were \textit{new companies}, neither spin-offs of existing ones, nor incubated by existing ones, nor academic spin-offs; they started off without a direct access to early adopting customers, without an already formed team that had worked together in the past.
    \item Their investors wanted \textit{high returns over short periods of time}: for every dollar invested, the aim was to return 5 or more, ideally above 10, within 2 to 4 years.
    \item \textit{The only way to achieve these returns was through innovation}; 
    \item We were tasked with coming up with new ways to solve underspecified problems. These problems were not posed in clear and precise terms, had no sharply set constraints, were given in  vaguely identified and delimited environments, and had no decided scope. Many of us involved were trying to understand the context, history, and much else about these problems for the first time-- including whose problems they are, that is, who are stakeholders were.
\end{itemize}

My initial aim with fieldwork was to understand how requirements engineering is done in such contexts. At the time, I was close to completing and defending my doctoral thesis on the notion of quality in requirements engineering. One of the important personal observations that came out of being confronted to predominantly academic discussions on what requirements are, how to elicit, analyze, document, negotiate, verify, and validate them, was that there must be interesting research questions in the field. Moreover, I was convinced that to have a relevant discussion, one has to have experienced hard requirements problems. I thought the hardest ones must be those where requirements are not known up front, where they are formed through the process itself, which means that requirements are unstable, inconsistent, incoherent, and the problem itself, so-called requirements problem, is not a given, but formed thgouth many decisions and iterations. Hence my interest in innovation, and specifically in particularly adverse conditions of low resources, small teams, and ambitious goals, that is, technology startups.

Besides many other interesting questions and results that came from fieldwork, the most interesting one seems quite simple in retrospect. Twelve years later, this led to the notion of \ncnf s. 

\ncnf s come from a seemingly trivial practice I considered to be basic hygene in innovation processes that I was getting myself into. As soon as we would start working, even when even brainstorming was going into many directions, I insisted on writing down keywords, key terms that popped up in those sessions, and more generally in all team communication, and sharing a tentative definition of each with the team, asking everyone to either accept them or suggest changes. 

Most of the people involved were surprised, but interested, while some saw this as futile, a distraction even. Ideas changed so fast. What good does it to to try to define them? Isn't definition something we should do later? These questions kept coming back.

It was true that many definitions of terms were thrown away, as well as many terms altogether. But I insisted on keeping even a simple list of new terms and definitions available to all, and asking for all to know it and voice any disagreements at any time.

It turned out that, even in my small sample of innovation projects in the last decade, that value grew over time. It grew to the point that in many businesses where we did this, it became an accepted practice. This meant that I no longer needed to activaly promote it. I was no longer the only person to do this. 

Design, engineering, technical and other documentation started including pointers to a centralized list of new terms and their definitions, which was actively maintained, revised, commented. Most interestingly to me, this looked like a sticky tool, something that people accepted without formal requests to do so. When definitions stabilized, and these lists became a regular terminology, this the backbone for knowledge management. I certainly did not rediscover with this the value of glossaries, terminologies, and dictionaries of technical terms; their value is clear in established disciplines, and many businesses, especially those with a long history and a complicated and large set of products and services, know how important it is to be careful about their internal technical language. But as we will see, getting to a terminology when you do innovation, is different than taking an off-the-shelf terminology which is part of a widely-accepted body of knowledge.

Where I was able to observe the use of such changing terminologies of new terms, it became simpler to talk about teams' innovations with interested parties, lawyers, future users, marketers, and any stakeholder group which was getting involved over the lifecycle of  inventions.

The next five chapters present data from startups where we were defining new terms for new ideas. Each Chapter is a different innovation process, spanning anywhere between 3 months to 4 years. At the time, these did not benefit fully from the methods outlined in this book, especially those presented in Part \ref{pt-3}. They did, however, involve the writing, revision, rewriting of definitions for terms that were meant to convey new ideas, and as such led to the production of sets of definitions which constitute \ncnf s.

% Chapter bibliography
\printbibliography


%%%%%%%%
\chapter{Case: Defining a New Contracting Process}\label{c:case:contracting-process}

\section{Background}
In 2017, a team that combined business consultants, product designers, and software engineers, was asked to invent a process and supporting software, which would together speed up the negotiation of simple business consultancy contracts. 

These contracts involve always only two parties, where one delivers intellectual services to the other. The service provider is compensated at an agreed hourly rate. The contracting process goes from how they identify each other, i.e., get matched, through negotiation of deliverables, timelines, and price. The blueprint of the process that we were to invent would serve for the production of a software, which would be used by both sides.

When we started, the default way to negotiate such a contract, is for the service provider, or consultant, and the client to hold meetings, and communicate in other ways, until both parties were confident enough to agree on terms and commit. It is a frequent, but unstructured process. It is hard to repeat, measure, and improve. It can take varying amounts of time, and end in no contract at all.

\section{Challenges}
The initial discussion session focused on identifying the benefits that the new approach would have for buyers and providers of services. Quickly, many questions arose, key among them being the following:
\begin{itemize}
	\item How do we move a complex process, in which key discussions and decisions are made through live communication, to asynchronous communication, where much now needs to happen through writing? 
	\item If this can be done, then how do we help service buyers identify the most relevant service providers, for the goals they want to have solved? 
	\item How do we recommend buyers to providers, and \textit{vice versa}? 
	\item How do we help both sides reach an understanding on goals to achieve, tasks to do, timeline to do them, price and payment?
	\item How do we collect feedback of each party about the other, and use it to improve future recommendations?
	\item How do we help these parties raise and resolve disputes? How do we distribute responsibilities for issues between them, and the future business which will offer this service to providers and buyers?
	\item Before, and when we have the process, how do we grow the marketplace from a small set of pilot buyers and providers? 
\end{itemize}

We spent the next three months proposing, testing, changing, refining, throwing away different answers to these questions. The entire time we invested in this was an innovation process, in that we converged to an agreement on the process and software specification only towards the very end of that period. 

We produced various artefacts during the innovation process. Drawings, diagrams, algorithm specifications, working prototypes of potential pieces of the future software, interview guides to use with pilot buyers and providers are some of them. 

A list of definitions of new terms was the sole artefact which was made in the first days, and remained relevant throughout the innovation process -- it was indeed used and debated by everyone on the team. It was also a notoriously unstable artefact, changing frequently to reflect key design decisions, open questions and dilemmas.

\section{Finding New Terms to Define}
Where did we begin? What were the first definitions we had, and why did we start from them? 

The start is simple, and it always looked roughly speaking the same in all cases I was involved in: we take time to discuss the motivation and expectations from the invention that the innovation process should produce. In practice, this looks like a few hours in live meetings, trying to find a starting point that everyone sees as important for the success of the invention. 

The starting point has three parts to it:
\begin{itemize}
	\item What is new? There has to be an idea about how the invention will differ from what the team is aware of; evidently, we may go back and throw this idea away later, after we find out more.
	\item Who do we want to expose to this novelty, so that they decide if they will use it? It has to be new, but not in an absolute sense; it may be new to a specific audience, even if it isn't to another one, or its use in another context may be the novelty, and we expect it to be useful to an audience concerned by that specific context.
	\item Why would the invention also be useful for this audience? What do we think would make the target audience interested in trying the invention out, so that they can decide whether it is useful or not?
\end{itemize}

This really was only a starting point: it was never indicative of the scope, depth, or complexity of what followed, it was not stable, and so couldn't be an anchor or a reference point to which we might come back later. 

Hours to get to the starting point never turned into days, however. One day of four to six hours of meetings is usually more than enough to have the starting point, and with that, a few terms that keep coming back, or ideas which need naming; in either case, the next step was to write these terms, make a definition for each, and ask others to make changes, or assume acceptance. 

The starting point in this case were five terms: \nterm{marketplace}, \nterm{project}, \nterm{deal}, \nterm{company}, and \nterm{expert}. They came out of the starting point, which was roughly this: When a company needs services of an expert, for a short time, on a specific project or to help close a deal, it can be very hard to identify actual competency, to trust available evidence of reputation, and then, to get to a formal agreement on what to deliver, when, and what to do in case of dispute. Word-of-mouth is still the best tool, but it lacks transparency, structure, availability. On the other hand, experts need to make themselves available to companies, in a way in which they can provide trustful evidence of relevant competence. Both sides need to be able to find each other, and follow a clear process to get to an agreement, having a third party that will mediate the contracting and disputes which may take place. That third party would need to run the marketplace which would match the demand and supply of experts' time to work on projects and deals.

None of the five initial terms are neologisms. Here are their general definitions. 
\begin{quote}
	Marketplace: ''An open space where a market is or was formerly held.'' \cite{oed-marketplace}\\
	Project: ''An individual or collaborative enterprise that is carefully planned to achieve a particular aim.'' \cite{oed-project}\\
	Deal: ''An agreement entered into by two or more parties for their mutual benefit, especially in a business or political context.''\cite{oed-deal}\\
	Company: ''A commercial business.''\cite{oed-company}\\
	Expert:  ''A person who is very knowledgeable about or skilful in a particular area.''\cite{oed-expert}
\end{quote}

These definitions did not work for us, because we needed them to be more specific to the invention we were looking to produce. 

We realized already at the very start that the starting point was too generic. It wasn't specific to an industry, geography, market, or something else that would limit its scope. That, in turn, would reduce the risk we would be taking, and, we hoped, would shorten the time to put the invention in front of the target audiance. This was consequently reflected in the first definition of \nterm{marketplace}.

\newterm{Marketplace: Online software which mid-size information technology companies would use to find and contract with business development experts who can help these companies expand into other territories.}

''Marketplace'' and \nterm{marketplace} may be the same nouns, but they have significantly different definitions. One  difference is specificity. Another difference is the absence to a market independently of the place, which is software in our case, where the supply and demand would meet. Even if ''marketplace'' is certainly no neologism, we did have a meaning for it that was different from a generic one. At the same time, and to the best of our knowledge, there were no comparable marketplaces available, which might include terms defined in a similar way.  

You may disagree with how we defined \nterm{Marketplace}, but the details of that do not matter for the moment; it was useful at that time, and stayed so only for a short while. The interesting question is what we did once we had that first definition. 

Figure \ref{fig-steps-1} shows an idealized sequence of next steps we took. I say idealized, because the illustration abstracts away discussions, offline and online, which we did to get through those four steps. 

\begin{figure}[p]
 \centering
 \includegraphics[width=\textwidth]{fig-steps-1.pdf}
 \caption{}
 \label{fig-steps-1}
\end{figure}

In the Figure, Step 1 shows the term \nterm{Marketplace} and its definition, same as the one above. Step 2 shows where that definition mentions \nterm{Company} and \nterm{Expert}. We then made a definition for \nterm{Company}, shown in Step 3, and subsequently one for \nterm{Expert}, in Step 4.

Leaving aside the specifics of those definitions, the important observation is that once we have one definition, we can grow a network around it. The network shows how one definition depends on another one, and to some extent how the meaning intended for one concept depends on the meaning of the other.

Returning to specificity, it was also clear in definitions of the remaining terms \nterm{company}, \nterm{expert}, \nterm{project}, and \nterm{deal}. 

\newterm{Expert: Individual looking for an efficient way to find work for leading companies they can help the most, and which will pay high rates.}

\newterm{Company: Mid-size information technology business looking for people who are experts in the same technologies, products, services, with real contacts in the markets where the business is interested to establish itself.}

\newterm{Project: List of deliverables, dates at which deliverables should be provided by expert to company, and budget that the company will pay in return the expert.}

\newterm{Deal: A contract that the company signed to sell products and, or services to another party, and where the company recognizes that the expert facilitated the creation and signing of that contract.}

Are these definitions precise, accurate, clear? None of them is. 

In the definition of \nterm{marketplace}, at the very least, we have to be precise about what mid-size companies means, as this may be a marketing or a legal category, and if it is the latter, then we need to say what the applicable geography is, namely, which country or countries the software will be available in. The category ''information technology'' also needs a definition, which can start from a generic one, but may also need to be made more specific, or replaced altogether. 

The more general, and important point, is that for each definition, we need to break it down and consider individually each term that it mentions. Doing that also grows the network. Step 5 in Figure \ref{fig-steps-2} shows more terms, in rectangles, which need definitions, and which are mentioned by \nterm{Marketplace}. We grow the network by assuming each of these should have a general defintion, and then make changes and removals if the team disagrees. This works if the term is not a neologism; if it is, a definition needs to be made for it. Step 6 in that Figure shows just how fast this network will grow, since all terms in rectangles will have need a definition, be it new, or one from a dictionary or a terminology.

\begin{figure}[p]
 \centering
 \includegraphics[width=\textwidth]{fig-steps-2.pdf}
 \caption{}
 \label{fig-steps-2}
\end{figure}

Notice that \nterm{Deal} and \nterm{Project} do not appear in the definitions in Figures \ref{fig-steps-1} and \ref{fig-steps-2}. We can see this by showing only the terms highlighted in Step 6, and their relationships, as in Figure \ref{fig-steps-3}. 

\begin{figure}[]
 \centering
 \includegraphics[width=10cm]{fig-steps-3.pdf}
 \caption{}
 \label{fig-steps-3}
\end{figure}

We said at the starting point that \nterm{Companies} are looking for \nterm{Experts} precisely because of the interest in getting new deals signed and projects completed. The network does not reflect this, leading to changed defintions as follows. Removed parts are striked through, new parts are underlined.  

\newterm{Marketplace: Online software which mid-size information technology companies would use to find and contract with business development experts who can help these companies \sout{expand into other territories} \uline{complete projects and get deals signed}.}

Interestingly, \nterm{Marketplace} was eventually removed from the \ncn. It was replaced with the term \nterm{product}, which had a simple definition. 

\newterm{Product: Software offered for use to Stakeholders.}

\nterm{Stakeholder} also had its own definition in the \ncn. \nterm{Product}'s definition looks poorer than the one for \nterm{Marketplace}. The problem with \nterm{Marketplace}, was that it was saying too much too loosely about what the software is going to do and who will be using it. It was not loose at the starting point, but became so as we progressed in deciding what exactly the software will be, and how it will work, our commitments about both of these questions became parts of definitions of other terms, which were more specific, and which together gave a clearer idea of the product; eventually, in other words, \nterm{Marketplace} became useless -- it was removed, \nterm{Product} replaced it, together with many other terms which covered, refined, or changed what we initially had in the definition of \nterm{Marketplace}.


\section{Growing the \ncnf}
Starting from the initial definition of \nterm{Marketplace}, we expanded the network to include \nterm{Company}, \nterm{Expert}, \nterm{Project}, and \nterm{Deal}. 

That initial network evolved in two directions in parallel, the service that the product would be delivering, and the stakeholders involved in that service. In addition, many terms were added to capture various kinds of data and actions of the software which supports service delivery.

An important change in stakeholder-related terms happened when we started making the distinction between stakeholders in the domain and the representation of some of them in the software, i.e., the data that the software holds about them. 

We saw this as a distinction between domain and in-product stakeholders, leading us to distinguish what an \nterm{Expert} is in the domain, independently of the data we have about them in the product, if at all, and what we consider, in the product, as a representation of any \nterm{Expert}, what we called \nterm{Expert User}. Here are the two definitions.

\newterm{Expert: Individual looking for an efficient way to find work for leading companies they can help the most, and which will pay high rates.}

\newterm{Expert User: Data that the Product holds about an Expert who registered.}

There is a lot to say about the definition of \nterm{Expert User}, and what needs to be changed in it. Before making such changes, observe the following:
\begin{enumerate}
	\item The definition does not say which data is held about an \nterm{Expert}. Eventually, we need to be precise about this and define all data that we want it to hold. 
	\item Data can be about \nterm{Experts} who registered. The definition is silent on data that we may have about experts in the domain who might want to register, but haven't done so yet. One way to put this, is that the \nterm{Product} does not hold marketing data about experts in the domain whom we may want to market the \nterm{Product}. 
	\item The definition ignores the important distinction between data which the person, the expert provided explicitly, that is, typed in or uploaded to the \nterm{Product}, and data which she generated through the usage of the \nterm{Product}. 
\end{enumerate}

Let's consider these in turn, since addressing each leads to important changes in the definition of \nterm{Expert User}, but also requires decisions from the team which have much broader effects than the definition itself. 

The first and third point are closely related. A first improvement to the definition would be simply to be clear that \nterm{Expert User} is indeed defined through both user-provided and product-generated data about that user's activities. So a first iteration could be the following.

\newterm{Expert User: Data that the Product holds about an Expert who registered\uline{, which covers both data which the Expert provided when registering and later, by inputting or uploading it herself, and the data which was collected through Expert's use of the Product}.}

This is no small change, since it matters for the design decisions about the \nterm{Product}, its legal compliance, and how the terms and conditions for it are written. 

The second point, which says that the \nterm{Product} holds no data about \nterm{Experts} who did not (but might) register, can be reinforced as follows.

\newterm{Expert User: Data that the Product holds \uline{only about Experts who registered to the Product} \sout{about an Expert who registered}, which covers both data which the Expert provided when registering and later, by inputting or uploading it herself, and the data which was collected through Expert's use of the Product.}

We can continue attacking this definition, and progress through additional iterations. One attack that requires more substantial changes, is to ask which exact data the \nterm{Product} will hold about an \nterm{Expert}.

There are many ways to answer that question, and it can be done at various levels of detail. Just how much detail will be needed depends on the expertise and experience of the team members in designing data stuctures for software products, but more importantly, will also change as they understand better the data they want to get and keep, and why, about \nterm{Experts}. Let's look at a few iterations.

The first thing to do, to make this easier, is to introduce new notions for the entities we want to develop in detail and scope further, without putting more weight in the definition of \nterm{Expert User}. We now have two kinds of data to define separately, \nterm{Expert User Profile Data} and \nterm{Expert User Activity Data}.

\newterm{Expert User: Data that the Product holds only about Experts who registered to the Product, which covers Expert User Profile Data and Expert User Activity Data. \sout{both data which the Expert provided when registering and later, by inputting or uploading it herself, and the data which was collected through Expert's use of the Product.}}

Now, the definition of \nterm{Expert User} depends on two other new terms, which we had no prior definition for. They are new terms. We then provide a list of data which we want to collect as part of user's profile. 

\newterm{Expert User Profile Data: Consists of the following data about an Expert: First name, Last name, Contact phone number, Email address, Scan of the photo page in the passport.
}

This definition will not last long. At the time, the team argued that the more data we have, the more value we can provide to the \nterm{Expert}; we can better recommend \nterm{Projects} and identify potential \nterm{Deals} that the \nterm{Expert} can be helpful on. Now, it is important to see that such discussions are not remote or disconnected from the definition, and that any decision we make should be reflected in one or other definition. Here is another iteration of the same term's definition.

\newterm{Expert User Profile Data: Consists of the following data about an Expert: First name, Last name, Maiden name, Location, Summary, Email address, Mobile phone, Services, Technology expertise, Regions, Industries, Accounts, Hourly fee, Deal Commission.}

Expecting that most of these will persist future iterations of the \nterm{Expert User Profile Data}, each had to get its own definition. Needless to say, the network started growing quite quickly, and was far from covering much of the \nterm{Product}'s scope, let alone that of the \nterm{Service} that the \nterm{Product} was to support.

In parallel with finding which data to hold about an \nterm{Expert}, we were considering \textit{how} to get this data; indeed, much of it may simply not be available, or \nterm{Experts} may not go through the trouble of providing it, because they may see no reason for us to have such data about them.

One of the ideas that led to important changes in data definitions, was that much of this data was already provided by \nterm{Experts}, but on other products which may or may not have similar aims as the one we were creating. We identified one such product, called Product-B below, which was open to sharing data as long as the person accepts to have her data shared. 

\newterm{First name: First name, copied from first name on Product-B Profile.}
\newterm{Last name: Last name, copied from last name from Product-B Profile.}
\newterm{Maiden name: Maiden name, copied from maiden name on Product-B Profile.}
\newterm{Current position: Value copied from headline on Product-B Profile.}
\newterm{Location: Value copied from headline on Product-B Profile.}
% \newterm{Summary: Summary, prefilled with summary from Product-B Profile (if available via Product-B application programming interface).}

In addition, we had to ask the \nterm{Expert} for the following data, independently from them being available on a different product. While \nterm{Email address}, or \nterm{Mobile phone number} certainly are common terms, they are specific on the \nterm{Product} in terms of where they are provided by the \nterm{Expert}; provenance makes them specific.

\newterm{Email address: Value input email field of Expert contact Form.}
\newterm{Mobile phone: Value input in mobile phone field of Expert contact Form.}
\newterm{Services: One or more of the following values, selected in the Expert profile form:
\begin{itemize}
	\item Make introductions to clients prospects in your network,
	\item Help clients move deals along by sharing your expertise,
	\item Help clients move deals along by sharing your expertise,
	\item Help clients find resellers or other channel partners,
	\item Help clients find resellers or other channel partners,
	\item Educate clients on entering your market of expertise,
	\item Help clients develop a strategic plan for growth,
	\item Join a sales team on a contract basis.
\end{itemize}
}

We approached in a similar way the definition of other data with which the \nterm{Expert} describes herself, such as \nterm{Technology expertise}.

\newterm{Technology expertise: One or more of the following values, selected in the Expert profile form: CRM \& Sales, Marketing, Accounting \& Finance, Analytics, CAD \& PLM, Collaboration \& Productivity, Content Management, Digital Advertising, E-Commerce, Hosting Services, HR, IT Infrastructure, IT Management, Cybersecurity.
}

The data above came from asking the \nterm{Expert} for it. In addition, recall that we have \nterm{Expert User Activity Data}, which is what we want the \nterm{Product} to collect about the person's activity in \nterm{Projects} and \nterm{Deals}.

\newterm{
	Expert User Activity Data: Includes the following data: 
	\begin{itemize}
		\item Average time between the datetime when Expert User is notified of an RfP, and the datetime when she submits her Proposal to Company,
		\item Maximal (longest) of all times between between the datetime when Expert User is notified of an RfP, and the datetime when she submits her Proposal to Company,
		\item Minimal (shortest) of all times between between the datetime when Expert User is notified of an RfP, and the datetime when she submits her Proposal to Company,
		\item Total number of Projects completed, since becoming Expert User,
		\item Average number of Projects completed per Company since becoming Expert User,
		\item Total number of Successful Projects,
		\item Testimonials,
		\item For each Testimonial, the response which Expert provided to that Testimonial,
		\item Number of Projects in which the Expert is involved and which are not completed,
		\item Total number of Deals completed, since becoming Expert User,
		\item Average number of Deals completed per Company since becoming Expert User,
		\item Total number of Successful Deals,
		\item Testimonials,
		\item Number of Deals in which the Expert is involved and which are not completed.
	\end{itemize}
}

Here is one iteration of definitions of these terms.

\newterm{Average time to respond to RfP: Average time between the datetime when Expert User is notified of an RfP, and the datetime when she submits her Proposal to Company.}
\newterm{Longest time to respond to RfP: Maximal (longest) of all times between between the datetime when Expert User is notified of an RfP, and the datetime when she submits her Proposal to Company.}
\newterm{Shortest time to respond to RfP: Minimal (shortest) of all times between between the datetime when Expert User is notified of an RfP, and the datetime when she submits her Proposal to Company.}
\newterm{Total number of Projects completed: Total number of Projects completed, since becoming Expert User.}
\newterm{Average number of Projects completed per Company: Average number of Projects completed per Company since becoming Expert User.}
\newterm{Total number of successfully completed Projects: Total number of Successful Projects.}
\newterm{All Project Testimonials received since becoming Expert User: Testimonials.}
\newterm{For each Testimonial, a Testimonial Respose: For each Testimonial, the response which Expert provided to that Testimonial.}
\newterm{Ongoing Projects: Number of Projects in which the Expert is involved and which are not completed.}
\newterm{Total number of Deals completed: Total number of Deals completed, since becoming Expert User.}
\newterm{Average number of Deals completed per Company: Average number of Deals completed per Company since becoming Expert User.}
\newterm{Total number of successfully completed Deals: Total number of Successful Deals.}
\newterm{All Deals Testimonials received since becoming Expert User: Testimonials.}
\newterm{Ongoing Deals: Number of Deals in which the Expert is involved and which are not completed.}

Are these definitions precise, accurate, and clear? They are not, and this is probably obvious if you ever wrote software and/or worked with software engineers. Open questions include, for example: Which of this data needs to be validated when it is recorded? What rules should their values be validated against? Which values, for example, cannot be accepted? Which of this data is mandatory, i.e., must be recorded, obtained, or provided by the user in order to register her? Which of this data can be provided once, and never changed again, if any? Is there data which the user provides, but must contact us to change it? And then, there are more technical questions, of database structure, search over this data, its storage, and security, among others. 

If this seems like many new terms, note that this was one of the less complex products that \ncn s were used on. We had about 50 terms that defined the data to hold on an \nterm{Expert}, about 30 terms for data about a \nterm{Company}, and this only kept growing subsequently, as we were considering how the service that the \nterm{Product} supports should be designed. Keep in mind that we also had many additional terms for data about \nterm{Companies}, \nterm{Deals}, \nterm{Projects}, and so on, which took us to a few hundred terms even in the first month or so of the innovation process.


\section{Defining the Service, Project, and Deal}
Terms which focused on data about \nterm{Experts} and \nterm{Companies}, while important, are only there to support the service that \nterm{Product} should help deliver. We saw \nterm{Projects} and \nterm{Deals} as types of services that \nterm{Experts} would be delivering to \nterm{Companies}. Hence the following definitions.

\newterm{Service: Designates the collaboration of one specific Expert and one specific Company using the Product, and under the rules and guidelines set out by Product Provider and accepted by Expert and
Company Users.}

It was important for us early on to capture why we assume \nterm{Experts} and \nterm{Companies} would engage on the \nterm{Product}, that is, what each one expets from participating in \nterm{Service} delivery. This led to the \nterm{Key Expectation} concept.

\newterm{Key Expectation: Indicates the principal reason why a Stakeholder participates in a Service using the Product. That is, designates that which one party (Company or Expert) expects to receive from the counterparty (Expert or Company) in the Service that they participate in.}

We used the \nterm{Key Expectation} to distinguish \nterm{Project} from \nterm{Deal}. 

\newterm{Project: Service in which:
	\begin{itemize}
		\item Company’s Key Expectation is to have the Expert achieve specific Goals on each of the agreed Targets;
		\item Expert’s Key Expectations are:
		\begin{itemize}
			\item To receive Expert Project Fee, and
			\item To create Accounts, which, if the Expert manages 
			% (during the Tail) 
			to convert into Deals, will yield the Expert the payment of the Deal Commission.
		\end{itemize}
	\end{itemize}
}

\newterm{Deal: A Service which can only be done between a Company and an Expert which the Company had initiated and/or completed at least one Project, and in which:
	\begin{itemize}
		\item Company’s Key Expectation is to have the Expert convert an Account into a Deal;
		\item Expert’s Key Expectation is to receive Expert Deal Fee for converting an Account into a Deal for the Company.
	\end{itemize}
}

By now, you know that definitions like these grow the network. We had a definition for each capitalized term there. Below, the focus is on the \nterm{Project}. 

\newterm{Target: Legal entity which may also be a Company on the Product, and to which Experts can connect Companies with, and Company Users may want to have their Companies connected with.}

\newterm{Goal: Designates that which a Company User wants to achieve by purchasing Services.}

\newterm{Project Fee: Fee paid by the Company for a completed Project using the Product.}

\nterm{Project Fee} has a na\"{i}ve definition, as it ignores the question of how \nterm{Product Operator} generates revenue. When it was decided that the broker, which owns the \nterm{Product}, takes a margin off the \nterm{Project Fee} and pays the remainder to the \nterm{Expert}, the definition was revised to the following.

\newterm{Project Fee: Fee paid by the Company for a completed Project on the Product. \uline{Project Fee includes:}
	\begin{itemize}
		\item \uline{Broker Project Fee , which remunerates the Broker;}
		\item \uline{Expert Project Fee , which the Broker pays to the Expert.}
	\end{itemize}
}

\newterm{Broker: Legal entity which owns the Product.}

A \nterm{Project} involves the creation of \nterm{Accounts}, which can yield additional revenue to \nterm{Experts}, of they are converted into \nterm{Deals}. Hence the following definitions.

\newterm{Account: A Target which satisfies the following conditions:
\begin{itemize}
	\item A Company User has approved that if the Target is converted into a Deal, then Expert can claim her Expert Deal Fee to Broker, and Broker can claim the Deal Fee to the Company;
	\item There is a Tail approved by the Company User and Expert User
	\item There is a Deal Fee approved by the Company User and Expert User.
\end{itemize}
}

\newterm{Deal Commission: Monthly payment to Expert for the first 24 months from the date when the Deal was signed.}

\nterm{Deal Commission} expectedly became much more complicated later. Not every \nterm{Deal} lasts 24 months or more, and it is hard to set and agree on the amount of the payment on the basis of the amount agreed in the \nterm{Deal}. 

Terms such as \nterm{Target}, \nterm{Goal}, \nterm{Project Fee} are relevant at different phases of an overall \nterm{Project} execution process. 

As we progressed in the design of the \nterm{Project} execution process, the scope of the problem widened. The team was tasked to propose a process which covered structuring and monitoring of an \nterm{Expert}'s execution of a \nterm{Project} for the \nterm{Company}. In other words, when contract is signed, execution starts, and we wanted to look at how to improve the monitoring of that execution. The motive was that we wanted to use that monitoring to evaluate the reputation of both the \nterm{Company} and \nterm{Expert}, then use these for future recommendations of \nterm{Experts} to \nterm{Companies}, and vice-versa.

At some point in the design, we converged to the following structure of \nterm{Project Execution Phases}.

\newterm{Project Execution Phases are:
\begin{itemize}
\item Call For Help, in which a Company posts an Opportunity;
\item Proposal, in which a Expert responds to the Opportunity;
\item Contracting, in which the Expert and Company negotiate the Targets, Goals, Tasks, and Project Fee;
\item Delivery, in which the Expert does Tasks to achieve Goals, and creates Accounts for the Company;
\item Approval, in which the Company confirms that Goals are met and Project is completed, and provides feedback on the Expert;
\item Continuation, in which the Company decides if and how to continue collaboration with the Expert.
\end{itemize}}

We grew the set of terms, refined it and their definitions, while deciding how the processes should work. That ''how'' means setting rules the process should satisfy. We used the terms to define the rules. Here is a sample, on \nterm{Project} execution.

\begin{enumerate}
\item In \nterm{Call For Help} and \nterm{Proposal} \nterm{Project Execution Phases}, \nterm{Matching Experts} are shown to \nterm{Company} only in an anonymized format, so that their full name and contact information cannot be known to the \nterm{Company}.
\item \nterm{Contracting Phase} lasts one week, with the possibility on demand to extend this by one or more increments of two calendar days each.
\item \nterm{Project} duration is one calendar month.
\item When posting an \nterm{Opportunity}, \nterm{Client} must provide at least one \nterm{Target}, and one Goal for that \nterm{Target}.
\item \nterm{Project Agreement} includes one or more \nterm{Approved Tasks}.
\item In an \nterm{Opportunity Response}, \nterm{Expert} provides, for each \nterm{Goal} in each \nterm{Target}, at least one of the following:
	\begin{enumerate} 
		\item Acceptance of \nterm{Task} as-is, that is, as set in the \nterm{Opportunity};
		\item Rejection of \nterm{Task} as-is;
		\item At least one \nterm{Sub-Task}.
		\item Optionally, number of \nterm{Man-Hours} that the \nterm{Expert} will dedicate to complete the \nterm{Task} or \nterm{Sub-Task};
		\item List of zero or more \nterm{Accounts} of the \nterm{Expert}; 
		\item \nterm{Project Fee}.
	\end{enumerate}
\item In an \nterm{Opportunity Response}, \nterm{Expert} can input relevant experience to the \nterm{Targets}, \nterm{Goals}, and \nterm{Tasks} in the \nterm{Opportunity Response}.
\item An \nterm{Opportunity Response} must include \nterm{Effort Estimate} and \nterm{Hourly Price}.
\item \nterm{Client} should be able to provide feedback and ratings on all \nterm{Projects}.
\item Retainer as a mechanism for extending \nterm{Services} should only be available to \nterm{Client} and \nterm{Experts} who are involved already for several months in an ongoing \nterm{Service}.
\end{enumerate}

These rules show how the terms become building blocks of the artifacts that the team is designing. In turn, rules placed further constraints on the meaning we wanted the terms to have. Rule 8, for example, tells us that an \nterm{Opportunity Response} must specify \nterm{Effort Estimate} and \nterm{Hourly Price}, without these it is not an \nterm{Opportunity Response}. 

This gets us to an interesting question, if rules are part of the language, or are something else, something built through language use. How, in other words, do we separate the \ncnf from the ideas it is used to create, the artifacts it is used to make?

Rule 8 is not the only one leading to this question. Rule 10 says when a \nterm{Retainer} can be used. Should Rule 10 be part of a definition of \nterm{Retainer}? Rule 3 defines the allowed duration of a \nterm{Project}, isn't it, then part of the definition of \nterm{Project}?

To the extent that each rule restricts the interpretation of a term, it must be part of the definition of that term. If that is the case, then would everything written in the \ncnf somehow constitute that language? 

\section{Using \ncnf s with Data and Process Models}
Software engineering research and practice have produced, over the last five decades at least, a number of approaches, methods, abstractions, rules and procedures for how to define why software should do something, how it will do it, and how to implement it so that it can be verified that it indeed does. The ''why'' question concerns the requirements, goals, constraints on the software, while the ''how'' question concerns the data that the software should manipulate, and procedures it will apply to do so.

If you have produced, or even only read requirements models and specifications \cite{}, conceptual models \cite{}, entity-relationship models \cite{}, class diagrams \cite{}, statecharts \cite{}, process models (including, but not limited to workflow models \cite{}, business process models \cite{}, and many more \cite{}), this question probably occurred to you by now: \textit{How are \ncnf s related to these existing models and the underlying modeling languages?}

In the case at hand, the team used its own ad-hoc modeling language to represent processes. The language was defined in order to emphasize several ideas which were judged critical, and which we felt -- at the time -- that we could not represent as easily in text and established modeling languages which team members were aware of at the time, such as the Business Process Modeling Language \cite{} and Unified Modeling Language \cite{}. We were specifically concerned with the following mix of topics:
\begin{itemize}
	\item Which decisions are we asking a user to make?
	\item Do we limit the time that the user has to make these decisions?
	\item What information do we show to the user, for each of these decisions? 
 	\item  Which actions are done by the software, independently of the user's involvement, during \nterm{Project execution}? 
 	\item Which actions happen on the \nterm{Product}, and are therefore observable by the \nterm{Product} (and so we can record data on them), \textit{versus} actions which we expect to be performed outside the \nterm{Product}?
 	\item Do we want to stimulate some actions from the user, by notifying them that they should or can do something at a specific moment in the \nterm{Execution process}?
\end{itemize}

In parallel with the growth of the \ncn, our discussion of these questions led to rough process models, where we would show sequencing of various actions. These actions mentioned terms defined in the \ncn. Figure \ref{fig-process-early-version} shows an early draft, made with an undefined modeling language. 

\begin{figure}[h]
	\centering
	\includegraphics[width=\textheight,angle=90,origin=c]{fig-process-early-version.pdf}
	\caption{}
	\label{fig-process-early-version}
\end{figure}

Later on, we stuctured both the modeling language and the process we were designing. Primitives of that new language, the elements to combine to create models, are shown in Figure \ref{fig-process-proposal-phase}. Any of the rectangles can stand in the place of ''X'' and ''Y'', that is, be in the relationships shown there. 

Figure \ref{fig-process-proposal-phase} shows the sequence of actions and responsibilities of \nterm{Company Users}, \nterm{Expert Users}, and \nterm{Staff} who represent the \nterm{Broker}. Terms which appear in the instances of the modeling primitives are defined in the \ncnf. They are shown as terms written in the various rectangles in that Figure. Many more models like this one were made, revised, and removed during the innovation process.

\begin{figure}[h]
	\centering
	\includegraphics[width=\textheight,angle=90,origin=c]{fig-process-proposal-phase.pdf}
	\caption{}
	\label{fig-process-proposal-phase}
\end{figure}

The same rationale applies to models used to specify the data that software should record and compute on. Conceptual models, class diagrams, and the like, all mention terms which have, in our cases in this book, natural language definitions. This applies not to terms which appear as concept and entity names, but also to those which name relationships, properties. 

Finally, rules which place constraints on instances of classes or entities, and their relationship, can be written using one of many modeling languages (Object Constraint Language \cite{}, a description logic, a formal specification language such as Z or B come to mind). Propositions written in these languages will mention classes, properties, or more generally will mention terms defined in an \ncnf. For example, the following rules place constraints on who sees what on the \nterm{Product}.

\begin{quote}
Product only performs blind matching, which means all of the following:
\begin{itemize}
	\item Company Users must not be able to browse Experts on Parachute.
	Experts must not be able to browse Companies on Parachute.
	\item Company User sees only information about the Company she is associated to.
	\item Expert has only access to information about Opportunities which the Product delivers to her.
	\item Expert has only access to Services in which she is involved. 
\end{itemize}
\end{quote}

These rules could be rewritten as a theory in first-order logic, and thus is specifiable in various languages mentioned above; what matters in this book, is that an \ncnf{} which defines the capitalized terms will enrich and should make it easier to read and discuss such formal models.

In a summary, an \ncn{} does not, and cannot replace a model or specification which you may want to make as part of the software innovation and engineering processes, using an established, or your own \textit{ad hoc} modeling language. \ncn s are meant to precede, complement, and anchor a formal model, by providing an accessible format to capture the meaning intended for the terms which are mentioned in such models. 

% Chapter bibliography
\printbibliography


%%%%%%%%
\chapter{Case: Matching Physicians to Patients}\label{c:case:matching-physicians-to-patients}

\section{Background}
In 2015, owners of a hospital network in the USA were interested into the following problem: specialized equipment (e.g., MR, CT, U/S, X-ray, PET/CT, and so on) was not fully used by the demand within the network, and they were aware that there was demand for more outside of the network. The opportunity they wanted to explore was how to make these resources available for booking to physicians outside their hospital network. A team of analysts, physicians, and product designers was assembled, initially to design a new organization which would focus on matching supply of specialized equipment, that is, its availability, with interested and approved physicians. 

\section{Challenges}
The team saw this as a problem of designing a two-sided marketplace, call it Market-X here (MX hereafter). Early on, scope increased on the basis of initial data collection, and grew beyond equipment bookings. It was described as follows.

\begin{quote}
''MX is a marketplace where the demand and supply of diagnostic and prognostic care meet at scale, efficiency, and cost which have been out of reach in the past.
\begin{itemize}
	\item Supply is the time of human resources and equipment, which are available in X-approved healthcare institutions. These human resources include physicians, but can include other profiles in the healthcare workforce, such as nurses, dentists, etc. 
	\item Demand is made of:
	\begin{itemize}
		\item Referring physicians, who book resources on X for the benefit of their Patients;
		\item Customers, who book resources on X, without having a Referring physician to approve or otherwise influence that booking.''
	\end{itemize}
\end{itemize}
\end{quote}

The normal process by which demand and supply meet, was described as follows:
\begin{enumerate}
	\item Referring physician, previously approved by and registered on MX, accesses MX software.
	\item If the Referring physician has not yet added her Patient’s information and contact details, she adds them. 
	\item Referring physician searches MX and finds relevant Supply side resources, which she decides to book for her Patient. MX uses Patient information and search input from the Referring physician to recommend best fitting resources in search results to the Referring physician.
	\item Referring physician makes a tentative booking of the resources. MX starts countdown until payment and booking confirmation. If countdown completes before Patient makes the payment, resources are freed up and available to others.
	MX notifies Patient that Referring physician has made a tentative booking, and that she needs to accept or refuse that booking. 
	\item Cases:
		\begin{enumerate}
			\item If Patient accepts, Patient makes the payment on X. MX notifies the Referring physician and the booked resource of this booking and shares the relevant Patient information with the booked resource.
			\item If Patient refuses, X notifies the Referred physician, and cancels the tentative booking to free up the tentatively booked resource.
		\end{enumerate}
	\item Patient uses the booked resource, meaning the booked consultation or otherwise takes place. MX allows the booked resource to add results of the consultation. MX notifies the Referring physician and the Patient that they can access the results on X.
	\item MX asks Referring patient to give feedback and score the booked resource. MX asks Referring physician to rank give feedback on the results which the booked resource posted for the Patient on X. X uses the evaluations to describe the quality or reputation of the booked resource.
\end{enumerate}

\section{Growing the \ncnf}
The initial set of terms was the following.

\newterm{Patient: An individual who needs an Appointment, and whose Appointment is booked via MX. There are two types of Patients:
\begin{itemize}
	\item Referred Patient: Patient who has a Referring Physician. The Referring Physician books Appointments on behalf of the Patient on MX.
	\item Self-Service Patient: Patient who does not have a Referring Physician outside MX. Patient needs to book Appointments by herself on MX.
\end{itemize}
}

\newterm{Physician: Medical professional who holds a valid license to practice medicine. There are the following types of Physician Roles on MX:
\begin{itemize}
	\item Referring Physician: Physician who is not part of MX (the business), and uses MX to refer the Patient to a Referred Physician.
	\item Referred Physician: Physician to whom the Patient is being referred via MX, and whom the Patient will have an Appointment with. 
\end{itemize}
}

\newterm{Visitor: Individual who is visiting MX and viewing public MX material (such as website, mobile app, etc.) and who has not yet registered as a Patient on MX.}

\newterm{X Administrators: An individual employed by MX, and who is authorized to make changes to User Rights, Appointments, and more generally any subset of data which MX holds.}

\newterm{Appointment: A limited amount of time with a set start and end, which a Patient purchased from a Physician on MX, and during which the Physician is available to provide diagnostic and prognostic care to the Patient.}

\newterm{Appointment Slot: A limited amount of time with a set start and end, which a Physician offers for sale on X, and during which the Physician will be available to provide diagnostic and prognostic care to one Patient.}

\newterm{Equipment: any clinic or hospital capital equipment which is available to Physicians for use during their Appointments. Examples are MR, CT, U/S, X-ray, PET/CT, Nuclear Medicine, resources used in Blood and Chemistry tests, tissue pathology, and Genetic Profile Analysis, among others.}

\newterm{Customer: a synonym for Patient.}

\newterm{MX Health Record (XHR): the electronic health record of the Patient on MX.}


It turned out that many \nterm{Patients} wanted to see \nterm{Physicians} without this being initiated by a referring \nterm{Physician}. In such cases, it was necessary for \nterm{MX}, the business, to provide a referring \nterm{Physician}. This also meant refining the definition of \nterm{Physician}, to include so-called \nterm{MX Physician}, giving the following new \nterm{Physician} definition.

\newterm{Physician: Medical professional who holds a valid license to practice medicine. There are the following types of Physician Roles on MX:
\begin{itemize}
	\item Referring Physician: Physician who is not part of MX (the business), and uses MX to refer the Patient to a Referred Physician.
	\item Referred Physician: Physician to whom the Patient is being referred via MX, and whom the Patient will have an Appointment with. 
	\item MX Physician: Physician employed by MX, and who acts as a Referring Physician for every Self-Service Patient on MX.
\end{itemize}
}

Beyond simple cases such as the above, an interesting question was how to define a \nterm{Physician}'s specialties, as well as procedures and tests that this \nterm{Physician} may be most competent to refer \nterm{Patients} to. The definition of \nterm{Physician}, even the refined one above, is not helpful in this respect. 

Each \nterm{Physician} thus needed to be described by the medical specialties they were competent in, following standard naming conventions \cite{wikipedia-specialty-medicine}. But this was not enough, if we wanted to allow patients to look for physicians by other means, such as:
\begin{itemize}
	\item Procedure or test, such as ''Amniocentesis'';
	\item Equipment, such as ''PET'';
	\item Desired Appointment slot, as one or more dates and time ranges;
	\item Geographical location where the Supply Physician holds appointments, such as a city, ZIP code, address, etc.;
	\item Maximal distance from Patient’s address.
\end{itemize}

Adding these properties to the definition of \nterm{Physician} is easy, but choosing the set of procedures and equipment is not. It was important to recognize that these choices could not be made definitely during the innovation process. New equipment and procedures can become available after the initial version of the service is released and used. This led to the need to design functionality that allowed \nterm{MX Administrators} to add new procedures and equipment.

What if we chose the following properties for Physicians?
\begin{itemize}
	\item First name,
	\item Last name,
	\item Email address,
	\item Direct medical messaging address,
	\item Contact phone,
	\item National Provider Identifier,
	\item If physician acts as Supply Physician, Demand Physician, or both on X,
Medical Licenses.
\end{itemize}

This depends -- if we want these alone, then the relationship of the physician to medical specialties needs to happen via medical licenses, i.e., the definition of medical licenses needs to be done via -- among others -- via medical specialties. 

What about equipment? Perhaps it needs to be related to medical procedures in which it can be used. Either way, note how the choice in the definition of one term influences definitions we have to make for others. Note, also, that there may be established definitions of physicians, such as the following one, but in the context of this innovation project, the term Physician needs to be more specific, and not only a refinement of the following definition, but one which has both a different scope and depth, which in turn makes it part of the \ncnf, not of the base language.

\begin{quote}
Physician: A person qualified to practise medicine, especially one who specializes in diagnosis and medical treatment as distinct from surgery. \cite{def-physician}
\end{quote}

Further work on how \nterm{Physicians} would be using such a service, led to the observation that they will most likely delegate many of the responsibilities we expected them to fulfil, such as indicating their availability. It was thus necessary to introduce so-called \nterm{Physician Staff}, as follows.

\nterm{Physician Staff User: MX User who was invited by a Physician to MX, and will accomplish tasks (registering Patients, managing Appointment Slots, etc.) on behalf of Physician on X.
}

This was only an early definition, as it left open how delegation happens, that is, how the software can know which \nterm{Physician Staff} is related to which \nterm{Physician}. That in itself was done by defining a process by which the delegation relationship is established. The following is an early draft of the process.

\newterm{Delegation Process:
To allow Physician Staff User on X to act on her behalf, a Physician User proceeds as follows
\begin{enumerate}
	\item Physician User logs into MX.
	\item Physician User chooses Staff.
	\item Physician User chooses to invite new Physician Staff Users.
	\item MX asks the Physician User to provide Physician Staff User Registration Properties.
	\item Physician User fills out the information and clicks to submit it to MX.
	\item MX does the following:
		\begin{enumerate}
			\item MX changes the Physician Staff User’s MX Status to Awaiting;
			\item MX sends New Physician Staff Password Setup email to the Physician Staff.
		\end{enumerate}
	\item Physician Staff receives the New Physician Staff Password Setup email. 
Physician Staff clicks on the link in the email.
	\item MX opens in Physician Staff User’s web browser, and asks Physician Staff to enter twice a new password, which Physician Staff User wishes to use to access MX.
	\item Physician Staff User submits the password.
	\item MX checks if the password is strong enough, and if not, requests a new password from Physician Staff User.
	\item If the chosen password is acceptable to MX, MX performs the following:
		\begin{enumerate}
			\item MX registers Physician Staff as a Physician Staff User on MX.
			\item MX changes Physician Staff User Status to Active.
			\item MX gives Physician Staff the User Rights chosen by the Physician who invited the Physician Staff User.
			\item MX sends an email to the Physician Staff User, to confirm that she is registered to MX.
			\item MX sends an email to the Physician User who invited the Physician Staff User, to notify Physician User that Physician Staff User is registered and Active on MX.
		\end{enumerate}
\end{enumerate}
}

Notice how processes, such as the one above, extend the scope of the \ncnf. The design of such processes signals the need for new functionality and new artefacts that the software should have and manage, such as registration, statuses, email messages, and so on. As we will see in later Chapters, design that proceeds in this way, during the innovation process, literally branches out the \ncnf, where by applying the actions on definitions, we identify the need for new terms.

% Chapter bibliography
\printbibliography


%%%%%%%%
\chapter{Case: Defining a New Product Development Process}\label{c:case:defining-new-product-development-process}

\section{Background}
In 2012, after experiencing success with its initial product, a manufacturer of designer low energy light bulbs wanted to expand its product line. To do so, its owners wanted to get a better understanding of its existing new product design and development process. This would help them plan the investment in new products, and understand what to improve and how, in the process itself.

\section{Challenges}
The approach was to document the existing new product design process, which proved successful, before considering any changes. This meant understanding who does what, and how they coordinate, from new ideas to having a definite design sent for manufacturing and distribution. 

To understand how work was done, and why, and so be able to document the existing process, we had to understand the language which they used to speak about that process, and to coordinate throughout that process. It was a new language, partly because the process was specific to the team there, but also because it was invented as they went. Both of these were expected. Design and manufacturing of new products, and in this company's case, of a product widely recognized as highly innovative, involves innovation at the level of the product, but also of the organization which makes that product happen, so to speak, from idea to the customer. 

This is also a case which shows how widely-used terms get a local meaning that makes very much sense to the team, but not necessarily to outsiders. There are few new terms in their \ncnf, but definitions aren't standard at all.

\section{Growing the \ncnf}
Initial interviews led to the following simple set of terms, where many definitions are rather straightforward, and reflect the common-sense understanding of these terms.

\newterm{Product: That which \nterm{Clients} purchase from Company X.}

\newterm{New Product: Product that Company X plans to, but has not yet started selling.}

\newterm{Process: Sequence of Activities performed to achieve a particular objective.}

\newterm{Activity: Meaningfully related Steps within a Process (e.g., these Steps realize related tasks in a Process).}

\newterm{Step: Smallest part of an Activity, defining a task that a single Role needs to accomplish.}

\newterm{Process Guideline: A recommendation on how to perform Steps in Processes.}

\newterm{Role: Set of responsibilities that the individual playing the Role has to discharge in one or more Processes.}

\newterm{New Product Prototype: A preliminary model of the New Product.}

\newterm{New Product Packaging Prototype: A preliminary model of the packaging for the New Product.}

\newterm{Product Developer: Individual with expertise in Product design and development.}

\newterm{Product Specifications Designer: Individual with expertise in making specifications for Product manufacturing.}

\newterm{Product Manufacturer: Individual representing the company capable of manufacturing the New Product for Company X.}

\newterm{New Product Brief: Document specifying the requirements that the New Product should satisfy.}

\newterm{New Product Specifications: Document defining the specifications of the New Product which satisfies the requirements given in the New Product Brief.}

\newterm{Manufacturer Estimate: Document by which the Product Manufacturer responds to New Product Specifications.}

\newterm{New Product Packaging Design: Document describing the design of the packaging for the New Product.}

Subsequent discussions of how the current product was designed, how and why the design changed, and eventually, how it went into manufacturing, led to many changes. The term \nterm{New Product Prototype} was not precise enough, and was removed. It's role was filled by two new terms, \nterm{First Sample} and \nterm{Final Sample}, defined below.

\newterm{First Sample: A first and preliminary model of the New Product.}

\newterm{Final Sample: A final model of the New Product, accepted by Company X.}

\nterm{New Product Packaging Prototype} was removed. It was replaced by \nterm{New Product Packaging Brief} and \nterm{New Product Packaging Design}.

\newterm{New Product Packaging Brief: Document defining requirements on packaging for the New Product.}

\newterm{New Product Packaging Design: Document specifying the packaging design for a Product.}

The term \nterm{Product Developer} changed. It shifted from vaguely pointing to skills needed, to a list of responsibilities in the new product development process.

\newterm{Product Developer:
\begin{itemize}
	\item Produces New Product design concepts;
	\item Presents design concepts to Creative Director and Managing Director;
	\item Adapts design concepts until Creative Director and Managing Director approve a concept.
\end{itemize}
}

This points again, as in Chapter \ref{c:case:contracting-process}, to the problem of deciding where a definition starts and stops; in other words, what is part of a definition, and what isn't? The following is an early account of the initial steps of new product development process. We put it as a new term in the \ncn, and wrote it in text, but we could have also made a process model, using a workflow or business process modeling language, and had that as the definition in place, or alongside even, of the textual one below.

\newterm{New Product Development Process:
\begin{enumerate}
	\item Creative Director and Managing Director discuss New Product ideas.
	\item Managing Director initiates research on the New Product.
	\item Creative Director, Managing Director, and Product Developer narrow down the requirements to include in the New Product Brief.
	\item Managing Director:
		\begin{enumerate}
			\item Produces the New Product Brief;
			\item Obtains from the Creative Director the approval of the New Product Brief;
			\item Sends the New Product Brief to Product Developer.
		\end{enumerate}
	\item Product Developer:
		\begin{enumerate}
			\item Produces New Product design concepts;
			\item Presents design concepts to Creative Director and Managing Director;
			\item Adapts design concepts until Creative Director and Managing Director approve a concept.
		\end{enumerate}
	\item Managing Director updates and sends New Product Brief to Product Manufacturer.
	\item Product Manufacturer responds to Managing Director on New Product Brief.
	\item Creative Director, Managing Director, and Product Developer revise, if needed the New Product Brief.
	\item Managing Director sends revised New Product Brief to Product Manufacturer.
\end{enumerate}
Steps 8 and 9 are repeated until the Product Manufacturer can provide the Manufacturer Estimate to the Managing Director.
\begin{enumerate}
\setcounter{enumi}{9}
	\item Product Manufacturer delivers Manufacturer Estimate to the Managing Director. Manufacturer Estimate includes estimates of:
		\begin{enumerate}
			\item Product development cost;
			\item Product development timeline;
			\item Minimal order size;
			\item Estimated unit cost.
		\end{enumerate}
	\item Creative Director and Managing Director decide whether to accept the Manufacturer Estimate; 
		\begin{itemize}
			\item If no, contact another Product Manufacturer and go back to Step 6;
			\item If yes, go to next Step.
		\end{itemize}
	\item Product Manufacturer:
		\begin{enumerate}
			\item Produces First Sample;
			\item Delivers First Sample to Managing Director.
		\end{enumerate}
\end{enumerate}
}

More steps followed. What is interesting, is how the process defines the terms it mentions. Step 5 defines the responsibilities of the \nterm{Product Developer}. Step 10 defines the content of the \nterm{Manufacturer Estimate}.

How does this process description relate, then, to definitions of the terms it mentions? Do we even need to have separate definitions, if we have descriptions of various processes? 

Eventually, many of the generic early definitions got refined through the definition of various processes, as above. For example, the definition of \nterm{Product Manufacturer} does not mention what exactly a \nterm{Product Manufacturer} is responsible for, in the \nterm{New Product Development Process}. But the process is more specific, and refines the definition by, for example, saying in Step 10 that a \nterm{Product Manufacturer} is responsible for producing the \nterm{Manufacturer Estimate}. 

In practice, new product development processes are rarely as structured, and steps as clear cut.  This makes it harder to settle on definite definitions, but only emphasizes the need for openness to change. It also highlights that you can never really understand the term, if you do not know where and how it was used, since through usage, or mentions, it gets refined, choices are made, parts are added, or removed. This openness to change cannot only be grasped by looking at snapshots of definitions, as I gave them above, but also through changes of the artifacts which mention the term, as in the partial description of the new product development process above. That is a thorny batch of issues, which will keep coming back throughout this book.

% Chapter bibliography
\printbibliography


%%%%%%%%
\chapter{Case: Defining an Automated Running Coach}\label{c:case:defining-automated-running-coach}

From 2011 to 2013, the \ncnf in this case went from a dozen or so terms, to about a hundred. Not only was it used in verbal and written communication, but also in algorithm specifications, software specifications, software code, and marketing material for the service which this software was enabling. It is a rare case where the \ncn{} was built for completely new ideas, and graduated to be included in most artefacts used and produced by the company who commissioned the underlying innovation.

\section{Background}
The aim of the innovation process was to design algorithms which would provide personalized advice to amateur runners, interested in improving their performance, and reaching specific goals. For example, a goal would be to run a certain race which is scheduled to happen in a few months or later. There were no such algorithms at the time, and invariably, no software which could provide such advice.

\section{Challenges}
The goal of the innovation process was hard to achieve. Providing running advice cannot be reduced to a machine learning problem. There is not enough historical data on the progression of a given runner, or similar runners, from low to high performance,. There are various ways to improve one's performance, i.e., different training methods that coaches subscribe to. These methods are rarely well documented. The outcome of the innovation process was a combination of expert rules and learning algorithms. Advice is generated on the basis of specific coaching methods, and for an individual runner, that advice was parametrized according to the learning of how that runner responded to advice in the past, and prediction of how she may respond to advice in the future. 

\section{Sample from the \ncnf}
The following is a sample of the more stable terms and their definitions. 

\newterm{Athlete: 16 or more years old person wishing to use Running Algorithm X.}

\newterm{Coach: individual controlling the parameters of Running Algorithm X.}

\newterm{Algorithm: Running Algorithm X.}

\newterm{Module: self-contained part of the Algorithm performing a related set of operations.}

\newterm{Session: period of training associated to exactly one day in the calendar.}

\newterm{Training Plan: sequence of Sessions.}

\newterm{Athlete Level: set of Athletes sharing similar Athlete properties and receiving similar training advice.}

\newterm{Athlete Property: characteristic obtained from the Athlete at registration for Running Algorithm X. Athlete Properties are used to compute the Athlete level.}

\newterm{Test Time: time Athlete inputs for each Test Distance.}

\newterm{Test Distance: distance selected by Coach, and for which Athlete should provide Test Time.}

\newterm{Injury Type: injury that the Athlete selects from the Injury Type List.}

\newterm{Zone: quantification of the level of intensity to maintain during a Session.}

\newterm{Target Pace: running speed to achieve and maintain in a Session.}

In the very first months of this innovation process, few of these terms had stable definitions. \newterm{Athlete} could be anyone, without age restrictions. It was unclear if any tests should be done before a person could use the algorithm, and the notion of \nterm{Test Time} came only later. The role of injuries became critical, but only after normal usage of the algorithm was better understood. 

To help someone improve performance in running, a coach needs to be able to measure performance, plan how to vary future stress, decide how and when to apply planned stress, measure outcomes, and plan subsequent sessions, while taking into account measured outcomes. All this needs to be placed in context of the overall performance gain that this person is aiming for, itself a function of the goal that the person set to herself. Linear increase in stress does not correlate positively with increase in performance. And it does not make sense to do the same thing over and over; it does not produce performance improvements.

For coaches, it is straightforward to think in terms of a ''training plan''. As in the definition above, it is indeed a sequence of sessions. But even if that definition fits the general idea, it is not operational at all -- it does not tell you why some sequences of sessions are good training plans for a person, but not for another, for example. When you need to automate the creation of training plans, such questions become central. What are the parts of a training plan? Should sessions be grouped, why, and how?

Eventually, we used the notion of \nterm{Training Period}. The following is an informal, non-mathematical description of something called the \nterm{Training Macro-Structure} of a training plan, itself a new concept.

\newterm{Training Macro-Structure:
A Training Plan has the following structure for every Athlete who is neither Injured nor a Starter:
\begin{enumerate}
	\item A Training Plan is made of Periods, including Periods called: Base training, Race training, Taper and Recovery; each Training Plan is specific to a Training Goal and to the Athlete Level; (an Introduction period can be defined for Athletes with Starter Athlete Level);
	\item A Period is made of one or more Blocks;
	\item A Block is made of Sessions. 
\end{enumerate}
}

A convenient way to think about the structure of a \nterm{Training Plan} is that a \nterm{Session} is assigned to a day and that a \nterm{Block} corresponds to a week’s worth of training; a \nterm{Block} does not exactly correspond to a calendar Monday-to-Sunday week, because the first \nterm{Block} does not necessarily begin on Monday. The \nterm{Period} can last up to a few months.

\begin{quote}
The Macro Training Plan Setup Module allows the \nterm{Coach} to define as many \nterm{Training Plan} structures as she wishes. The \nterm{Coach} defines a new \nterm{Training Plan} by giving it a name, and associating the \nterm{Training Plan} with \nterm{Athlete Levels} and \nterm{Training Goals} for which the \nterm{Training Plan} should apply. This allows the \nterm{Coach} to define different \nterm{Training Plans} for different \nterm{Training Goal} and \nterm{Athlete Level} combinations. In this module, the \nterm{Coach} can define new \nterm{Training Plans}, and define the \nterm{Periods} of each \nterm{Training Plan}. For each \nterm{Period}, the \nterm{Coach} defines the name of that \nterm{Period} (e.g., \nterm{Taper}) and defines which \nterm{Period} precedes it and which other \nterm{Period} follows it in the \nterm{Training Plan}. The \nterm{Coach} uses other \nterm{Modules} to define \nterm{Blocks}.
\end{quote}

The text was part of an early version of an explanation, for how one of the many modules works, within one of the various algorithms works. The text is filled with capitalized terms, each having its own definition in the \ncnf that we were creating. It makes the text hard to understand in absence of definitions. It is an example of how the \ncn{} becomes an important tool for the team. 

The \ncn{} then appears in pseudo-code of the algorithm, the database structure, and software code, ensuring consistency across the many artefacts. The following is a piece of pseudo-code from the specification, followed by the definitions of the terms which appear in it.

\small{
\begin{verbatim}
IF ( interval_distance(u,s) = 0 ) 
  THEN { training_load(u,s) <-- empty; }
ELSE {
    IF ( interval_time(u,s) = 0 ) 
      THEN { training_load(u,s) <-- empty; }
    ELSE {
        IF ( interval_recovery_pace(u,s)
                < interval_distance(u,s) 
                / interval_time(u,s) )
          THEN {
            IF ( interval_recovery_pace(u,s) < 0 ) 
              THEN { training_load(u,s) <-- empty; }
            ELSE {
                training_load(u,s) <-- 
                actual_DTERM(u,s) 
                * actual_Q(u,s) 
                * actual_Density(u,s); }
            }
        ELSE { training_load(u,s) <-- empty; } 
        }
}
\end{verbatim}}

\newterm{Interval Distance: distance in meters to run in an Interval in a Session.}

\newterm{Interval Time: Time it takes the Athlete to complete a given Interval Distance.}

\newterm{Training Load: quantification of effort during training. Training Load is function of Zone, Interval Distance and Interval Count. Training Load is computed using the Training Load Module.}

\newterm{Interval Recovery Pace: pace to run between interval runs.}

The names \verb|actual_DTERM|, \verb|actual_Q|, \verb|actual_Density| are defined in the remainder of the algorithm specifications, as functions over other variables there.

The example above, even if trivial, illustrates the important idea that an \ncnf{} can be a significant record of the content of the innovation process, and as such appears throughout the various artefacts an innovation process may produce.


% Chapter bibliography
\printbibliography


%%%%%%%%
\chapter{Growing \ncnf s}\label{c:case:defining-new-concept-networks}
An \ncnf, as its name says, is a network. Its nodes are term and definition pairs; there is a link from a node A to another node B if the definition in B mentions the term in A. Any node's two parts correspond to two parts of a definition, called definiendum and definiens.
\begin{itemize}
        \item \textit{definiendum} (plural is \textit{definienda}), also known as \textit{term}, which is a word or phrase that you use to name the thing or idea you are defining, and
        \item \textit{definiens} (\textit{definientia}) is an account about the things or ideas you are defining; it should ideally be such that, if you know the definiens, and you see an actual instance of the thing it defines, or you think the idea it defines, then you will know how to name it, that is, you will know that this is something to call by the term.
\end{itemize}

What the definition does, is state states the equivalence of these parts: if you and I agree on a definition, then we agree that, whenever either of us mentions the term, we also mean the definiens we agreed for it, not something else. To make an \ncn, then, we have to make and relate definitions.

A definition exists within a community of people. There are definitions which will work for large communities; if you look at an ordinary encyclopedia, it will include definitions that are widely accepted.

The reason I mention this, is that the definition is \textit{local} to that community. Again, if the community is large, then there are not many interesting implications to saying that a definition is local. But the smaller the community, the harder it is to say that the definition they take for granted will be agreed on by others. 

A definition, then, comes from consensus within a community. In practical terms, this consensus on a definition is useful primarily because decisions and actions that people in that community take, should not contradict that definition.

Here's a blunt example: if you and I agree that ''public transportation'' is defined as ''buses, trains, subways, and other forms of transportation that charge set fares, run on fixed routes, and are available to the public'', then I shouldn't think that I was using public transportation when you drove me to work in your car.

Most people get in touch with some form of public transportation sooner or later, and independently from the experience itself, they encounter the term, and get an explanation for it at an early age. 

Agreement on a definition is part of the bigger picture, which is the \textit{coordination} within a community. We can work together if we have an agreement on words we use and the meaning we have for them. Even if nuances will inevitably vary -- I know less about public transportation than a driver of a public bus -- I know I should pay for using public transportation, and she knows I know I have to do it, for example. Even if her and I never needed to confirm our mutual agreement on a general definition of public transportation, our actions will, or will not be consistent with that, and many other definitions that are accepted, even if implicitly, in a community.

But even these general definitions came about either because there is a potential for misunderstanding, or there was actual misunderstanding, which mattered only so far as it led us to actions which somehow clashed with whatever may be the usual functioning of that community. 

Disagreement, as I wrote in Chapter \ref{c:introduction}, has that value of forcing us to make and change definitions, especially on things and ideas which frequently play a role in our individual and joint decision-making and action. 

New ideas are rarely great on arrival. They need discussion, testing, refinement, revision. The push for any of this comes from disagreement. You are doubting I can make fuel from rainbows, so what shall I do? Either ignore you from now on, which means cease collaboration, or do something more about that, and thereby change what I was thinking about rainbows as fuel in the first place.

An \ncn{} includes a set of terms. No two terms in it are the same, no single term can have more than one definition, and no single definition can apply to more than one term. Anything else will only be confusing.

If we look only at the nodes of an \ncn, it looks like a special kind of glossary. In it, each term must be itself new, or in any case must have a definition which identifies \textit{new} indeas and, or things. Novelty of the term, that is, the fact that the name of things and ideas is new, is not that important. It is, however, important that the definitin points to new things and ideas. Take these two definitions of ''grizzly bear'':
\begin{itemize}
    \item ''The grizzly bear, also known as the North American brown bear, is a large population of the brown bear inhabiting North America.'' \cite{wiki-grizzly-bear}
    \item ''Software to make, which will be used to record sightings of grizzly bears.''
\end{itemize}
Same term, ''grizzly bear'', but two completely different definitions, which point you to different things. The second one refers to software which is new: it needs to be built. The first one cannot be in an \ncn.

In an \ncn, \textit{all} terms have definitions specific to the problem or topic you are working on, and all definitions are new. While names of terms can come from anywhere you chose, these terms will be part of an \ncn{} only if their definitions were newly made for the purpose of addressing the problem, or working on the topic you are interested in. This is why it is about new concepts -- definitions will have come from an innovation process you are involved in, where novelty requires new terms and definitions, or sticking new definitions on old terms. 

My experience has been that an \ncn{} is never made in one step, and is never done. Making it in a single step would mean that you write all new terms and their definitions, and that's it; in other words, you know it all up front, which also means that the glossary was not made as you progressed through your innovation process, but was documented at its end. As we saw in the Chapters \ref{c:case:contracting-process}-\ref{c:case:defining-automated-running-coach}, you start from one or two new terms, write a rough version of their definitions, this points you to other new terms to define, and you iterate as long as resources allow you. It stops inevitably, but not because the glossary was perfect; rather, the problem may be consdered solved, or we may abandon its resolution, or for any other reason that makes it irrelevant to invest in improving the \ncn{}.

An \ncn{} is made and improved as follows:
\begin{enumerate}
    \item Start an \ncn{} from scratch in the following two steps:
        \begin{enumerate}
                \item Identify one or more frequently used terms in your innovation process.
                \item For each frequently used term, find a general definition for it, check if that definition fits how it is used in your innovation process, and if usage differs from the definition, then the term requires a new definition and you should add the term to the \ncn.
        \end{enumerate}
    \item Improve the \ncn{} by repeating the following steps: 
        \begin{enumerate}
            \item For each term:
                \begin{enumerate} 
                    \item Verify if the term's definition fits how the term is used in the innovation process, and if not, update the definition.
                    \item Identify, in the term's definition, all other terms which are mentioned, and which are \textit{not} defined in the \ncn; for each of these terms, find a general definition, check if the term's usage in the innovation process fits its general definition, and if not, add the term to the \ncn{} and make a new definition for it.
                    \item Evaluate the confidence in the term's definition, and identify actions to take to improve it.
                \end{enumerate}
            \item Across all terms in the \ncn, prioritize actions which improve confidence, and decide which to invest it. As its result, change the term or terms it applied to.
        \end{enumerate}
\end{enumerate}

The process has two big pieces. If you want to start from scratch, Step 1 suggests how to identify one or few first terms that will make up the \ncn. Step 2 then improves the glossary from there. Step 2.a says what to do with each individual term; Step 2.a.iii gives an evaluation of confidence, and gives ideas on how to improve it for a single term. Since there are several, perhaps many terms, there will be many potential actions to improve the confidence of them all, and thus we need to look across the entire \ncn{} to prioritize improvements to confidence.

% Chapter bibliography
\printbibliography



%%%%%%%%%%%%%%%%%%%%%%%%%%%%%%%%%%%%%%%%%%%%%%%%%%%%%%
% PART II
%%%%%%%%%%%%%%%%%%%%%%%%%%%%%%%%%%%%%%%%%%%%%%%%%%%%%%
\begin{partbacktext}
\part{Rationale for \ncnf s}
\label{pt-2}
\end{partbacktext}

%%%%%%%%
\chapter{New Terms For New Ideas}
%\chapter{When innovating, which terms do we use?}
\label{c:new-terms-for-new-ideas}

What are \ncnf s? Who creates them, and why? Why is it interesting to analyze and improve innovation languages? I argue in this chapter that teams tasked with innovation -- the invention of new ideas and things, and the transformation of these into something useful for others -- always create new terms in order to communicate about, and coordinate when working on new ideas. I define an \ncnf{} as a combination of new terms, definitions of new terms, and relationships among these definitions, and of these definitions to ''old'' or established concepts. I argue that the analysis of an \ncnf{} is relevant if we are interested in how innovation is done and in how to improve innovation processes.

\section{Say that again?}
\label{c1:s1}
\begin{quote}
We're gonna use the best pointing device in the world. We're gonna use a pointing device that we're all born with - we're born with ten of them. We're gonna use our fingers.
We're gonna touch this with our fingers. And we have invented a new technology called multi-touch, which is phenomenal. It works like magic.'' \cite{Wright2015a}
\end{quote}

The quote is from Steve Jobs' 2007 MacWorld keynote. He was introducing the first generation iPhone. This was the official start of a product line which, more than a decade later continues to shape communication, work, and entertainment of hundreds of millions of people. 

At that time, only few people had a notion of what he might have meant by "multi-touch", "swipe", ''home button'', ''you can touch your music''. Most of the audience was seeing the first iPhone for the first time. As he was demoing it, it became clearer what he meant by multi-touch, swiping, and so on. 

Why did he use these new terms? Why couldn't he use plain English, so to speak? 

The reason should be obvious. He was showing something new. Isn't it also evident that he had to have new terms to talk about it? It certainly seems right that new ideas and things merit new names, and as a result, we must extend whichever language we use with these new terms. 

What role do new terms play in innovation processes? How do we create them? How can we analyze them, to help us progress in innovation? The Part of the book focuses on these questions.

\section{How Do We Talk About New Ideas?}
\label{c1-s2}
When people do innovation, they create new words and phrases to talk about new ideas and things that they are creating.

While I was neither an observer, nor participant in innovation processes that led to the first iPhone models, I did have such roles in many others over the last decade in various companies. Each time, teams which did innovation, also made and spoke their own new language, pieced together from a general-purpose language, relevant technical languages\footnote{By technical language, I mean a set of terms and definitions which may have been new in the past, but are now commonly used when dealing with specific kinds of problems and tasks, in the specialities of medicine, engineering, and many other fields of research and industry.}, and new terms. These new terms were needed to communicate about new ideas created, changed, and thrown away during the team's innovation process. 

If we take that observation seriously, then it must be possible to learn something more about how innovation is done, and how it could be done, by looking more closely at the terms that teams use to communicate about, and coordinate around new ideas. 

We should be able to understand and change innovation processes if we take a closer look at these \ncn s: extensions to ordinary and technical languages, that teams responsible for innovation create as the work through new ideas.

\section{How to Spot New Terms in the Field?}
\label{c1-s3}
New terms are an artefact of an innovation process, in the sense of being something made and changed throughout that process. 

It does not matter how exactly they were made: it could have been accidental or intentional, created in methodical or messy ways. But, my claim and premise is that \textit{new terms are made every time innovation is done, as long as innovation required collaboration}; it certainly did happen in all innovation processes I was involved in. You can test this yourself, that is, if people use new terms to talk about new ideas and things. 

If you are doing innovation, ask someone from outside your team to join a typical team meeting, and ask them how much of what was said they understood. Was there any word or phrase that was new to them, and if so, are these not already in an established technical language? If not, they are part of your team's \ncn. Alternatively, were there old words and phrases, but which are used with a new meaning?

If you are not a participant, ask to sit in a few meetings of the innovation team. Ask yourself these same questions. There will be ideas you understood roughly just like everyone else, and much of that would be communicated through the base language, the general-purpose language such as English, French, or other, that you and they all speak. Then, there will be established technical terms and phrases, which rely on, and convey information and knowledge that both them and you (as well as others) have, or can have access to. But then, there will be new words and phrases, with their own definitions, which do not exist as such outside that innovation team - this is what makes a big part their innovation language (but not all of it, as I will explain later); if these add-ons, to the language they use anyway, is lacking, how could they possibly be using old language to talk about new ideas and things?

\section{Why Are There New Terms?}
\label{c1-s4}
New terms are made for \textit{two reasons} at least; one, it is a tool for communication between those involved, and therefore \textit{critical for their coordination}, and in turn, it must influence how innovation processes unfold; two, it reflects these people's intentions, assumptions, and knowledge about the new ideas and things they work on, and relationships with old ideas and things, \textit{telling us about what is, and is not new in the outcomes of innovation}.

To work together, we have to coordinate, and we can only coordinate through communication. This, in turn, forces us to create terms with which we can refer, or talk about, ideas and things. For most obvious things, and many ideas which are, so to speak common knowledge, we already have natural language and its broad terminology.

But \textit{how can we use old terms to refer to something new}? We can use them: either by redefining them, or else it will be confusing. In the former case, when we \textit{redefine}, old terms remain syntactically the same, but their meaning has changed, and so we are mistaken to think that we are still using the same language. Either way, coordination around new ideas and making new things together will inevitably lead those involved to create extensions of the underlying general-purpose language (English, French, and so on), that they use for communication. This extension is called an innovation language, and its emergence, design, and analysis merit attention.

\section{Why Are \ncnf s Interesting?}
\label{c1-s5}
We create innovation languages when we do innovation, and studying them can help us understand and improve how we do innovation. If this is self-evident to you, then you might want to skip the rest of this chapter.

But it is not a simple claim to make, as it touches on many old and much-discussed topics in philosophy, linguistics, and innovation. The rest of this chapter elaborates the argument, in the following steps.
\begin{enumerate}
	\item \textit{New ideas, just like old ones, are inaccessible to others}. I can access my ideas in some way (whichever the specific biological, psychological or other mechanisms involved), but I cannot access yours \textit{in the same way}; the way you can access yours is not the way for you to access mine.
	\item Since others' ideas are inaccessible, \textit{communication is about ideas, not of ideas}. I cannot be certain that you have the same ideas as I do, or that my are the same as yours.
	\item Communication being \textit{about} ideas, the only way to ascertain that your and my ideas are aligned, is that my actions signal to you my alignment, and that yours signal yours to me. \textit{Actions and outcomes of actions are accessible, and are evidence for alignment in ideas}.
	\item To align on ideas, we need to communicate about them in such ways that our communication leads to actions and outcomes which signal to us something about presence or absence of that alignment. Therefore, \textit{if we need to get aligned on new ideas, as we do need to do in innovation, we need new terms for these new ideas}.
	\item But since others' new ideas, just like their old ones, are inaccessible to you, you cannot simply and only add new terms to the natural language you are using in communication (e.g., English, French, etc.). If we were only to add new terms, their meaning would not be clear enough to others, even if my new terms are clear to me. We also need to establish reference between the new terms and these ideas in a way which others can access, which makes it necessary to produce definitions of these new terms, to make it clear what they refer to.
	\item By this point, we have agreed that coordination during innovation requires new terms \textit{and} their definitions. Those two are still \textit{not} an \ncn. This is because they are not standalone: definitions will use a mix of old and new terms, i.e., a definition leans on the underlying natural language. It follows that understanding the new terms requires not only their definitions, but also the relationships \textit{of those definitions to established terms}. An \ncn{} will include, therefore: the set of new terms, the definitions of these new terms, and the relationships of these definitions to the terms in the underlying natural language.
	\item From all above, it follows that a team invested in innovation has to, and will create new terms, and thereby, even if implicity, an \ncn. It will be one of many artefacts created during innovation. Since innovation involves uncertainty, and there is thus a possibility for it to be unsuccessful, we get to the problem which motivates this book, and in general motivates paying attention to innovation languages: \textit{How can we create, and analyze to improve, the \ncn{} so that it helps the innovation process?}
\end{enumerate}

\section{Are Ideas Accessible, And Does That Matter?}
\label{c1-s6}
New ideas, like all thoughts, are by definition abstract - something "[existing] in thought or as an idea but not having a physical or concrete existence" \cite{def-abstract}.

Ideas are from the mind and in the mind. You do not ''have'' them because you take them from somewhere else, but because you make them. Despite what Plato was arguing, I never saw an idea outside the mind (maybe I did, just could not recognize it), only descriptions, representations, i.e., only some imperfect proxies of what someone was thinking, and I see them only because that person decided to do something in relation to the ideas she was having.

The point is that ideas are not representations of something, they are whatever they are, but not a reflection of something that exists independently of the individual who has them. They are not, at least in this book, Plato's ''eternally existing pattern[s] of which individual things in any class are imperfect copies'' \cite{def-idea,Kraut2017}. 

Importantly, if ideas come from one mind, they are ''in'' that single mind, whatever that ''in'' may mean exactly in the language of physics, chemistry, neurology, and so on. They are best known only to that mind. This is simply to underline that what I know, especially if it is something new I just came up with, is certainly better known to me, than it is to you. I mean ''known to me'' in the colloquial sense. There is a lot to say about the epistemology of these new ideas, but it is clear that they are not knowledge, i.e., not justified true belief.

For example, if I'm imagining now, how creatures on a distant planet may look like, even if we agree that the planet is HD 156668 b \cite{wikipedia-hd156668b}, you don't really know what I'm thinking. Unless we spoke about it, of course.


\section{Do We Communicate Ideas Or About Ideas?}
\label{c1-s7}
Even if we communicate, words and language are not for copying and pasting thoughts as-is between minds. Daniel Sperber, a linguist, puts this nicely on two separate occasions.

\begin{quote}
''Just as the human mind is not a blank slate on which culture would somehow imprint its content, the communication process is not a xerox machine copying contents from one mind to another. 

[Mechanisms] of human communication [are] much more complex and interesting than is generally assumed, and much less preservative and replicative and more constructive than one might think: understanding involves a lot of construction, and not just reconstruction, and very little by way of simple replication.'' \cite{Sperber2005}
\end{quote}

\begin{quote}
''[Communication] is not a replication system. When I communicate to you, you don’t get in your mind a copy of my meaning. You'll transform it into something else. You extract from it what’s relevant to you. It involves both understanding and misunderstanding. But even if you’re understanding me perfectly, your goal will not be to have a copy of what was in my mind, it will be to extract from it some thoughts of yours which will have been usefully informed by mine, but which will be relevant to you.'' \cite{Sperber2017}
\end{quote}

In short, you can't know my thoughts like I do. I can't know yours. And I cannot copy my ideas to you, nor can you copy yours to me. Thinking otherwise is odd. 

\begin{comment}
Could you, or anyone really claim to know exactly what Bruce Sprinsgsteen meant when he wrote the following part of his "Lost in the flood"?

\begin{quote}
"The ragamuffin gunner is returnin' home like a hungry runaway
He walks through town all alone--"He must be from the fort," he hears the high school girls say
His countryside's burnin' with wolfman fairies dressed in drag for homicide
The hit-and-run plead sanctuary, 'neath a holy stone they hide
They're breakin' beams and crosses with a spastic's reelin' perfection
Nuns run bald through Vatican halls, pregnant, pleadin' immaculate conception
And everybody's wrecked on Main Street from drinking unholy blood
Sticker smiles sweet as Gunner breathes deep, his ankles caked in mud
And I said, 'Hey, gunner man, that's qucksand, that's quicksand, that ain't mud
Have you thrown your senses to the war, or did you lose them in the flood?'" \cite{wikipedia-lost-in-the-flood}
\end{quote}

Your ideas about his ideas are typically called your interpretations of the meaning he intended. Aren't they speculations? You are trying to make sense, using what you think, feel, intend, and so on, at some time (now?) about what someone else was thinking, feeling, intending, a long time ago - sometime before or in 1973 for that song. Best of all, you do this only on the basis of signals that you get from communication with him. In the case of the song, you listen to it, and that's most of us have, unless you have a chance to talk with Bruce. 

This struggle with meaning should be obvious. Don't you know the feeling of trying to get someone to understand something? You often think you failed: they are not doing what you thought they would, and, you think, it must be that you were not clear and convincing.

For that song, the struggle with meaning is not a major issue \footnote{Many heard that song, but only a few go so far as to share their interpretations and claim to be right. For them, the stakes are higher \cite{songmeanings-lost-in-the-flood}.}. There is not much at stake.
\end{comment}


\section{When Does Alignment Matter In Innovation?}
\label{c1-s8}
So far, I noted that ideas are personal and inaccessible. Communication is about ideas, not of ideas. In contrast, actions, including communication, and at least some of their outcomes are accessible to others. It is only through your interpretation of these actions and outcomes, that you judge if others did as if their ideas are aligned with yours.

Now, if I judge that some ideas only matter to me, and that actions that I expect they will lead me to, only affect me, and finally, I see no need to try to convince anyone else in them, then it is good enough that only I know them; you don't need to bother.\footnote{It is more complicated, since I might hold, and not communicate, ideas which might lead me to destructive actions.}

When does that alignment matter? In most cases, or at least any interesting case, something is at stake in our interactions with others.

In innovation, two things are at stake: how the process unfolds, and its outcomes. As soon as your new ideas require you to get help from others, your and their communication and actions start to matter. 

Suppose that I am having ideas about a new kind of bicycle, and I want you to help me make it, or at least up to a prototype. I'm having these ideas, about what the bicycle should look like, run like, behave like, I'm thinking of all sorts of features, properties, characteristics, qualities I want it to have, a booming set of intertwined thoughts. But I also do not know all of it; there's much I haven't considered, and I haven't made up my mind about each one I did. I'm still wondering about the steering, braking, colors, and so on. That is where you come in. I ask you for help. What's next for us? What is the immediate next thing we should do?

Before we start investing more of our time in the bicycle, probably even before you decide if you want to help, I need you to have an idea about the bicycle. But I cannot just copy and paste my ideas into your mind. I need to convey, to communicate about my ideas to you, to help shape your ideas about it, in ways which align them with mine. To be more precise, which will align your actions with those I think I need you to have, so that I will believe that you are helping me. In short, and abstracting all those complexities away, we need to start aligning. We need to get on the same page. 

Suppose we did get on the same page, hopes are up, we are working through the design of the bicycle. It turns out we need more people. They need to be on the same page too, at least about the things we decided by then, so that they can help us out.


\section{Why Are New Terms Alone Not Enough To Convey New Ideas?}
\label{c1-s9}
Through collaboration with you, my original ideas will change, yours too, and all of those will get changed as new people come on board, as they make their contribution in that innovation process. Remember, their ideas are theirs, and we cannot simply transfer, as-is, our ideas to them.

While it is certainly desirable to change ideas as we progress through innovation, we also want to make sure these ideas improve upon previous ones, whatever our exact criteria to judge improvement. In other words, we want learning to happen as we go along, and we want that learning to have a direction. This, in turn, requires  coordinated action.

That is where new terms come in. We need names for new ideas and things we make in the innovation process, so that we can communicate about them in order to coordinate actions. We need to refer to them.

There is a lot to say about the reference relation, one that supposedly exists between words, phrases, or statements, and ideas. At this point, all we need to agree on, is that reference is somehow close to the idea of aboutness, and so, that a statement refers to an idea is somehow similar enough to that statement being about the idea. Note, clearly, that I'm simply saying that reference is aboutness, which is not much better than where I started. In other words, perhaps, I'm avoiding to commit on the exact meaning of reference, or what exactly the reference relation may be. It just does not matter much at this point, my argument remains unchanged if you adhere to any of widely discussed notions of reference.\footnote{See, and pay special attention to Section 5 in \cite{Michaelson2019}.}

Therefore, \textit{we cannot only add names of new ideas and things to the natural language and technical languages we already use}. 

We must also \textit{define the new terms, i.e., explain what they refer to}. How can we otherwise explain them to others, to make them understand what these terms refer to? In other words, how can I explain, describe, or do whatever I need to do, to make sure you do not mistake what I am telling you, with something I did not want to say?


\section{Are New Terms and Their Definitions Enough?}
\label{c1-s10}
Having only the new terms and their definitions is not enough. A definition of a new term cannot only mention other new terms - it also mentions those with an established meaning, those that are part of the ordinary and technical languages used in the team. A definition of a new term leans on old and new terms.

Understanding new terms, then, not only requires the new terms and their definitions, but also being clear about the relationships among the new terms, and between the new and established, or old terms. 

An \ncnf{} is, then, made of 
\begin{itemize}
	\item new terms,
	\item their definitions, 
	\item relationships among definitions of new terms, and 
	\item relationships of these definitions to terms in the underlying natural language and any technical language that may be relevant.
\end{itemize}

\section{Why Think More About \ncn s?}
\label{c1-s11}
Many artefacts are made in an innovation process. There will be documents written, technical drawings made, messages exchanged, calls made and recorded, scale models made, and so on. We do this towards consensus on what needs to be done next, be it more design, prototyping, development, manufacturing, or other.

These artefacts act as a record, however partial and imperfect, of the actions we have taken, themselves the outcome of our joint decisions and design interactions, them, in turn, guided by the ideas we had and have, shaped by context, and communication and action of others. 

A substantial amount of thought and effort, both in academic research and perhaps more in business-oriented research and development, has gone into the design, analysis, and testing of the right structures of these artefacts. The question there is What template should an innovation team use, when solving this or that problem that regularly pops up as they do innovation? 

There is a lot of sense in searching for, designing, and testing such templates, be they for processes and tasks, or for keeping records of what was done, and for planning what to do next. 

The basic motive is that, if you learn the rules once, you don't need to learn them again every next project where they are used, for example, when you look at the next blueprint made according to the same rules. From everything that you need to learn to reach consensus, on a given project, at least you know the rules for reading and making some of the artefacts which record design decisions. You might choose to disagree about the content, but at least you know where to look for it, and how to read it. You know the rules that define the structure of the artefact.

When you study architecture, industrial design, civil engineering, software engineering, management, or one of many others, part of the effort goes into learning rules on how to read and produce such artefacts, so that they fit standardized or well-established practices. 

But even if you know the rules for making and reading the artefact, you still need to know how to read its contents. If you give me two architectural blueprints, one for a family home, another for a hospital, my interpretation of these will depend on the extent to which I was exposed to ideas about the former, and about the latter. It will, evidently, also depend on the knowledge which applies to both, say, something I might have studied as a student of architecture. But each project is different. 

The problem I am interested in, and which motivates looking more closely at \ncn s, happens before we make any artefacts, and persists while we do the design of the bicycle I mentioned earlier, and lives as long as we remain interested in improving the design. 

The problem happens, for the first time, as soon as I need your help, and need to explain to you what it is that I'm thinking to design with you. 

If you think about this in terms of artefacts made during design, blueprints, specifications, documentation, and so on, the problem is not to select or make the right structure for them, but what to fill them with - the content, substance that these artefacts need to contain. 

This problem is particularly interesting when doing design, or innovation more generally, because the ideas will be new to you.\footnote{Design and innovation are different kinds of processes, but I have never participated in design which did not involve some degree of invention, and which, if these outcomes proved useful, was not called innovation. For the record, Oxford English Dictionaries define the three as follows. To design is to ''[decide] upon the look and functioning of (a building, garment, or other object), by making a detailed drawing of it'' \cite{def-design}. To innovate is to ''[make] changes in something established, especially by introducing new methods, ideas, or products'' \cite{def-innovate}. To invent is to ''[create] or design (something that has not existed before); be the originator of'' \cite{def-invent}. It looks safe to say that design and invention go hand in hand, and if their outcomes are useful, they are in retrospect called innovation.} They would not be my references or allusions to something you and I have learned in the past, in roughly the same way, e.g., arithmetic, geometry, especially if it is precisely defined and the definition is widely shared. 

Instead, these ideas will be something that you need to learn from me about. At the same time, since new ideas will not be definite, polished, and complete, you may have yours, which come to change mine, as we collaborate during design, inventing and innovating together.

The problem can be rephrased as the following question, which is the central motive for being interested in innovation languages in the first place:

In innovation, when ideas are new, incomplete, unfinished, brittle, how can I be precise, accurate, and clear about them, so that I can convey them to you, that we can reach consensus on what innovation is about, and that you can then help me replace these ideas with better ones? 

Phrased otherwise, How can I be precise, accurate, and clear about unstable content of innovation? And if I were, Would this helps us reach consensus faster?

This is not the classical problem of what meaning is and is not, in general or of specific notions. It is not about how to convey any kind of ideas, in any situation. It is about new ideas, and they are different from stable ones. In many situations, for stable ideas, it is simply not place or time to debate them much, since they proved to work in the past. If they worked for you, and you have no effort to invest in trying to improve them, then they will not change. This is not to say that they will remain stable forever, but their stability is testament to their past relevance. 

For most children, when they start to learn natural numbers, it is not the point to debate if Peano axioms are good enough a definition; they can do plenty without bothering with it (or any simpler definition they were given). But innovation is a setting and a process, in which ideas need to be short-lived. They are by definition changeable, since the earliest ones are unlikely to be the best; not because they were perfect, but killed by premature criticism, but because they have come up at a time when the person who had them, simply knew less about the problems, situations, and people these ideas should relate to. 

This is also \textit{not} a question of how you arrange the innovation process. Whichever activities it has, however these may be arranged over time, they operate on the content, the substance of innovation, those ideas that each of us has, and which we need to confront, rethink, replace, until we have converged to a consensus that we should take the next step, of taking action which is no longer about the change of ideas, but about the bending of the world to these ideas, be it only by stopping the design of the bicycle I mentioned above, and arranging its manufacturing. 

To summarize, new ideas are necessary for invention, and therefore, for innovation. But new ideas are in individual minds, and cannot, as-is, be carried between them. So we cannot work together unless we communicate these ideas. What's more, we cannot communicate these ideas; we can communicate about them. If people need to collaborate in invention and innovation, their communication will lead each of them to keep or replace her own ideas. 

How can we do this faster while remaining relevant, this process of aligning, diverging, replacing, then aligning again, diverging again, replacing again, and so on? 

One part of the answer is about methods, processes, tasks, that is, how we do invention and innovation; another is how you structure the outputs processes, which is about the structure of artefacts you make. The third part, intertwined with the two others, is the content that the process manipulates, and that the artefacts structure and document, and what needs to be done about that content, for the sake of speed and relevance. 

And as soon as you think about that content, and because of the fact that it needs to be communicated, then you also need to look at the terms which are used to express, convey, represent that content. I look in this book at content, and it is only because of arguing how that content needs to be, that I touch on the process and artefacts. 

An unavoidable present problem is that content of innovation is not tangible. It is not text, it is not drawings, it is not specifications, and it is not the innovation process; it is precisely the ideas of participants, each hidden away and inaccessible to others -- hey are the content, and the only way to do something about them is through material proxies, innovation languages being one kind of those.


% Chapter bibliography
\printbibliography

%%%%%%%%
\chapter{Why Analyse and Design \ncnf s?}
\label{c2}

Even if an innovation language is made in any innovation process, as argued in Chapter \ref{c:new-terms-for-new-ideas}, should we invest effort in analyzing and designing it? To do analysis and design, we need to document the new terms throughout the innovation process, along with their definitions and relationships of new terms to old terms, that is, those terms mentioned in the new definitions, but whose own definitions are not made or changed in the innovation process. In this Chapter, I argue that there are benefits to doing both analysis and design of an innovation language, during the underlying innovation process; both analysis and design are done with the aim to increase clarity, accuracy, and precision of the new terms, and expected benefits for the innovation team are less misunderstanding and faster identification of agreement and disagreement on new terms, and thus, on the content of the innovation process.

\section{What Should an \ncnf{} Do?}
\label{c2:s1}
I argued in Chapter \ref{c:new-terms-for-new-ideas}, that when you do innovation, you also have to extend the base language you use; by ''base language'', I meant what we call a natural or general-purpose language (e.g., English, French, Swahili, etc.) together with applicable technical languages.\footnote{I use the term ''technical language'' to refer to terms and definitions considered as part of an established body of knowledge in a specific domain. A synonym would be ''domain language'';. I consider controlled languages as one of many kinds of technical languages \cite{Kittredge2012}: ''A controlled language (CL) is a restricted version of a natural language which has been engineered to meet a special purpose, most often that of writing technical documentation for non-native speakers of the document language. A typical CL uses a well-defined subset of a language's grammar and lexicon, but adds the terminology needed in a technical domain.'' \cite{Kuhn2014} for a survey of controlled languages.}

In other words, people in an innovation team add new terms and definitions to communicate about new ideas and things they make; in addition, they define the new terms, and thereby connect them to the base language. 

I also argued that innovation teams always do this. It does not matter if they are aware in any way of the concept of new or old terms. They do it because they need it. They need a new language to communicate about the new ideas they are creating, so that they can work together.

Wanting to understand innovation begs the question of how to analyze new terms, in addition to all we already study about innovation. The need to improve how innovation is done begs a different kind of question, one of how to design new terms.

Both analysis and design are biased towards a specific purpose they should serve; the one I argued for in Chapter \ref{c:new-terms-for-new-ideas} had speed and relevance at its core: How can I be precise, accurate, and clear about unstable content of innovation? And if I were, Would this helps us reach consensus faster about the content made, changed, and otherwise manipulated during innovation?

A good \ncnf, roughly speaking, would be one which is designed to be precise, accurate, and clear, and because it is such, it would help us progress through an innovation process faster, and in a relevant way, that is, towards outcomes that everyone involved judges to be constructive. 

\section{Who Does the Analysis and Designs of an \ncnf?}
\label{c2:s2}
It is important to distinguish analysis from design of an \ncnf, not only because they are different tasks, but because of who can do which. 

Analysis can be done on \ncnf s that an observer records by observing a team at work; there is no need for the team to invest resources to record the new terms, prepare them for analysis, and do analysis. In short, analysis of an \ncnf{} can be done without changing (or with minimal changes to) a team's innovation process. 

In contrast, the design of an innovation language changes the team's innovation process. It requires everyone involved to be aware of what an innovation language is, how it is made, and what their role is in making and maintaining that innovation language. 

It is an easier decision to do analysis than design. If someone wants to observe, and if the team is open to it, then why not analyze the \ncnf? If you agreed with my arguments in Chapter \ref{c:new-terms-for-new-ideas}, then an \ncnf{} is something that is made anyway, and since it is specific to the team, it is a phenomenon worth looking into if you want to understand how they do innovation.

\section{Why Design an \ncnf?}
\label{c2:s3}
Why would a team want to design their \ncnf? Why would they want to document and maintain it? 

In this Chapter, I argue in favor of design, and suggest how to do it. Here is the outline of the argument.
\begin{enumerate}
	\item \textit{Speed matters in innovation}, because the faster the innovation process, the faster we can get to observe the outcomes of innovation, see what worked and what failed, then make changes to get more desirable outcomes.
	\item \textit{If speed matters to the innovation team, then they better have precise, accurate, and clear communication.} Why so? Regardless of the specifics of their innovation process, they need to work together, which in turn means that they need to identify what they agree and disagree on; if that can be sped up, then it should speed up their overall process. Precise, accurate, and clear communication, even of new ideas, should help identify faster what they agree and disagree on.
	\item \textit{But what about relevance of communication?} Isn't that what we are looking for? If so, aren't precision, accuracy, and clarity less important? I argue that one should prefer precision, accuracy, and clarity to relevance in the following sense: if we have some limited amount of time to prepare communication before delivering it, then we should prefer to spend time making it precise, accurate, and clear. This is because relevance is hard to predict, we can only really say if something was relevant after we communicated it; the issue there is that relevance of my communication with you is based on my judgment of how my communication related, after I did it, to the actions you took. But this isn't the case for precision, accuracy, and clarity, since I can judge these against my own standards, not yours. As I defined them in Chapter \ref{c:new-terms-for-new-ideas}, I am talking about precision, accuracy, and clarity of communication relative to my ideas, not yours, whereas relevance is judged relative to your actions.
	\item What does precise, accurate, and clear communication have to do with the design of an \ncnf, and especially with documenting it? An \textit{explicit, documented \ncnf{}} helps make communication precise, accurate, and clear. There are three reasons for this. One, to document a term and its definition, you have to think through both of them, thus making you clean up at the very least the misunderstandings you have about them. Two, a documented definition is easier to access than one in someone's mind; I do not need to ask you for it, you do not need to make yourself available to answer; instead, I can look for it wherever it is documented. Three, as your definition is documented and accessible to everyone involved, this puts more pressure on me, to say if I disagree with it.
	\item \textit{However, it takes effort to make an \ncnf{} explicit.} More effort means more time, yet the innovation process needs to take less time. This is called the \textit{innovation language paradox}.
	\item Defining new terms and phrases takes the most effort in making and maintaining an innovation language. I argue that this effort makes sense only if it consists of making a special kind of definitions, called \textit{\newdef s}.
\end{enumerate}

\section{Why Speed Matters in Innovation?}
\label{c2:s4}
There are two reasons to want to do innovation faster. Innovation takes time, and time has a cost. Trivially, if we can do the same or better in less time, then why not? But the second reason is more interesting and important. 

Innovation is not simply the production of new ideas and things. They also have to be useful, or more precisely, prove useful after they have been produced through innovation. Of all new ideas you had today about the future, perhaps only some of them are about something that will be useful in that future. Any innovation process must rely on predictions of what will be useful in the future. 

Future is unknown and predictions of it can be wrong.\footnote{Although it is obvious to say that the future is unknown, this is so only superficially; an eloquent recent and elaborate treatment is N. N. Taleb's triptych \cite{taleb2005fooled1,taleb2007black1,taleb2012antifragile1}. Moreover, it is not the same to try to predict the near or distant future, regardless of what we decide is near and far -- minutes and hours, or days and years).}

What does prediction have to do with speed? If your innovation process takes less time, then it takes less time for you to see if the new is also useful. So you can work with predictions of the near future, rather than distant future; you wait less to see if and how innovation is useful.

It should be clear why speed matters: you want to go from the idea, to seeing it executed quickly, so that you can see if it worked; and if you do this faster, then you invested less to get feedback, and change ideas, and eventually change that which you have been designing. You want to have short iterations from idea to its market.\footnote{See the following small sample of a large body of work on the need for speed in innovation \cite{teece1992,teece1997,von2005democratizing,christensen2013innovator,ries2011leanstartup,christensen2016}.}

\section{How Do Precision, Accuracy, and Clarity Relate to Speed?}
\label{c2:s5}
Is it speed at any cost, and in any way? Speed in innovation can be destructive. The fastest way to go through many new ideas is, perhaps, to reject each one\cite{dougherty2008interpretive,dyer2009innovators,martins2003building}. That makes no sense.

What would be the ideal way to increase speed? There would be no innovation process at all. You would have the right new idea, the one that produces the most useful change; thus, you would have the right prediction. You would need to somehow perfectly know the future. That also makes no sense. 

How to approximate that ideal? 

We can cut uncertainty up, and say that some things about the future can be known today and we are confident will remain so in the future (laws of physics being a trivial example), others could be known (but we cannot spare the cost to learn them, and/or we do not know how), and some unknown number remains ignored, i.e., Donald Rumsfeld's ''unknown unknowns'' \cite{wikipedia-known-knowns}. The ignored ones are by definition not knowable today; we might only find out ex post that we in fact ignored them. We were not even aware of questions which we do not know the answers to. Those we could know, are answers to questions we are aware of now.

Leaving aside what we later discover that we ignored, the apparent way to look to increase what can be known, \textit{is to get input from others}. Since your new ideas are predictions about an uncertain future they share, they may be of help as long as what they provide is what you did not know already.

This only provides additional support to the argument made earlier, that \textit{communication about new ideas is critical in innovation}.

But how is this related to speed? It should be, since if outcomes of innovation require agreement inside the team, it should matter to reach that agreement faster. Even if agreement does not matter, joint work and delegation means that everyone involved has similar ideas about what is new, why it may be useful, and so on.

Therefore, some kinds of communication should be better than others. Specifically, if communication should help speed up innovation, then it should not be misleading or cause misunderstanding. Either of these cases create a need for additional communication, to better explain and align on who meant what. That takes more time, not less.

It looks better, then, if communication is \textit{precise, accurate, and clear}.


\section{How Is Relevance Related to Speed, Precision, Accuracy, Clarity?}
\label{c2:s6}
Is precise, accurate, and clear communication always also relevant?

There is a long history of research on relevance of information \cite{sperber1986relevance,sperber1995relevance,mizzaro1997relevance}. A persistent idea is that relevant and irrelevant information differ as follows: the former influences action in some constructive way, and the latter does not. ''Constructive'' here means that, as you try to solve a problem or, perhaps equivalently, achieve a goal, the relevant information pushes you towards actions which move you closer to the solution or goal. By ''pushes'' I mean that it may be necessary, but not sufficient in your decision-making. In a setup where you collaborate with others to solve problems and achieve goals, as you usually would do in innovation, constructive ends up meaning that you are progressing towards consensus on the prediction, that the outcome of your effort will be useful in the future.

Nothing guarantees that precise, accurate, and clear communication is also always relevant. Even if you and I needed to work together, we could be leading two monologues, both precise, accurate, and clear, but mine about ideas which you see in no way related to your decision and action, and vice versa.

Let's take a simple example. If I should spend some time outdoors tomorrow at my present location, let's say its Frankfurt, then it is more relevant for me to know the weather forecast for Frankfurt, rather than the forecast at some remote location like Vancouver. You might be giving me a precise, accurate, and clear weather forecast for Vancouver; and I might be recognizing its precision, accuracy, and clarity, but if it makes no difference to me, if it has no effect on the options, criteria, and eventually my choice on how to act, then it is irrelevant to me at that time.

How can I then make sure that what I'm communicating, even if precise, accurate, and clear, is also relevant to you? 

For that, I need to make assumptions about what may be relevant to you, which means that I have to make assumptions about problems you are trying to solve, what you know and ignore about these problems, what matters to you, what preferences you have, and alternative solutions you may know and ignore, and much more. I need to think about where you are, when, with whom, how you perceive where you are, when, and with whom, why and how that matters to you, why something matters and other does not, and so on. I might also want to think, not only about what you may be observing, thinking, remembering, but also what you may have observed, thought, remembered in the past, as well as what you may expect of the future, and how you predict it to be. 

Ideally, I would work on facts, not assumptions. That is, I would be able to know in some perfect way your state of mind. That being elusive, the relevance of my communication critically depends on my predictions being right about your state of mind. 

This gets us to an important step in this argument. How much time do you spend speculating what might be relevant to me, before you communicate precisely, accurately, and clearly? How do you split the effort you are able to invest in making something relevant, and to make it precise, accurate, and clear? Is it 50/50? Do you put more into relevance? 

You can decide to communicate later, so that you have more time to speculate about relevance. Or, you can decide to communicate sooner, and spend less time worrying if and how it is relevant to me. 

This is where speed of innovation comes in, and connects precision, accuracy, and clarity to relevance: if I want to spend less time making assumptions about what is or is not relevant to you, then I better communicate precisely, accurately, and clearly, wait for you to decide and act, and then draw my own conclusions about if and how my communication was relevant to you. Rather than spend more on speculating about relevance, my argument is that precision, accuracy, and clarity should be preferred - they let us converge through interactions to agreement or disagreement.

In other words, if you want speed, then you should worry less about the relevance of your communication to me before you actually communicate, and worry more about precision, accuracy, and clarity. If you spend more time thinking about relevance, and therefore communicate later, you will find, in my actions and their outcomes, either evidence for relevance or irrelevance of what you communicated. If you find irrelevance, then you wasted the time spent focusing purely on relevance.

In short, relevance of what is being communicated matters to speed in innovation, but less than precision, accuracy, and clarity. 

This only makes sense in a corrective, rather than preventive approach: \textit{provide precise, accurate, and clear input, then correct if there is evidence for irrelevance}; instead of spending much time trying to anticipate what will or not be relevant, before communicating.

The corrective and preventive approaches distribute responsibility in different ways. The corrective approach asks you to be precise, accurate, and clear about your ideas in innovation, and it is the responsibility of others to do the same with theirs; because of that, you are also responsible to voice disagreement and irrelevance with others' ideas, rather than expect them to be responsible for making their communication relevant for you.

Let's continue the weather forecast example. Suppose we are talking over the phone, and you do not know that my current location is place Frankfurt. The corrective approach is for you to give me the forecast of Vancouver, even without checking if that is my location; the corrective part is that I should stop you and ask about the forecast for Frankfurt. The preventive approach would be, for example, that you spend time trying to identify my location, then choose which forecast you give me. 

This is not a minor point in the overall approach. \textit{It is not that relevance is, in some sense, irrelevant.} If it were, this constructive approach easily turns into a joke, for I would adhere to it also every time I was to throw at you a bunch of very precise, clear, and accurate ideas, but completely disconnected from the situation we are in, and the goals, constraints, values, and so on, which each of us may have and assume for the other. Relevance must be there, and the claim here is not that it should be disregarded altogether. The claim is only that given a minute more to spend, trying to improve communication during an ongoing collaboration we have, is better spent on precision, accuracy, and clarity, since you have more control over that, instead of spending it on relevance.

This aspect of the overall approach, an attention to precision, accuracy, and clarity, and a preference for it over improving relevance that's already there, also rests on a certain set of values. It assumes that \textit{we want speed in innovation}. It also assumes that you and I both have a voice in the innovation process.


\section{Why Should an \ncnf{} Be Explicit and Documented?}
\label{c2:s7}
What does precise, accurate, and clear communication have to do with the design of an innovation language, and especially with documenting that innovation language? An explicit, documented innovation language helps make communication precise, accurate, and clear. 

There are three reasons for this. One, to document a term and its definition, you have to think through both of them, thus making you clean up at the very least the misunderstandings you have about them. Two, a documented definition is easier to access than one in someone's mind; I do not need to ask you for it, you do not need to make yourself available to answer; instead, I can look for it wherever it is documented. Three, as your definition is documented and accessible to everyone involved, this shifts responsibility to them, to voice disagreement with or suggest improvements to your definitions, that is, explanations of your ideas.


\section{How Does This Result in a Design Approach?}
\label{c2:s8}
There is an approach to design \ncnf s, which caters to the need for speed and relevance, and lets you use your preferred innovation process or method.

The approach is simple: as soon as you need to get others involved in the innovation process, to help you move you from your new ideas to something more convincing (for them and for you), then \textit{your team should define the key terms you all are using, in an explicit, clear, accurate, and precise way, and revise these definitions throughout their innovation process}. Doing so, the team is making its innovation language explicit, and documents it for access to everyone involved.

\textit{Explicit} means in a format accessible to others without requiring the attention of its author; for example, written in a document or database. 

\textit{Clear} means there are no multiple interpretations, especially conflicting; it means avoiding ambiguity, vagueness, synonymy. 

\textit{Accurate} means being true to, or exact with regards to the ideas that it should say something about. 

Finally, \textit{precise} means being specific, or close to, or focused on only the ideas to communicate about, rather than other things and ideas. 

The approach, in short, is that the team should make explicit and document their \ncnf, maintain and revise it throughout the innovation process.


% Chapter bibliography
\printbibliography

%%%%%%%%
\chapter{What Is the \ndparadox{} and How to Deal with It?}
\label{c3}

Analysis and design of an \ncnf{} revolves around continuously improving the clarity, precision, and accuracy of definitions that constitute the network. At the same time, I argued that we should want to speed innovation up. I argue in this chapter that this leads to a paradox, and that the resolution of the paradox makes us progress in the design of an \ncnf.


\section{What is the \ndparadox ?}
\label{c3:s1}
In Chapter \ref{c2}, I outlined an approach to the design of \ncnf s. That approach should look odd, because definitions are typically associated with knowledge, that is, statements which usually need to satisfy many hard conditions. There has to be evidence of truth, which may amount to evidence for fitness between the statement and that which it is about when it is observed; and there need to be reproducible procedures to generate such evidence. Look at dictionaries, encyclopaedia, academic, and technical literature. We make definitions for what we know enough to want to pin it down, and for most practical purposes, to keep it that way until there are strong reasons for change.

You are served many definitions during formal education, when you get them precisely because they are, and have been considered as useful by a community that worked on the underlying knowledge for some time. Those definitions graduated scrutiny, or better, they survived it. By learning them, you prepare yourself for situations when you might need to coordinate, to work with others who learned them too, and this should help you align on what needs to be done. Definitions, in short, are one of the few enduring ways with which we have been trying to move knowledge around.

Where is the paradox?

The approach from Chapter \ref{c2} suggests producing definitions for terms that play a role in new ideas. At the same time, we want to be precise, accurate, and clear in those definitions. And we want to progress through these ideas, essentially replacing them with more convincing ones. Finally, we want to do this quickly. 

In other words, the approach says that we should make definitions of ideas we want to quickly replace. 

Isn't this a paradox? Making a definition takes time, yet we want to be fast, and worse, the definition tries to pin down something that we expect to dispose of, in favor of a better replacement.

\section{Why Is the Paradox Useful?}
\label{c3:s2}
The paradox is the engine of the approach for analysis and design of \ncnf s. It is the temporary resolution of the paradox that moves us forward through cleaning up the irrelevant content of innovation. 

The resolution is this: you want to define, in order to be precise, clear, and accurate in communication with others; in turn, this lets them be accurate, clear, and precise in what they disagree with, in what they want to change in what you provided; this puts them in your position, of putting forward their counterproposal, which itself needs to be accurate, clear, and precise, for you to improve on. 

This focus on definition helps reduce waste in communication, which is due to, roughly speaking, missing the point, digressing, and diverging in accidental directions. Moreover, it slows you down when you are preparing your communication about your ideas to others. It forces you to be accurate, clear, precise on your own first, before you ask for others' attention and time. Definitions expose your choices and preferences. And if you cannot be precise, accurate, and clear, then you simply do not know that which you are trying to define - this is far from a failure; it is, instead, a signal that you need others' help. If done well, the definitions you make set the standard for others, which eventually benefits to you.

It is convenient to think about the content of innovation as of an interconnected set of definitions. The notion of "a definition" is a common one. It is hard to meet someone who has never had to learn one. 

Even if these definitions we want to make are accurate, clear, and precise, they cannot be stable - after all, they are created during innovation, a process through which we learn about the problems and opportunities as we move towards a solution. 

Definitions that we make in this approach must be open to change, as few will be stable throughout the innovation process. Many will be thrown away. Some will make it, in a changed form; they will lose some pieces, get new ones, keeping a few all along. 

This is why we need a different kind of definition than usual ones.


\section{How to Deal with the Paradox?}
\label{c3:s3}
The paradox is addressed by making the \ncnf{} from a special kind of definitions. 

A \newdef{} is made to be shaped and reshaped easily, when and how we need it. Just like plastic objects, we want them to be easy and cheap to make, and to stay around for a long time if they have a use.

The \ndparadox{} is about speed through precision, accuracy, and clarity, which slow down. Living with the paradox means that you need to define your ideas in such a way, that it invites change and highlights where these changes should likely occur. As we will see in more depth in the rest of the book, a \newdef{} advertises its targets for improvement, showcases its limitations, and begs destruction. You make it in such a way that it is easier for others to find targets for improvement, refinement, or replacement. This makes it harder for you, not only because it is more work to do, but also because it requires you to make your choices transparent - and many of those will be refined and replaced, because others have something more convincing to offer.

What about relevance? How does a \newdef{} relate to relevance of the content in an innovation process?

This is about how a definition begs action. You want to have definitions which incite team members to challenge them, which is why precision, accuracy, and clarity matter. I'll use blunt examples to illustrate this.

Let's say you and I need to design a new airplane. I'm looking for a design that is very different from anything already out there. At some point, I'm telling you that it will have such different wings, that anyone will know this is our design, as soon as they see them. But since there are innumerable wing designs (most not feasible, since I have no expertise relevant for this task), I might give you the following definition of these wings I will be calling x-wings:

\newterm{
(Definition 1) 
X-wings: Oddly shaped wings for the new airplane design.
}

This is pretty much a useless definition, as far as the approach here is concerned. To see why, consider the following one instead.

\newterm{
(Definition 2) 
X-wings: Four wings, two on each side of the airplane, the first pair placed 1/5th (or thereabout) of airplane's length from its tip, and the second pair placed 1/6th (open to change) of airplane's length from its tail, each with a width equal to 70\% of the airplane's length. The first pair is inclined at a 2 (might be more) degree angle up from the plane's horizontal axis, and the second is at a 2 degrees (more, but not above 4) angle towards the ground.
}

Why is the second definition any better? Because it is clearer, more precise, and - as these are the wings I was thinking of - it is more accurate to what I had in mind. 

It is important to know that I can, of course, be completely wrong about this wing design. It may not be feasible at all, or if it is, it may be impractical for many reasons. Can such an airplane be docked on a standard airport? How fast can it fly? How far? And so on. 

This is an important point: I could be proposing the wrong design, a crazy useless one, but the only way for you to tell me that is if I am clear, precise, and accurate about it. The reason for this should be clear: If I were to give you Definition 1, yet I had in mind what I described in Definition 2, then you would need to invest effort in discovering what I had in mind. I would expect you to ask me enough questions, that you can go from Definition 1 to Definition 2. This is hard to do, precisely because my thoughts are only mine, and yours are only yours, as I recalled in Chapter 1. 

There are two other major problems with me giving you Definition 1 when I could have given Definition 2. One, there is nothing that guarantees you will ask the questions you need to ask, to go from the former to the latter Definition. Two, isn't asking you to do it a waste of everyone's time? I have to wait for you to guess and ask the right questions, and answer your questions, while you struggle to find your way around something that may be so out of the ordinary (that wing design certainly is), that simply producing that definition proves a major challenge, only so that after all that is done, we can agree that design makes no sense at all. 

Five actions apply to \newdef s: keep, refine, add, remove, and choose. Keep or refine what you agree with in a definition, add something that's missing, remove what you disagree with, and chose on my behalf for choices that I left open in the definition. 

To have something to agree or disagree with, the definition needs to expose your choices and options. It needs to make explicit, clear, precise, accurate (with regards to your ideas) statements, and the clearer, more precise, or accurate they are, I argue, the easier it is for you to take either of these actions.

This approach is hard work, and it is not a substitute for everything  that needs to be done in an innovation process anyway. There still needs to be research, brainstorming, prototyping, testing, and so on. But the aim with adding an Innovation  language is to have everyone be precise, clear, and accurate in communication, to avoid, or more realistically reduce misunderstandings, to ask everyone involved to be as explicit as possible, even if mistaken, about the ideas they are trying to convey, and to be transparent, when these ideas are yours, on what these ideas are, in ways which make it easier for others to agree, disagree, and build on what you are suggesting.

To close this Chapter, remember two key ideas. One, the approach I am suggesting is for the team to make, improve, and maintain an \ncnf. Two, the engine of this approach is the unstable resolution of the innovation language paradox: the pieces of an innovation language are precise, clear, and accurate, but malleable and disposable definitions; we invest in making them, not because we want to keep them, but to make it easier for everyone to replace, add to, or remove from them. The paradox is that we want precision, clarity, and accuracy from something we know is going to be thrown away or changed, and we see progress through that change, rather than stability. It's resolution is unstable because we go from one version of a definition to the next, and each time, if we are precise, clear, and accurate, it is easier to spot flaws, deficiencies, and incompleteness. 


% Chapter bibliography
\printbibliography



%%%%%%%%
\chapter{What are \newdef s, and how different are they?}
\label{c4}

The design of an \ncnf{} involves making and revising  definitions of a particular kind, called \newdef s. In this chapter, I discuss the plasticity of a definition, and relate it to well-known discussions of what makes a good definition, from Kant, Belnap, and Kripke.}


\section{What Is a \newdef?}
\label{c4:s1}
A \newdef{} is a precise, clear, and accurate definition which you make for someone else to change, by inviting five kinds of actions: 
\begin{itemize}
	\item to \text{keep} what they agree with, in that definition,
	\item \textit{refine} what they can make more precise, accurate, and clear,
	\item \textit{add} what they are convinced in missing,
	\item \textit{remove} what they disagree with, and
	\item \textit{choose} among options which you were indecisive on.
\end{itemize}

Expectation of change is an essential property of \newdef s. This is what ''plastic'' in \newdef{} is meant to convey: these definitions are open to debate and completion, revision, or other kinds of changes we might want to make. Their plasticity is intended to underline that they are not meant to remain the same for long, that they are not immutable, but to the contrary, that they invite change.

The relationship between definitions in general, and \newdef s is not an easy one. Just from the name, you might think a \newdef{} is a kind of definition, that of all definitions, some are plastic, others not. But this depends on what definitions are, in general, that is, how you define a definition, a big question on its own.

I do want to spend time in this Chapter, however, to flush out the relationship between \newdef s and key ideas about definitions in general. This should help clear out some of the many potential misunderstandings, and pre-empt some of the many expectations you may have from this book, and which I did not want to realize at all.

\section{Kant's Theory of Definition}
\label{c4:s2}
For Kant, to define is to identify all primitive properties of that which you are trying to define, whereby that set of properties allows you, me, others, to unambiguously distinguish the thing from others. It is important that all these properties in the set, i.e., properties which together make up the definition, are primitive. 

Primitive means that there is no one among them, which you can conclude from knowing only the others. Each one is necessary, none can be removed, and together they are sufficient for you to make the distinction, none is missing.\cite{kant1999critique,kant2004lectures,beck1956kant} 

Still for Kant, a definition can be analytic or synthetic. It is analytic if it is defining something which is a given. That which is being defined is already there, independently from its definition, and its definition identifies only the necessary and sufficient primitive properties needed to distinguish that given thing from everything else. 

A definition is synthetic, if it creates the thing it defines. The thing that is being defined is not there, and starts being there by virtue of us producing its definition. The definition contains the necessary and sufficient primitive properties for us to introduce the thing and make it unambiguously distinct from all else.

Then, there is a distinction between nominal and real definitions. 

A real definition will be such that, at least some properties it identifies are grounded in experience of reality. 

A nominal definition will identify properties which do not have to be grounded in such experience, for you to accept that the thing being defined exists. 

Peano axioms, for example, seem to be a nominal definition of natural numbers if you take them to exist as abstractions in the framework of mathematics. They do not define what a natural number refers to, but define the relationships which we must stand between instances of natural number - it does not matter if you or I have ever had experiences of actual things which we would reasonably call natural numbers. In more technical terms,

\begin{quote}
''Kant is saying that in a real definition we do not merely equate a word with a logical product of arbitrarily chosen logical predicates, but we make at least a problematical existential judgment and state the conditions under which this judgment could be verified so that the definiendum [i.e., name of that which is being defined] will be seen to have 'objective reference.' There must be, in the definiens [the statement of properties, defining the definiendum], some determination or compound of determinations that can be 'cashed' in possible sensible (intuitive) experience.'' \cite{beck1956kant}
\end{quote}

Analytic/synthetic and nominal/real are orthogonal, independent distinctions, combining into four kinds of definitions: analytic nominal, analytic real, synthetic nominal, and synthetic real.

How does this relate to \newdef s? 

I said earlier that a \newdef{} is open to change, and it needs to be precise, clear, and accurate, because that makes it easier to change. 

The analytic/synthetic distinction has nothing to do with either of these - there is no reason why an analytic or a synthetic definition cannot change. If you do take that distinction seriously, then note that the plasticity of an analytic definition reflects the acquisition of more information about the properties of the given thing being defined; the plasticity of a synthetic definition means going from defining and making one thing, and when the definition changes, the thing being made changes too, as when, for example, one software specification, a synthetic definition of the software product to make, leads to one release of that product, and then a change in the specification leads to a next release, of a different software product. 

Plasticity is also independent from the nominal/real distinction. Nominal definitions can change, if the abstraction they define changes - this could be because you know more than you did when you defined it initially, you thought it through more, you discussed it with others, and so on. These changes will amount to adding or removing of, the properties appearing in the definition. Same applies to real definitions, in which case that change reflects again the change in what we know or assume about the actual thing being defined.

If a \newdef stops changing, if it graduated to a stable definition, then it may be analytic or synthetic, and nominal or real. 

That independence from Kant's dimensions of definitions is a first useful observation so far in this Chapter. It consequently does not matter if you do, or don't take Kant's distinctions relevant. 

A second observation has to do with the economy of definition, specifically that a good definition only identifies primitive properties. If a property can somehow be derived from primitive properties, then it shouldn't be mentioned in the definition. 

But how do you derive these other properties? What does it mean, in practice, to derive properties? 

These questions matter, because what you answer decides if you care at all about the distinction between primitive and derived properties. If you don't, it is one less complication to think of, when making and changing \newdef s.

A property is derived, if you can take the definition, and by applying some procedure, find that derived property of the thing being defined. So you do not need to have those derived properties written in the definition, as long as you know the primitive ones, and the procedure.

We need to distinguish two cases: Is the procedure well defined, or not? 

Mathematical logic has well-defined procedures. Primitive properties need to be stated by grounded well-formed formulas of your logic of choice, and the logical closure over these formulas will include all derived properties. Same goes for statistics. If you have a dataset that describes, thus defines some sample of a population, a mean of a variable in there is a derived property, to be derived by computing the trivial, but well defined formula. 

What if the reasoning procedure is not well defined? This is the realm of personal, subjective, biased, ordinary, everyday reasoning procedures, those which might be partly accounted for by Walton's informal logic \cite{walton1989informal},  behavioral economics \cite{mullainathan2000behavioral}, starting with Tversky and Kahneman \cite{kahneman2013prospect}, or other, but certainly not well defined procedures, procedures which you and I, if we applied them, would consistently lead us to the same conclusions. In those cases, if there is no guarantee that you and I will derive the same conclusions, from the same definition, is the primitive and derived distinction useful?

It remains useful, but in a restricted sense: there is no need for synonyms in a definition, for example.

For \newdef s in particular, the primitive/derived distinction loses its appeal. \newdef s are used when working with new ideas, when trying to make them precise, clear, and accurate, so as to prepare them for others to change. It is more important to do that, than to worry about a \newdef stating the obvious, which is the worry that the economy of derivation promotes. Better for the content of a \newdef{} to be obvious to all, than missed by some. 

Does this argument apply even when the reasoning procedure is well defined? I cannot compute the mean of a categorical variable, we all know that. I also cannot derive a formula with a predicate which did not occur somewhere in the premises.\footnote{Unless I am drawing conclusions from an inconsistent set of formulas, and doing so under a procedure which is explosive (as in classical propositional or classical first-order logic, for example). But in that case, I can only conclude anything, same as you, and we may disagree, then. In other words, since you and I can conclude anything from inconsistency, how likely is it that we will come to the same conclusions? So my point holds for paraconsistent logics too.} 

A well defined reasoning procedure ensures that both you and I, and anyone else, starting from the same explicit, written down premises or inputs, will compute the same conclusions. We will compute the same mean of the same variable, if we both have the same dataset. We will compute the same grounded formulas from the same premises, if we apply the same reasoning procedure. But while we will get the same formulas, or the same numbers, what we read from this, the references we make, can be different, and there is nothing the well defined reasoning procedure can do about that. So even if the procedure is well defined, it is you and I who read the conclusions, and when we do it, we end up applying our individual procedures which are not well defined, and are inevitably different.

Ultimately, then, there's no need to worry if your \newdef s only talk of primitive properties, or if they fail the economy of derivation. In practice, this translates into \newdef s which repeat the same things, in different ways, trying to clarify the important ideas they are being designed and changed to convey.

This closes the discussion of Kant's theory of definition and its relationship to \newdef s. The short summary is that we can leave Kant's remarks aside when making and changing \newdef s in innovation languages.


\section{Belnap's Rigorous Definitions}
\label{c4:s3}
Belnap is less concerned with categories of definitions, than with the ''good'' properties of definitions. For him, like for us here, a definition tries to explain the meaning of a word or phrase.

\begin{quote}
''I consider [definitions] only in the sense of explanations of the meanings of words or other bits of language. (I use 'explanation' as a word from common speech, with no philosophical encumbrances.) As a further limitation I consider definitions only in terms of well-understood forms of rigor. Prominent on the agenda will be the two standard 'criteria' - eliminability and conservativeness - and the standard 'rules'.'' \cite{belnap1993rigorous}
\end{quote}

A definition explains by relating the new word or phrase, that which it defines, with words and phrases you may know already. I can only explain X to you by using something which you already know. I cannot, of course, know what you know, but I can make assumptions, and adapt my definition through our communication. 

The idea that a definition explains by relating to prior knowledge is an important one, and is the basis for various analyses I present in later chapters. 

Belnap wants to be more precise about what qualifies an explanation for a definition:

\begin{quote}
''Under the concept of a definition as explanatory, (1) a definition of a word should explain all the meaning a word has, and (2) it should do only this and nothing more. That the definition should (1) explain all the meaning of a word leads to the criterion of eliminability. That a definition should (2) only explain the meaning of the word leads to the criterion of conservativeness.'' \cite{belnap1993rigorous}
\end{quote}

A definition satisfies the criterion of eliminability, if you can replace the term it defines by the definition, anytime you use that term, and this would change nothing to your and others' understanding of what you are or were trying to say. In other words, if some term T is defined by D, with D being a sentence, a few of them, or anything else that you consider being the definition of T, then T:D satisfies eliminability if you can use D anytime you used T, and it would change nothing at all.

Let's illustrate this with a definition of ''giraffe''. 

\begin{svgraybox}
Giraffe: ''a large African mammal with a very long neck and forelegs, having a coat patterned with brown patches separated by lighter lines. It is the tallest living animal.'' \cite{def-giraffe}
\end{svgraybox}

I need a convention to make writing easier. Above, ''giraffe'' is called definiendum, the term being defined, and the sentences which define it are called definiens.

What eliminability requires, is that anytime you, I or anyone else uses the term ''giraffe'', we can replace that term, in that context in which she or he used it, with the definiens above, and this replacement would change nothing (relative to using the term ''giraffe''). Which is to say that eliminability requires definiendum and definiens to be interchangeable at all times, everywhere, without effect on what is being understood, in any context where either is used. Continuing the illustration, if you define ''giraffe'' as above, then the following sentence, and the subsequent paragraphs need to have the exact same meaning, if the definition above satisfies eliminability.

\begin{quote}
''Okapis are very different in their ecology and behavior from giraffes.''
\end{quote}

\begin{quote}
''Okapis are very different in their ecology and behavior from large African mammals with a very long neck and forelegs, having a coat patterned with brown patches separated by lighter lines, they are the tallest living animals.''
\end{quote}

Even for simple examples, as above, it is a stretch to claim that the definiens and definiendum are such that nothing changes if you change one for the other. If anything, the mood of the statement changes - imagine we talk, and I tell you that okapis and giraffes are different in ecology and behavior; then try to imagine the same conversation, where I'm giving you the definition of giraffe at length. And the definition above is a simplistic one; what if I were instead using the following one:

\begin{svgraybox}
''The giraffe (Giraffa) is a genus of African even-toed ungulate mammals, the tallest living terrestrial animals and the largest ruminants. The genus currently consists of one species, Giraffa camelopardalis, the type species. Seven other species are extinct, prehistoric species known from fossils. Taxonomic classifications of one to eight extant giraffe species have been described, based upon research into the mitochondrial and nuclear DNA, as well as morphological measurements of Giraffa, but the IUCN currently recognises only one species with nine subspecies.'' \cite{wikipedia-giraffe}
\end{svgraybox}

You could argue I am caricaturing. Would anyone do this replacement, except for irony, sarcasm, or comedy? 

But the ''why'' does not matter much, the point is that the very notion of preserving meaning probably makes little sense if you take seriously - as I did in chapter 1 - that you and I cannot have the same ideas in mind, regardless of how same the things we say or write. So there is no perfect preservation, and eliminability can only be something to want, but which you cannot have.\footnote{This is probably fine to say for natural language, but is too pessimistic for formal languages. In a mathematical logic, eliminability makes perfect sense, since we can - provided it is computable - determine if there is preservation of meaning, when meaning means all logical consequences. So you can take eliminability seriously, but only if you put many constraints on the language you use to make definitions. But innovation languages are in the realm of natural language.}

In the context of innovation languages, eliminability remains a useful idea. Remember that we want to be precise, clear, and accurate, in the specific sense I discussed in Chapter 2. So if you replace definiendum with its definiens, and this leads you to easily draw conclusions which make ideas less precise, clear, and accurate, then this needs to be looked into. Again, there is a limit to how useful this eliminability idea is, since it is defined with a counterfactual: should I have replaced it, we would have had something else; but if I do not, then I cannot know that it would. Despite flaws, it remains a sanity check. 

What about conservativeness? Eliminability was about having the definition explain every meaning, or meaning in all contexts, of the defined term. In innovation languages, I suggested settling with eliminability as a reminder to check what might be concluded when definiendum replaces definiens, especially if conclusions go against precision, clarity, and accuracy. Conservativeness is about a definition doing not more than explaining meaning. 

Consider the following example, where I have two definitions of soccer.

\begin{svgraybox}
Soccer: ''Association football, more commonly known as football or soccer, is a team sport played with a spherical ball between two teams of eleven players. […] The game is played on a rectangular field called a pitch with a goal at each end. The object of the game is to score by moving the ball beyond the goal line into the opposing goal.'' \cite{wikipedia-association-football}
\end{svgraybox}

\begin{svgraybox}
Soccer: ''Association football, more commonly known as football or soccer, is a team sport played with a spherical ball between two teams of eleven players. It is played by 250 million players in over 200 countries and dependencies, making it the world's most popular sport. The game is played on a rectangular field called a pitch with a goal at each end. The object of the game is to score by moving the ball beyond the goal line into the opposing goal.'' \cite{wikipedia-association-football}
\end{svgraybox}

If I wanted a definition which remains neutral on how this sport is or is not popular, or widely played, then the first definition is conservative, and the second is not. I can conclude nothing about popularity from the first, but I can from the second. 

It might seem that conservativeness has an obvious and hard problem. Namely, it is relative to the purpose of the definition, or intention of the maker of the definition. Or, it is relative to the meaning intended by the maker of the definition. If I was making a definition of ''soccer'', I might think it necessary to say something about popularity, while someone else would not. So this is not about what soccer is, in the context of, say, a foundational ontology. If it is my intent that sets meaning, then it is me who draws the line between success or failure to satisfy conservativeness. And this is a problem, because if it is up to intentions, it is up to something inaccessible. I cannot, as I emphasized several times by now, see or otherwise access directly your intentions, since they are ideas ''in your mind''. So again, just like for eliminability, conservativeness is easiest to precisely define if we are making definitions in a formal language, such as some mathematical logic. There, a definition is conservative if replacing definiendum with definiens leads to the same conclusions.

There is a way to approach this, which without giving too much attention to either eliminability or conservativeness, and in the context of natural language, as I argue next.


\section{Carey's Origin of Concepts}
\label{c4:s4}
When you define something, where did your idea of that thing come from? And where does your definition of it come from?

You saw it, you sensed it, you thought it, it occurred to you. And so on. Many verbs and phrases, to say variations of the same thing: you learned it, and you wanted to share what you learned with others.

According to Carey \cite{carey2009origin,carey2011precis}, learning itself is always some variation of hypothesis testing.

\begin{quote}
''Ultimately learning requires adjusting expectations, representations, and actions to data. Abstractly, all of these learning mechanisms are variants of hypothesis testing algorithms. The representations most consistent with the available data are strengthened; those hypotheses are accepted.'' \cite{carey2011precis}
\end{quote}

Does that work with new concepts? No, you don't know what to test. You need to have some variables, to have some categories to think about, so you can hypothesise their relationships, and test, that is, see what survives. Innovation involves what Carey calls developmental discontinuities.

\begin{quote}
''However, in cases of developmental discontinuity, the learner does not initially have the representational resources to state the hypotheses that will be tested, to represent the variables that could be associated or could be input to a Bayesian learning algorithm. Quinian bootstrapping is one learning process that can create new representational machinery, new concepts that articulate hypotheses previously unstatable. In Quinian bootstrapping episodes, mental symbols are established that correspond to newly coined or newly learned explicit symbols. These are initially placeholders, getting whatever meaning they have from their interrelations with other explicit symbols. As is true of all word learning, newly learned symbols must necessarily be initially interpreted in terms of concepts already available. But at the onset of a bootstrapping episode, these interpretations are only partial — the learner (child or scientist) does not yet have the capacity to formulate the concepts the symbols will come to express. The bootstrapping process involves modeling the phenomena in the domain, represented in terms of whatever concepts the child or scientist has available, in terms of the set of interrelated symbols in the placeholder structure. Both structures provide constraints, some only implicit and instantiated in the computations defined over the representations. These constraints are respected as much as possible in the course of the modeling activities, which include analogy construction and monitoring, limiting case analyses, thought experiments, and inductive inference.'' \cite{carey2011precis}
\end{quote}

Besides children being somewhat like scientists and the other way around, the idea to keep is that new concepts are anchored in old. Hard to escape what you already know. That is, when speaking of definitions, the definition of something new will lean on concepts you already know how to define. 

The new concepts, those playing a role in the explanation of your inventions in innovation, themselves come from your work on the placeholders. The placeholders, in turn, come from the distinctions, the differences from what you observe or think do or should exist. Simplifying, there's an apple with seeds, and you want it without, so there is a placeholder for a seedless apple, waiting to be thought out more - which other properties, other than the absence of seeds (the difference from the observed) you want? The further you think it through, the more substance in the placeholder, so to speak, and the less it is a placeholder. If something in what you know is not how you think things are, then creating a placeholder is about deciding what exactly should be different, so that it would turn out how you think it should. 

\begin{quote}
''The process of construction involved positing placeholder structures and involved modeling processes which aligned the placeholders with the new phenomena. For Kepler, the hypothesis that the sun was somehow causing the motion of the planets was a placeholder until the analogies with light and magnetism allowed him to formulate /textit{vis motrix}. For Darwin, the source analogies were artificial selection and Mathus' analysis of implications of a population explosion for the earth's capacity to sustain human beings. For Maxwell, a much more elaborate placeholder structure was given by the mathematics of Newtonian forces in a fluid medium. These placeholders were formulated in external symbols -- natural language, mathematical language, and diagrams.'' \cite{carey2011precis}
\end{quote}



\section{Kripke' Naming and Necessity}
\label{c4:s5}
When you try to define something, that thing -- be it concrete or abstract, chairs or thoughts -- is what your definition is about. If there is exactly one, unique such thing, your definition should, ideally, unequivocally identify it. If I were to learn that definition, I would know exactly what it is that you defined. There would be no doubt about it. I would not mistake it for something else.

In ''Naming and Necessity'' \cite{kripke1972naming}, Kripke discusses the relationship between the name, a proper name of the thing or person, and that thing or person it names. Inevitably, this has a lot to do with definitions. The definition should relate the name to the thing or person. How does that happen? How could a definition do this? How should a definition be, to successfully do it?

One thing Kripke does, which is interesting here, is summarize a common view, still, on how a name refers to what it names. You can see this as an account of how a definition defines the thing being named.

\begin{quote}
''...a theory of naming which is given by a number of theses…[as follows:]
\begin{itemize}
\item To every name or designating expression X, there corresponds a cluster of properties, namely the family of those properties P such that A believes P(X) [i.e., it is true to say that X has properties P].
\item One of the properties, or some conjointly, are believed by A to pick out some individual uniquely.
\item If most, or a weighted most, of the properties P are satisfied by one unique object y, then y is the referent of X.
\item If the vote yields no unique object, X does not refer.
\item The statement, 'If X exists, then X has most of the [properties] P' is known a priori by the speaker.
\item The statement, 'If X exists, then X has most of the [properties] P' expresses a necessary truth (in the idiolect of the speaker).
\end{itemize}
For any successful theory, the account must not be circular. The properties which are used in the vote must not themselves involve the notion of reference in such a way that it is ultimately impossible to eliminate.'' \cite{kripke1972naming}
\end{quote}

Reference, according to the above, works by selecting properties which uniquely distinguish the named individual. To define, then, the individual, is to identify such properties at least, or all properties - those included - which describe the individual. It is to provide a description which only fits that individual.

Kripke argues this is not how reference works. The crux of his argument, to my understanding, is that people do refer to individuals, they believe and act as if they referred uniquely, yet in many cases they use names and believe reference without being able to satisfy all conditions listed above. I agree with his conclusion, but for somewhat different reasons than those he focuses on. For one, many properties are hard to specify precisely enough, for them to be relevant in distinguishing an individual (object, person, thought) from others. Being red is a property, but there are many nuances of red, and picking the right red for the red object you are naming and defining, is a complicated matter. Not everyone sees reds in the same way, and not all red objects will consistently be observed as being red in the same way, in all conditions. So there is a practical difficulty in picking properties that distinguish. Another way to put this, is that just as we have trouble defining the thing by its properties, we have substantial difficulties defining those properties themselves. Second, suppose that there exists, as a matter of metaphysics, the possibility to know all properties of an individual (or the smallest such set which distinguishes that individual from all else). It is unlikely that I will accidentally know them all; if not, then I need to discover, or more generally do something to learn them. There is no single directory where I could learn this easily, some given universal ontology. And so, I would need to invest substantial effort to identify what picks out that individual over others. Think about all the effort that is needed to create and update biological taxonomies, medical ontologies, and really any other knowledge of classification, and it becomes clear that it is probably impractical, i.e., not feasible, for any one of us to know, for all individuals we refer to, the complete set of properties that unequivocally single out those individuals from everything else. Yet we do manage to get by in practice, as Kripke observes. So what if we cannot successfully use properties to refer precisely?

Kripke's proposal is, instead, that naming works through convention, and convention gets passed from person to person. Reference is part of culture, if culture is all non-genetic information passed across generations.

\begin{quote}
''Someone, let's say, a baby is born; his parents call him by a certain name. They talk about him to their friends. Other people meet him. Through various sorts of talk the name is spread from link to link as if by a chain. A speaker who is on he far end of this chain, who has heard about, say Richard Feynman, in the market place or elsewhere, my be referring to Richard Feynman even if he can't remember from whom he first heard of Feynman or from whom he ever heard of Feynman. He knows that Feynman is a famous physicist. A certain passage of communication reaching ultimately to the man himself does reach the speaker. He hen is referring to Feynman even though he can't identify him uniquely. He doesn't know what the Feynman theory of pair production and annihilation is. Not only that: he'd have trouble distinguishing between Gell-Mann and Feynman. so he doesn't know these things, but, instead, a chain of communication going back to Feynman himself has been established, by virtue of his membership in a community which passed the name on frok link to link, not by ceremony that he makes in private in his study: By 'Feynman' I shall mean the man sho did such and such and such and such.'' \cite{kripke1972naming}
\end{quote}

This idea of tying reference to historical use, in which conventions pick out the individual, gets us to the following topic, of crowd definitions.


\section{Crowds}
\label{c4:s6}
There are two ideas which cause trouble when thinking about definitions, and what a good definition may look like.

The first toxic idea is that you can produce a definition which explains all meanings of a term, for everyone, anytime, and everywhere. It is the idea that you can make a successful universal definition.

The second toxic idea, strongly related to the first, is that you can restrict, via the definition, the conclusions which anyone can draw from it. If you think that meaning are the results of all reasoning you can do from the definition, then this second idea is really a facet of the first.

Instead, consider if a definition, or its possible consequences matter to anyone, anytime, everywhere? Probably not. When was the last time you needed to reason from a definition of ''giraffe''? And if you did, what was the stake with that reasoning? Did it matter if you drew different conclusions from the nearest zoologist? This obviously does not apply to all terms, all the time, and everywhere. It is, for example, of significant importance, that customs officials in all countries agree on the definition of ''passport'', as disagreements there will matter to many, frequently, and in many places. But even then, not everyone has something at stake, all the time. 

The point is that the quality of a definition matter at any time only to some, and that they may identify themselves, if doing so can be done at some reasonable cost. 

This is where crowds come into play. Wikipedia is an example of a technology which creates term or topic-specific communities (whereby ''topic'' I mean somehow related sets of terms). It lowers the cost for people who care enough about the quality of definitions of ''soccer'', to debate, update, and extend the definition of that term. It gives anyone in the crowd, so to speak, to come forward and add or remove to a definition of a term, whose meaning they want to say something about. What matters here is not that the term is defined for universal agreement, or fitness to empirical evidence, but for acceptance by the community of those who worry about the phrasing, content, and consequences or implications of the term being defined in some or other way.

And this is important for innovation languages, because the team who does innovation are exactly those concerned with the meaning of the terms they use to describe that which they are coming up with.

With this in mind, a definition is good, or its quality matters only for and within a community - it is its community which promotes desirable, and sanctions undesirable consequences, and judges if it satisfies eliminability or conservativeness. That community, like in many areas of scientific research, may agree on highly specific rules and procedures for producing and validating evidence for something to be in a definition.

The extent to which different communities produce definitions of different quality, can be illustrated by looking at the definition of something recent, with strong emotional content. How about Pikachu?

Editors at OED are apparently not the community to ask about this. Sadly, OED has a definition of Pok\'{e}mon, but not of Pikachu.

\begin{quote}
''A series of Japanese video games and related media such as trading cards and television programs, featuring cartoon monsters that are captured by players and trained to battle each other.'' \cite{def-pokemon}
\end{quote}

Editors of the Pikachu page on Wikipedia are a slightly more interested group.

\begin{quote}
''Pikachu are a species of Pok\'{e}mon, fictional creatures that appear in an assortment of video games, animated television shows and movies, trading card games, and comic books licensed by The Pok\'{e}mon Company, a Japanese corporation. They are yellow rodent-like creatures with powerful electrical abilities. In most vocalized appearances, including the anime and certain video games, they are primarily voiced by Ikue Otani.'' \cite{wikipedia-pikachu}
\end{quote}

But it all pales in comparison with the community at Bulbapedia, an encyclopedia dedicated to Pok\'{e}mon. This is merely the overview, of the entry on Pikachu, a text of more than 15,000 words.

\begin{quote}
''Pikachu is an Electric-type Pok\'{e}mon introduced in Generation I.
It evolves from Pichu when leveled up with high friendship and evolves into Raichu when exposed to a Thunder Stone. However, the starter Pikachu in Pok\'{e}mon Yellow will refuse to evolve into Raichu unless it is traded and evolved on another save file.

In Alola, Pikachu will evolve into Alolan Raichu when exposed to a Thunder Stone.
Pikachu is popularly known as the mascot of the Pok\'{e}mon franchise and a major representative of Nintendo's collective mascots.
It is also the game mascot and starter Pok\'{e}mon of Pok\'{e}mon Yellow and Pok\'{e}mon: Let's Go, Pikachu!. It has made numerous appearances on the boxes of spin-off titles. Pikachu is also the starter Pok\'{e}mon in Pok\'{e}mon Rumble Blast and Pok\'{e}mon Rumble World.'' \cite{bulbapedia-pikachu}
\end{quote}

Pikachu is an example of an innovation, or of part of an innovation (if you take Pok\'{e}mon to be the innovation). It is one of many terms used when talking, or in general communicating about Pok\'{e}mon. 

Pikachu is an interesting example, because it is an abstraction which becomes somewhat more concrete in cartoons, video games, and any other place it appears, in drawn or other format. There is no thing, object, in nature, which was named Pikachu, and so, there are some properties it has, and which anyone who has access to that object can ascertain. There are properties its authors, designers, chose to represent. But as the community formed around it, the definition will be a reflection of what they see in it. Is there a unique, universal, objective, or otherwise observer-independent canonical definition of Pikachu? No, and thus we fall back on the interested crowd to tell us what they think about the definition of Pikachu, given the concretizations of Pikachu, the imagination of each member of this crowd, and the persistence of each one of them to promote and defend their own idea of Pikachu.

Now, replace ''Pikachu'' above, with the term ''nuclear weapon'', or ''agricultural subsidy'', and the important point remains, that there is an interested community, which decides the definition. Evidently, the effects of these choices, their gravity, is clearly different for ''Pikachu'' and ''nuclear weapon''.

To see the second important point, consider this question: What do we get by having a community agree on one definition of Pikachu? If differences of opinion on the definition of Pikachu have no bearing on the actions of those who are concerned with that definition, then there is no reason to ever worry about this. In any other case, it is a question of how much their actions matter, and how divergence in opinion, on the definition, affect their actions. So for nuclear weapons, agreement on the definition is critical, since it influences defence policies, and through them, the lives of many, including most directly those in defense industries, in various countries.

The second important point is that agreement on a definition influences coordination, that is, by influencing individual actions of those who agreed, it influences how they act together, and thus the outcomes of their joint actions.

The third important point is that the definition lives and dies with its community. As that community changes - which simply means as it gets new members, loses others, and as they change their minds - so will the definition. For some terms, especially if they refer to things whose properties can be ascertained through repeatable procedures that yield consistent outcomes, these changes may be mild over long periods of time (e.g., ''electricity'', ''magnetic field'', ''atom'', etc.), while they will be more turbulent for others.\footnote{Terms used in relation to controversial issues, for example, offer many examples of changing definitions, driven by crowds. Classical such terms include, e.g., ''socialism'', ''atheism'', ''terrorism'', and so on, with ''Napster'' and ''The Pirate Bay'' being some of many more recent ones.\cite{wikipedia-controversial-issues,yasseri2014most}.}

The fourth important point concerns what a definition looks like. If you are used to printed dictionaries, then the definition looks like a sentence, or a few paragraphs at most, explaining the meaning of a term. Digital resources, however, especially for topics which draw an active community, look nothing like this. They are assemblies of various, more or less structured data, on the term being defined. This is important, because it takes us away from the simplistic one or two sentence definitions, into recognizing that there may be a lot to say about the term, to explain its intended use.

\newdef s are made and changed by a community; early in the innovation process, that community numbers only the team tasked with that process, and grows if and as the innovation process becomes successful. At any time, a \newdef{} will be a reflection of a temporary consensus on how to read a term, and from there - and more importantly - on what to do about it. It should be precise, clear, and accurate, so as to help others' be, in turn, precise, clear, and accurate about what they agree or disagree with, and stimulate them to act to make changes. Finally, a \newdef{} is how the community explains the term. If it satisfies some weak form of eliminability and conservativeness, it does so only to the extent that this does not hinder it being precise, accurate, and clear for those in the community.


% Chapter bibliography
\printbibliography



\chapter{How Does an \ncnf{} Relate to Decision-Making?}
\label{c6}

An \ncnf{} is created and changes during an innovation process. I argue in this chapter that definitions which we thus create reflect the decision-making done during innovation. I argue that definitions record choices we made, as well as embed constraints on our future decision-making. A key idea in this chapter is that new terms we define need to rely on old terms, those whose definitions we do not change during the innovation process; this relationship between new and old terms tells us how the outcomes of the innovation process relate to what was there already.

\section{How can it relate to decision-making?}
\label{c6:s1}
I mentioned, in earlier Chapters, that a definition in an \ncnf{} relates a new term to old. It does so in the following sense: when you introduce a new term, to explain to others what you mean by it, you provide a definition; for them to understand this definition, or any of it, you need to explain the meaning of the term using existing terms, those which you assume they understand similarly to you. So any term you put in an \ncnf{}, and this is independent from the plasticity of its definition, you will invariably explain its meaning using at least some words which are not in the \ncnf{}, but are part of the base language.

This Chapter looks at how the definition of a new term relates to terms which are not part of the \ncnf{}, but bring with them their own definitions, those which were not shaped during a given innovation process.

These ''old'' terms are only old in the sense that their definitions were created before and/or outside a given innovation process. Someone, sometime, somewhere had to make decisions on how to define these terms, and therefore, their definitions embed these decisions. Definitions reflect the choices we make, about how we explain that which we want the term to mean, or refer to.

To ''embed decisions'', then, means that a term's definition is a product of decisions of those who made it. And so, if someone else defined the term T, and you mention T in your definition of some new term S you are introducing in your \ncnf{}, your definition of S will embed the choices from the definition of T; indeed, someone else cannot understand your S unless they understand T.

\section{How New Ideas Use Old Terms?}
\label{c6:s2}
What is the relationship between new words and old words in an \ncnf ? How do old terms build up definitions of new ones?

Consider an example: How do you explain a satellite to a child? Here are two definitions that you could rely on.

\begin{svgraybox}
Satellite (OED definition): ''An artificial body placed in orbit round the earth or moon or another planet in order to collect information or for communication.'' \cite{def-satellite}
\end{svgraybox}

\begin{svgraybox}
Satellite (Wikipedia definition): ''In the context of spaceflight, a satellite is an artificial object which has been intentionally placed into orbit. Such objects are sometimes called artificial satellites to distinguish them from natural satellites such as Earth's Moon.'' \cite{wikipedia-satellite}
\end{svgraybox}

If the child already has an idea of artificial bodies, orbit, earth, planet, of what collecting information and communication are about in relation to these notions, then the first definition might be a good start. It is not an explanation, but we will come back later to the differences between definitions and explanations.

How about this, which NASA says is for children ages 10+?

\begin{svgraybox}
''A satellite is a moon, planet or machine that orbits a planet or star. For example, Earth is a satellite because it orbits the sun. Likewise, the moon is a satellite because it orbits Earth. Usually, the word ''satellite'' refers to a machine that is launched into space and moves around Earth or another body in space.'' \cite{nasa-satellite-58}
\end{svgraybox}

Or the following version, from NASA too, for a younger audience.

\begin{svgraybox}
''A satellite is an object that moves around a larger object. Earth is a satellite because it moves around the sun. The moon is a satellite because it moves around Earth. Earth and the moon are called ''natural'' satellites. But usually when someone says ''satellite,'' they are talking about a ''man-made'' satellite. Man-made satellites are machines made by people. These machines are launched into space and orbit Earth or another body in space.'' \cite{nasa-satellite-k4}
\end{svgraybox}

If you had to choose one, which one would it be? Are either of the first two good candidates?

You can think of these questions by looking at definitions backwards, from the definiens which defines or explains, to the definiendum, i.e., the term being defined.

Take the Wikipedia Satellite definition again. Notice it mentions the term ''context'', a complicated term. Context is an abstraction that bears multiple precise definitions, which are likely to vary across specific disciplines; i.e., context has a different definition in different technical languages. For example, Wikipedia gives this, for ''context'' in computer science:

\begin{svgraybox}
Context (Wikipedia definition, computer science): ''In computer science, a task context is the minimal set of data used by a task (which may be a process or thread) that must be saved to allow a task to be interrupted, and later continued from the same point.'' \cite{wikipedia-context-computing}
\end{svgraybox}

This is certainly not a unique or conventional understanding of the notion of context in computer science altogether. There are specific, and substantially different definitions of context in, e.g., knowledge representation, requirements engineering, mobile computing, and so on. Here is a definition from mobile computing research.

\begin{svgraybox}
Context (mobile computing): ''Context is any information that can be used to characterize the situation of an entity. An entity is a person, place, or object that is considered relevant to the interaction between a user and an application, including the user and applications themselves.'' \cite{abowd1999towards}
\end{svgraybox}

In short, context has a definition that varies between technical languages. In philosophy, the relationship between context and knowledge is an ongoing and longstanding debate.\cite{Rysiew2016}

Oxford English Dictionaries give the following generalist definition.

\begin{svgraybox}
Context (OED definition): ''The circumstances that form the setting for an event, statement, or idea, and in terms of which it can be fully understood.'' \cite{def-context}
\end{svgraybox}

Being twisted in this way, ''context'' is not a term to use lightly, when explaining ''satellite'' to children.

The important takeaway here is not that it is hard to explain a satellite to a child, but this: how you define a term depends on two things, namely
what you assume the user of that definition already knows, and
what they may want to do, once they learn the definition or explanation you give them.

You may, sometimes in error, assume that the child does not know enough for you to use the Oxford English Dictionary definition. If you also think the child needs it simply out of curiosity, then perhaps either of the two NASA definitions will do.

But what if an astrophysicist asks an engineer for a definition of ''satellite'', not out of curiosity, but because they are working together on the design of a new satellite, and they are trying to align their ideas. None of the definitions above might work; perhaps their definition would speak of gravitationally bound objects in orbits, or use other technical terms more appropriate to their background knowledge and the task at hand.

This gets us to the second and third takeaways
\begin{itemize}
	\item it makes little sense to think about one term and its definition in isolation from everything else, and
	\item a technical language leans on natural language to some extent, even if only by using terms from natural language in definitions of technical terms.
\end{itemize}

The following illustrates how the Satellite definition from the Oxford English Dictionary presumably leans on other natural language terms. Some are emphasized below.

\begin{svgraybox}
Satellite (OED definition, emphasis added): ''An \textit{artificial body} placed in \textit{orbit} round the \textit{earth} or \textit{moon} or another \textit{planet} in order to \textit{collect information} or for \textit{communication}.'' \cite{def-satellite}
\end{svgraybox}

But that definition of Satellite is taken from the Oxford English Dictionary, so by design, it leans on other words from there, i.e., on how these other words are defined there.

Let's go to a different example now, where some words in the definiens are not in the Oxford English Dictionary at the time of writing.

\begin{svgraybox}
Biosafety (UNDP definition): ''The prevention of large-scale loss of biological integrity, focusing both on ecological and human health. Set of measures or actions addressing the safety aspects related to the application of biotechnologies and to the release into the environment of transgenic plants and organisms, particularly microorganisms, that could negatively affect plant genetic resources, plant, animal or human health, or the environment.'' \cite{undp-glossary}
\end{svgraybox}

This definition of Biosafety comes from the United Nations Development Programme (UNDP), in a glossary for documentation on sustainable development. Below, terms set in bold are not defined in the Oxford English Dictionary, and so for all practical purposes here, not part of natural language. At the same time, these terms are not defined in that UNDP glossary; terms set in italics are not part of natural language, but are defined in the glossary. The rest of the terms are part of natural language.

\begin{svgraybox}
Biosafety (UNDP definition): ''The prevention of \textit{large-scale loss} of \textit{biological integrity}, focusing both on ecological and human health. Set of measures or actions addressing the safety aspects related to the application of biotechnologies and to the release into the environment of transgenic plants and organisms, particularly microorganisms, that could negatively affect plant genetic resources, plant, animal or human health, or the environment.'' \cite{undp-glossary}
\end{svgraybox}

To summarize, in the Biosafety definition above, there are two terms outside natural language and outside the glossary itself: ''large-scale loss'' and ''biological integrity'', and only ''Biosafety'' is itself defined in the glossary, no other term from the definiens is.

Is that a good definition?

This is hard to say, as ''good definition'' itself is an open topic, one that spans most of this book - recall the various perspectives discussed in chapter \ref{c4}. It is clear, however, that there are several things to think about here:
''Biosafety'' is defined independently from the rest of the glossary,
''Large-scale loss'' is undefined within the topic of the glossary, i.e., the glossary does not help reduce the vagueness of that term in any way, and
''Biological integrity'' is undefined in the glossary, and since it is not part of natural language, the authors in the glossary must have leaned on a technical language, but did not point to it in the glossary.

If these look like quibbles, they are not. Here is a sample of the activity around getting to and agreeing on a technical definition of ''biological integrity'':

\begin{quote}
''[United States Environmental Protection Agency] convened a symposium [...] on the integrity of water soon after passage of PL 92-500 [Federal Water Pollution Control Act], but no clear definition of biotic integrity emerged. Many authors advocated the use of a holistic perspective. Karr and Dudley [...] argued that the ''integrity'' objective encompasses all factors affecting the ecosystem and developed a now widely quoted definition of biological integrity as the ability to support and maintain 'a balanced, integrated, adaptive community of organisms having a species composition, diversity, and functional organization comparable to that of natural habitat of the region.' A more recent paper defined ecological health (an umbrella goal, the maintenance of which motivates virtually all environmental legislation) as follows: 'a biological system...can be considered healthy when its inherent potential is realized, its condition is stable, its capacity for self-repair when perturbed is preserved, and minimal external support for management is needed'[...].'' \cite{karr1991biological}
\end{quote}

Disagreement is common when specialists work on definitions. The problem is not disagreement, but disconnect, which happens if the discussion involves parallel monologues, each pushing its own definition, and no dialogue where there is convergence to an agreement on the definition of each debated term.

Going back to takeaways, these examples illustrate how a definition, even of a new technical term, depends on natural language and therefore of old terms and their readings accepted as part of natural language, however unclear they may be. As long as your technical language leans on natural language, which is not bad in itself, you are building something that is intended to be more precise, accurate, and clear from pieces which may not be that at all. In the worst case, you are simply changing where vagueness, ambiguity, and other deficiencies are coming from.

New ideas need new words. At the same time, new words are anchored in old ones.



\section{How Definitions Embed Choices?}
\label{c6:s3}
How do definitions in an \ncnf{} reflect our problem-solving and decision-making?

A technical language is a record of past decisions. This is a simple idea with significant consequences.

Let's start with a simple example, from the consulting contract case introduced in chapter \ref{c5}. We had the following definition.

\begin{svgraybox}
\textbf{Service:} The collaboration of one specific Consultant and one specific Client, and under the rules and guidelines set out by the Service Contract to which they have both agreed.
\end{svgraybox}

Consider these alternative definitions. Differences from the original one above are emphasized.

\begin{svgraybox}
\textbf{Service1:} The collaboration of one or more specific Consultants and one specific Client, and under the rules and guidelines set out by the Service Contract to which they have both agreed.
\end{svgraybox}

Service1 allows any number of Consultants to work for a Client under the same Service Contract. This may be better or worse than the original, but in either case, it can only be so if we decided that it should be so. In other words, the original definition excludes this possibility, because we decided to exclude that possibility. The main reason was that it would require coordination between Consultants, and that could have an impact on the speed at which we get to an agreed Service Contract.

Service2 below, is another alternative.

\begin{svgraybox}
\textbf{Service2:} The collaboration of one specific Consultant and one specific Client.
\end{svgraybox}

Service2 is less specific than either Service and Service1. You can see it as a less mature variant of each of these. It fails to mention the contract as the framework for collaboration. By doing that, it fails to say how we go from negotiation to collaboration, the transition which happens because there is agreement on a contract.

The point here, is that definitions, especially about new things, and especially if these new things are abstractions, ideas, are not given to us. We make them, and as a result, we define.

To clarify that given/made distinction, here is the Oxford English Dictionary definition of giraffe.

\begin{quote}
Giraffe (OED definition): ''A large African mammal with a very long neck and forelegs, having a coat patterned with brown patches separated by lighter lines. It is the tallest living animal.'' \cite{def-giraffe}
\end{quote}

This definition conveys what one generally sees, when looking at the animal called giraffe. Giraffes, in other words, are given -- it is not man-made, but a product of evolutionary processes. Alternatively, if we used a definition that lists a different set of properties, as below, we are still in the same case: a giraffe has these properties not because we made them so, i.e., the definition is not a reflection of our design of giraffes, but of our observation of giraffes.

\begin{quote}
Giraffe (AWF definition):
\begin{itemize}
\item ''Scientific name: Giraffa camelopardalis
\item Weight: Males: 1,930 kilograms (4,254 pounds) Females: 1,180 kilograms (2,601 pounds)
\item Size: 5.7 meters tall from the ground to their horns (18.7 feet)
\item Life span: Average 10 to 15 years in the wild; recorded a maximum of 30 years
\item Habitat: Dense forest to open plains
\item Diet: Herbivorous
\item Gestation: Between 13 and 15 months
\item Predators: Humans, lions, leopards, hyenas, crocodiles'' \cite{awf-giraffe}.
\end{itemize}
\end{quote}

But things are very different for our ''Service'' term above, since it refers to an abstraction, one we fully designed -- a team chose its properties and from there followed its definition.

There are also in-between cases where we are trying to define a complex process that has and is being observed from various perspectives. 

''Globalization'' is an interesting case, and below are four definitions that the United Nations Economic and Social Council cites in an official document \cite{unesc-terminology}.

\begin{quote}
Globalization (UN definition): ''Globalization is increased global integration and interdependence. It has a multidimensional character: economic, political, social and cultural. It is characterized by unprecedentedly rapid flows of goods and services: private capital; circulation of ideas and tendencies; and the emergence of new social and political movements.''
\end{quote}

\begin{quote}
Globalization (IMF definition): ''The process through which an increasingly free flow of ideas, people, goods, services and capital leads to the integration of economies and societies. Major factors in the spread of globalization have been increased trade liberalization and advances in communication technology.''
\end{quote}

\begin{quote}
Globalization (Hirst \& Peters definition): ''Globalization describes the growth in international exchange and interdependence. With growing flows of trade and capital investment, there is the possibility of moving beyond an international economy (where 'the principle entities are national economies') to a 'stronger' version of the globalized economy in which 'distinct national economies are subsumed and rearticulated into the system by international processes and transactions'.''
\end{quote}

\begin{quote}
Globalization (World Bank definition): ''Globalization can be defined as universalization. In this use, 'global' is used in the sense of being 'worldwide' and 'globalization' is 'the process of spreading various objects and experiences to people at all corners of the earth'. A classic example of this would be the spread of computing, television etc.''
\end{quote}

The World Bank definition looks more abstract than the others, in that it defines globalization as a type of universalization (which I leave undefined here), while the other three definitions talk of integration and interdependence.

Below is another definition of ''globalization'', as it appears in the United Nations UNTERM, ''a multilingual terminology database maintained jointly by the main duty stations and regional commissions of the United Nations system.''

\begin{quote}
Globalization (UNTERM definition): ''Expansion of global linkages, organization of social life on global scale and growth of global consciousness, hence consolidation of world society. Most particularly (per Friedman), the term is used to refer to the seemingly ''inexorable integration of markets, nation-states and technologies to a degree never witnessed before -- in a way that is enabling individuals, corporations and nation-states to reach around the world farther, faster, deeper and cheaper than ever before.'' This phenomenon also involves the spread of the gospel and forms of market capitalism to virtually every country in the world.'' \cite{unterm-globalization}
\end{quote}

Is any of these, or is there, at all, the definition of globalization?

It is a complex process, which we do not know how to measure (or there are different ways, but we cannot agree on one). There is disagreement on when, where and how it started. We do not know what cues to look for, which data is representative of it. We do not know if it has reached a peak, what reaching a peak means anyway, and we have hardly any solid ideas on how globalization might end.

\begin{quote}
''Today few doubt the reality of globalization, yet no one seems to know with any certainty what makes globalization real. So while there is no agreement about what globalization is, the entire discourse on globaliz- ation is founded on a quite solid agreement that globalization is. Behind the current and confusing debates about its ultimate causes and con- sequences, we find a wide yet largely tacit acceptance of the factuality of globalization as such, as a process of change taking place 'out there': even otherwise constructivistically minded scholars tend to regard globaliz- ation as an undeniable and inescapable part of contemporary experience.'' \cite{bartelson2000three}
\end{quote}

Which of these would you choose?

This question depends on what, if anything you want to do about, with, or in relation to the phenomenon that is defined.

More specifically, a choice between definitions depends on the extent to which their differences make a difference in your decision-making and problem-solving.

Going back to our small-scale consulting contract case, the choice of one of those three definitions given earlier (Service, Service1, and Service2) does make a difference. ''Service'' requires a different negotiation process than ''Service1'', and both can involve different processes than ''Service2''.

Definitions indeed record choices. If they define what is given and (relatively) easily observable, such as a giraffe, then definitions are more a record of choices of properties to pay attention to, among all potential (and perhaps unknown) properties of giraffes. If, instead, they define abstractions which we fully imagined and designed, then the definition reflects our choices of properties to give to these abstractions. And in many cases, we will be creating abstractions to communicate about potentially complicated combinations of phenomena, as we saw for globalization, where properties reflect, again, what we see through observation, but perhaps too, some wishful thinking, what we want these phenomena to be.

A definition is the bridge of the new to the old, we piece together the new from old notions. The new, by being there, by being defined, becomes a tool or vehicle for change, for moving not only ideas from old to new, but also learning to make new choices and act differently than before. 

Conceptualizations of ''globalization'', its definitions, work the same way, embodying observations, choices, mashing them up with expectations, making it a vehicle, not a simple reflection of what seems to be a shared experience. Bertelson's summary, of his discussion of the concept of ''globalization'' nicely sums up the multiple roles a definition can have, especially for complex phenomena, which lend themselves to analysis from different angles, and through tools from various bodies of knowledge.

As he puts it, the term ''globalization'' links the old and the new (label [A] in the quote below), denotes the interest in the phenomenon itself [B], tells us about what can be [C] and when it will be [E], and points to a vague reality [D], i.e., does not refer to something tangible or otherwise easily accessible, for which you and I could rapidly come to same or similar enough observations.

\begin{quote}
''Through its various connotations, the concept of globalization functions as [A: a mediating link between the modern world with its crusty social ontology and the brave new world that remains inaccessible and unintelligible not only to the subscribers to that ontology but also to the believers in global change as well]. Not only is globalization a moving target for social inquiry, but it also [B: signifies the movement of that inquiry itself]. The concept of globalization thus makes modern political experience meaningful while simultaneously releasing political expectation from the strictures that the very meaningfulness of this experience has imposed on political imagination. It does so through a double gesture: by [C: projecting expectations onto the global while simultaneously making those expectations constitutive of globality as a point of reference and convergence]. In this respect, the logic of the concept of globalization resembles that of the concepts of civilization and revolution as they were shaped before and during the French Revolution: [D: these concepts also lacked stable referents, but functioned as vehicles of social change by signifying change in its purest, most necessary and irreversible form: change as the condition of possible objects and possible identities in a possible future]. And like these concepts, globalization does not represent a mere prognosis for the future, but a prophecy in quest for self-fulfillment. So far from being here to stay, the metaphors of globalization will perhaps die when the concept has fulfilled its destabilizing task, that is, [E: when globalization has become something that goes without saying and therefore no longer stands in need of being spoken about].'' \cite{bartelson2000three}
\end{quote}


\section{How Definitions Make Choices?}
\label{c6:s4}
How can definitions make it easy do do some things, and hard to do others? How do they impose constraints on future choices?

If you commit to a definition, meaning you choose one definition of a term over others, then you also take additional commitments. It is yet another case of buying one and getting more than you asked for. How much of this more do you know? Do you want this additional baggage? These are the two central questions.

Law is an interesting domain when thinking about definitions. This is for two reasons: one, it insists on using terms in a precise and consistent way, and two, how these terms are used has substantial effects on people's lives.

Consider the hard case, of ''terrorism''. How would you define it?

Here is how terrorism is defined in the United Kingdom Terrorism Act. The version below is from 2018. Most text is the original definition, and it was included in law in 2000; text labeled [A] was added to the original definition in 2006, and the updated definition was published in the Terrorism Act 2006; text labeled [B] was added in 2008, and published in Counter-Terrorism Act 2008.

\begin{quote}
''Terrorism: interpretation.
(1) In this Act 'terrorism' means the use or threat of action where 
(a) the action falls within subsection (2), 
(b) the use or threat is designed to influence the government [A: or an international governmental organisation] or to intimidate the public or a section of the public, and 
(c) the use or threat is made for the purpose of advancing a political, religious [B:, racial] or ideological cause. 
(2) Action falls within this subsection if it 
(a) involves serious violence against a person, 
(b) involves serious damage to property, 
(c) endangers a person's life, other than that of the person committing the action,
(d) creates a serious risk to the health or safety of the public or a section of the public, or 
(e) is designed seriously to interfere with or seriously to disrupt an electronic system. 
(3) The use or threat of action falling within subsection (2) which involves the use of firearms or explosives is terrorism whether or not subsection (1)(b) is satisfied. \\
(4)In this section \\
(a) 'action' includes action outside the United Kingdom, 
(b) a reference to any person or to property is a reference to any person, or to property, wherever situated, 
(c) a reference to the public includes a reference to the public of a country other than the United Kingdom, and 
(d) 'the government' means the government of the United Kingdom, of a Part of the United Kingdom or of a country other than the United Kingdom. 
(5) In this Act a reference to action taken for the purposes of terrorism includes a reference to action taken for the benefit of a proscribed organisation.'' \cite{uk-terrorism-act}
\end{quote}

This is a complicated, but carefully made definition of a complex and important phenomenon. It is also a poorly understood phenomenon. Any definition, therefore, however attentively made, will have limitations.

One of the possible consequences of applying this definition is that some acts, which normally do not look like cases of terrorism, still could be categorized as terrorism.

\begin{quote}
"Civil disobedience, public protest and industrial action are among the activities that could fall within the definition. These types of activities should be excluded from any definition of terrorism. [...] Unlike the definition in the Australian Criminal Code (and its State equivalents), the New Zealand Terrorism Suppression Act, the Canadian Criminal Code and the 2003 South African Anti-Terrorism Bill, the United Kingdom definition does not contain an exception in favour of advocacy, protest or industrial action. The legislation simply requires the purported terrorist to have committed an act (such as endangering a person's life, or seriously damaging property), and to have committed that act in furtherance of a political, religious or ideological cause with the aim of influencing the government or intimidating the public (or a section thereof). This encompasses groups whose methods are generally non-violent and who do not aim to intimidate or to coerce the government or the public. For example, a long-running nurse's industrial dispute where staffing levels in public hospitals have been seriously reduced could 'create a serious risk to the health or safety of the public', within the meaning of s 1(2)(d) (as could the industrial actions of other essential services, such as fire officers, police, and so forth). If the strike were directed towards convincing the government to increase pay and conditions in public hospitals then this could also satisfy both the 'political cause' and the 'influencing government' requirements, in s1. Similarly, a mass student protest against the deregulation of university fees by the British Government could also fall within the definition of terrorism." \cite{golder2004terrorism}
\end{quote}

These are interesting potential consequence, regardless of intentions that led to the definition.

Going back to less complicated topics, and more in the context of \ncnf{}s, your choice of one definition over another has consequences on the structure of the problem space, of the solution space, and on the innovation process. This means that how you define something has an impact on what problem you are going to be solving -- that is the problem space part -- the potential ways you will be looking at to solve it -- the solution space part -- and on how you go through the formulation of the problem and its resolution -- the innovation process part. In other words, it can affect really all that goes on during innovation.

In chapter \ref{c5}, I introduced the case where we were designing a process for contract negotiation. One of the issues that was highlighted early on, was that such contract negotiation tends to take an unpredictable amount of time. They can last long, and if they last too long, they may simply not lead to the contract at all. So one of the aims that was taken for granted, was that the process should shorten the time to contract, or more broadly, to the decision to sign or not.

How do these early assumptions shape the problem space? Consider these three potential definitions for the contract negotiation process.

\begin{svgraybox}
\textbf{Contract Negotiation 1:} Process in which parties agree on terms of collaboration.

\noindent\textbf{Contract Negotiation 2:} Process in which a party that needs to have assignments performed communicates with potential parties that can execute these assignments, until she has either agreed with one on terms for collaboration towards the execution of these assignments, or has agreed with none of the considered parties.

\noindent\textbf{Contract Negotiation 3:} Process in which parties agree on terms of collaboration within two calendar weeks from the time the asking party requested a proposal from the supplying party.
\end{svgraybox}

The first definition is the least constraining. It leaves a lot open. Nothing is said about how many parties can be involved, how much time they can take, what they can collaborate on. The second definition is clear about the number of parties that can be involved, and how the process ends. Only the third definition mentions the duration of the process, and is precise about it, but is open regarding the number of participants.

If the innovation team adopts the third definition, they still need to decide on the number of participants, among others. But can they allow any number of participants? Does allowing fewer participants make two weeks a long time for negotiations? Is two weeks too short, when there are many participants? Does the number of participants influence duration? Is it the other way around, does the duration limit the number of participants? Should the process differ in maximal duration if it has many participants? Which numbers would require which durations?

If the team does indeed adopt the third definition, and makes it immutable, then this affects substantially the possible designs of the negotiation process. And it does so in ways which may not be easy to see as early on in the innovation process. We did, in fact adopt a definition which included a time limit. Because of this, we had to restrict communication between the parties. If we left it open, letting them freely communicate, we believed that they would not converge to a go/no go decision quickly enough. This, in turn, required that we provide template contracts, where it was clear which exact parameters could be negotiated. And that, then, required that we restrict communication between parties only to the negotiation of values of those parameters. Yet another unanticipated consequence of this, is that it can work only for highly structured contracts, where these negotiable parameters are few and well known. For example, a monthly retainer amount and a price per hour, together with the list of tasks. That is a far reaching consequence, because it affects the size of the potential market for the process, and the software that was to be made to support that process.

This does not mean that a more open definition (one with fewer constraints) is better. If the team takes seriously the first definition instead, then it may ignore the relationship of duration to outcome, and more specifically that the duration of negotiations influences negotiation success.

All these issues seem obvious as you read this text, but they are hard to see and analyze when you are involved in the innovation process. It is often only in retrospect that their merits and drawbacks become clear.

In that respect, you can see a definition as a bundle of predictions. In the case I mentioned, these are that two weeks are a good duration for contract negotiation, that the number of participants needs to be two or more, and so on. These are predictions about what will make the outcomes of innovation useful.

An analysis to do, that helps understand the implications of a definition on the problem and solution spaces, is to identify choices in the definition, and identify at least one other option for each. I did it below on the three definitions.

\begin{svgraybox}
\textbf{Contract Negotiation 1:} Process in which (one, two, more parties) (agree, disagree, cancel negotiations) on terms of (delegation, collaboration).
\end{svgraybox}

I identified three choices in the first definition. One is the number of parties, with four options (either one of the three, and all three as the fourth). The second is the outcome of negotiations, with four options as well. The final is the object of negotiations, with two options. You could have gotten other choices and other options for each. What the analysis highlights, is that the first definition commits to one out of 4x4x3 possible combinations; the problem space is restricted to the chosen options in each choice, and any solution will have to solve a problem in that part of the problem space.

Consider now the second definition.

\begin{svgraybox}
\textbf{Contract Negotiation 2:} Process in which a party that needs to have (one, more assignments) performed communicates with (one, more than one potential parties) that can execute (all, subset of these assignments), until she has either agreed with (none, one, more than one party) on terms for collaboration towards the execution of (all, subset of these assignments).
\end{svgraybox}

Above, I highlighted five choices, and a problem space of 3x3x3x3x3 combinations. You could easily enlarge this problem space, by allowing participants to make their own decisions. That would look like this.

\begin{svgraybox}
\textbf{Contract Negotiation 4:} Process in which a party that needs to have (one, more assignments, party's own choice of number of assignments) performed communicates with (one, more than one potential parties, the number of parties decided for that contract) that can execute (all, subset of these assignments, a minimum set by the asking party), until she has either agreed with (none, one, more than one party, number chosen by the asking party) on terms for collaboration towards the execution of (all, subset of these assignments, minimal set required for the asking party to accept the contract).
\end{svgraybox}

And this is only for a rough definition of a single term. You can imagine how this becomes complicated to think about, if we start looking at choices in two, three, or more realistically, \ncnf{}s with a few dozen, and sometimes hundreds of terms. Even if you dedicated effort to the analysis of this kind, to each definition alone, you would still have to do this for interactions of choices which are in definitions of different terms, but depend on each other.

Same analysis is now applied to the third definition.

\begin{svgraybox}
\textbf{Contract Negotiation 3:} Process in which (two, more than two, number set by asking party) parties agree on terms of collaboration within (one, two, custom number chosen by asking party) calendar weeks (from the time the asking party requested a proposal from the supplying party, from the time the first supplying party responded to the request for proposal, either of these as requested by asking party).
\end{svgraybox}

Again, notice how a seemingly simple definition can cause headache, as the number of choices and options in each start to multiply.

Let's take a step back. What goes on in this analysis? I said I was identifying choices. In abstract terms, these are assignments of values to variables. I took easy examples in these definitions, because it is straightforward to think about, e.g., the number of participants as a variable that can take different numerical values, positive integers. But this is less apparent in the third definition, for the choice of trigger that starts the negotiation timer, where I had the choice "two calendar weeks from the time the asking party requested a proposal from the supplying party".

In simple terms, choices are all items in a definition that you could change. A choice in a definition is something that can or could be different.

A choice is not necessarily easy to identify in the definition. It could be straightforward, like in the cases I had above, or much harder, as in the following example.

\begin{svgraybox}
Shippable: A Shippable Unit (a.k.a. "Shippable" only) satisfies all the following conditions:
\begin{itemize}
	\item It involves one or more software functionality;
	\item It is represented, in the release management process, by exactly one User Story;
	\item It is released to UAT Environment;
	\item Prior to its release to UAT Environment, it has been tested by Vibe to ensure that this software functionality operates in accordance to its Product Specification;
	\item Its Product Specification has been approved, prior to the commitment of development resources to its implementation, i.e., prior to its User Story being introduced for the first time in a Sprint.
\end{itemize}
\end{svgraybox}

This comes from a terminology made by a software engineering team that was adopting for its own work style a general purpose development method. You might recognize that they were adopting and adapting for themselves, a variant of the Scrum method for software development. A first choice, which is hard to see in this definition, is that a shippable should involve pieces of functionality; it was only when the team started dedicating more resources to analysis tasks, which do not produce functionality, but do produce designs of future functionality, that they recognized that outcomes of these analyses cannot be presented as shippables, which skewed data on team's performance, and caused problems when performance assessments started being done by a third party. This was, in fact not a conscious choice, because the team was already - when the definition was made and approved - involved for months in developing functionality which was well specified up front, and there was no analysis effort to do. That option was not considered, and when no options are considered, this part of the Shippable definition seems, at that time, as immutable, it is not a choice at all, even if this turned out to be a mistake.

Another issue is the last item, where the Product Specification needs to be approved up front. This turned out to be a problem, because if we had to follow it rigorously, it would delay a number of decisions to change minor aspects of the product, decisions which we ended up taking anyway, without having the approved specification. However, when the definition was being made and approved, it seemed like the only option.

It can, in short, be hard to see that something in a definition is a choice in the first place. But even in the hard example above, choices were still items, parts of the definition.

It can be harder still, as when the definition itself is a choice, and taking another option would lead you to throw away that definition altogether.

In the Shippables example above, the notion of a shippable is associated, as are the notions of User Stories, Sprints, and so on, with the Scrum method for software development. The choice of going along with that method, of aligning to it, is in fact a choice like any other. If we wanted a different method, there may not be a notion of Shippable there at all.

This highlights the fact that you need to worry about choices that led to the introduction of the term to the \ncnf{}, then about choices captured in the definition itself, and the choices that are implicit, which you only learn through experience.

Troubles do not end there. I have a rather different understanding of options and choices when I am analyzing definitions that I create, as opposed to those that others made. Indeed, a term's definition is simply an outcome of the thinking and communication that led to it (including one's specific experience and expertise accumulated up to that point). If you did not participate in, or do that thinking, then your analysis of choices in a definition will be only on the definition itself, and the rest of the \ncnf{}.

But even if I had to make a definition of Shippable, I can still know little about Scrum, or software development altogether. Your analysis, in other words, inevitably leans on the knowledge you have.

And even if I knew a lot about software engineering methods, it would be more appropriate to know specifics of that team, who will be using the definition, than if my knowledge was generic.

The bottom line is that the less you know and the more indirect that knowledge is, the more difficult it is to anticipate the effects of choices in the definitions you make, or are handed over.

I mentioned three parameters so far, which play into your ability to predict the effects of choices in a definition. They are:
\begin{itemize}
	\item If your knowledge is direct or indirect; it is direct if you have expertise and experience on phenomena at hand; it is not if they are new to you, and much of what you can think of these phenomena is, thus, speculation based on, for example, analogy;
	\item How extensive your knowledge on the phenomenon is. The less extensive it is, the more likely it is that many choices in the definition will remain implicit to you;
	\item If you are an author of the definition, or only its recipient, in which case you did not participate in whichever explicit choices were made when that definition was designed.
\end{itemize}

These three dimensions are orthogonal. Your level on one has no relationship to your level on another. With this in mind, your worst case is being a recipient of a definition, while having low indirect knowledge of the phenomenon that the definition is about.

But what if choices in one definition are linked to choices in others? That requires a different view of an \ncnf{}, presented in the rest of this book.


% Chapter bibliography
\printbibliography


%%%%%%%%%%%%%%%%%%%%%%%%%%%%%%%%%%%%%%%%%%%%%%%%%%%%%%%%%%%%%%%%%%%%%%%%%%%%%%%%
%%%%%%%%%%%%%%%%%%%%%%%%%%%%%%%%%%%%%%%%%%%%%%%%%%%%%%%%%%%%%%%%%%%%%%%%%%%%%%%%
%%%%%%%%%%%%%%%%%%%%%%%%%%%%%%%%%%%%%%%%%%%%%%%%%%%%%%%%%%%%%%%%%%%%%%%%%%%%%%%%
%%%%%%%%%%%%%%%%%%%%%%%%%%%%%%%%%%%%%%%%%%%%%%%%%%%%%%%%%%%%%%%%%%%%%%%%%%%%%%%%
%%%%%%%%%%%%%%%%%%%%%%%%%%%%%%%%%%%%%%%%%%%%%%%%%%%%%%%%%%%%%%%%%%%%%%%%%%%%%%%%
%%%%%%%%%%%%%%%%%%%%%%%%%%%%%%%%%%%%%%%%%%%%%%%%%%%%%%%%%%%%%%%%%%%%%%%%%%%%%%%%
\begin{comment}

\chapter{Defining to Destroy}\label{c:plastic-definition}




\section{Can General Dictionaries and Encyclopedia Help?}
If there is reason to worry about the shared understanding of words, the natural reaction is to go and have this settled with a dictionary. 

\begin{quote}
''Dictionaries are often perceived as authoritative records of how people 'ought to' use language, and they are regularly invoked for guidance on 'correct' usage.'' \cite{atkins2008oxford}
\end{quote}

It should be obvious that this won't work here. There are two reasons neither a dictionary, nor an ancyclopedia will solve the disagreement around the naming and defining of new ideas.

One, if we were debating what \textit{load} or \textit{shipment} should be defined as, \textit{in general-purpose communication}, then a dictionary would be good enough. But when doing the design of something new, which turns out to require its own, local meaning for a common word, we need a definition which suits its specific use for that particular purpose. It is useful, as we will see later, to look at how this specific definition relates to the general-purpose one, but the former will replace the latter in this context of innovation that we were dealing with. So, one reason is that dictionaries and encyclopedias will provide a general-purpose definition for a word, yet we need a specific definition suitable to the innovation we are working on; in other words, we need to create a definition which fits the ideas we came up with, and made decisions about, in our innovation process.

Two, specificity is only part of the story: innovation means new ideas, and whichever words we choose to name these new ideas, the ideas are still new -- that's why we are talking about innovation in the first place. So, we musn't assume in an innovation process that a common word can keep its \textit{old} definition, even if it is a specific one. This is precisely because we look for novelty. 

The error of taking an old word to have its old definition, all the while knowing this word to be central in an innovation process (such as our \textit{load}, \textit{shipment}, \textit{to match}, and many more), is easier to avoid if we go for neologisms, rather than old words. But the difficult problem is not naming, it's definition. If we invent a new word, we still have to make a definition for it, which again disqualifies dictionaries and encyclopedias as solutions. 


\section{Is a Terminology the Solution?}\label{s:terminology-1}
If a general-purpose dictionary or encyclopedia cannot solve the disagreement over loads, shipments, and matching, then the next candidate to consider are more specialized definitions of these words. That candidate is a terminology applicable to, say, freight transportation, logistics, or some such area.

In the Oxford English Dictionarries \cite{oed-load}, ''load'' gets the following definition:

\begin{quote}
    ''A heavy or bulky thing that is being carried or is about to be carried.''
\end{quote}

One among many options for getting more specific about ''load'' is the Iowa Department of Transportation's glossary. It has no entry for ''load''; the closest is ''cargo'' \cite{iowa-dot-cargo}:

\begin{quote}
    ''Anything other than passengers, carried for hire, including both mail and freight.''
\end{quote}

Next, we could take a definition of ''truckload'' from the terminology at C.H. Robinson, a major freight brokerage company in USA \cite{chrobinson-truckload}:

\begin{quote}
    ''Truckload is a mode of freight for larger shipments that typically occupy more than half and up to the full capacity of a 48’ or 53’ trailer. This method is commonly used when shippers decide they have enough items to fill a truck, want their shipment in a trailer by itself, the freight is time-sensitive or the shipper decides it’s more cost-effective than other options.''
\end{quote}
Another comparable freight broker, XPO Logistics, defines ''truckload'' as follows:
\begin{quote}
    ''The ground transportation of cargo provided by a single shipper in an amount that requires the full limit of the trailer, either by dimension or weight. Cargo typically remains on a single vehicle from the point of origin to the destination and is not handled en route.''
\end{quote}
Here is one of the many versions of the ''load'' definition in our logistics venture:
\begin{quote}
    ''Data held about an actual load which needs to be transported; includes: origin location, destination location, pickup time window, delivery time window, load value, weight, length, height, width, load content, trailer requirements.''
\end{quote}
Notice how ''load'' can mean different things. All are more or less for related, seemingly overlapping ideas. But if you had to make sure there was a shared interpretation in a team, then they are about very different ideas.

Standardization of definitions remains the major motivation for developments in terminology as a field, and developments of specific terminologies. The expected benefit is to avoid costly misunderstanding.



\begin{quote}
''The first meaning of the word terminology is 'the set of special
words belonging to a science, an art, an author, or a social entity,'
for example, the terminology of medicine or the terminology of
computer specialists.
 
The same term, in a more restrictive sense, means 'the language
discipline dedicated to the scientific study of the concepts and terms
used in specialized languages.' General language is that used in
daily life, while a specialized language is used to facilitate
unambiguous communication in a particular area of knowledge,
based on a vocabulary and language usage specific to that area.'' \cite{pavel2001precis}
\end{quote}

Motives for somehow correcting or improving communication, what Herbert Picht calls ''terminological deficits'', have been central to the development of terminology:

\begin{quote}
''
Looking into the historical evidence we can state some central terminological deficits:
\begin{enumerate}
    \item Lack of or incorrect conceptual ordering. Linn\'{e} (1707-1778) established
a systematisation of concepts by his works on taxonomy. The superior
aim of all later classifications was the ordering of knowledge as expressed
by terms.
    \item Confusion caused by excessive synonymy. Beckmann (1739-1811),
professor of philosophy and economics, criticised the multitude of unnecessary
and confusing synonyms.
    \item Lack of terms for the concept in a particular language. Already in the
Middle Ages the translators of the School of Toledo had to struggle with this problem.
 \item Unclear and undefined concepts. Clausewitz, the German military
theorist, wrote: Only when a clarification of the names and concepts has taken place, may one hope to proceed easily and with clarity in the treatment
of the matter.
 \item Language planning deficits. D\"{u}rer tried to establish a German terminology
for mathematical concepts -- although without success. Berthollet,
de Morveau, Fourcroy and Lavoisier were successful in creating a chemical terminology in the 18th century. Czechoslovakia after 1919, the Baltic States after 1919 and 1990, the Catalans, the Basques and several others had to fight the language planning problem –- a problem which is increasingly acute in many language communities.
\end{enumerate}
From this small historical evidence we can deduce that it was first and
foremost the specialists and language for specific purposes mediators
(translators) who felt the need to improve professional communication
by solving basic terminological problems.
'' \cite{picht2011science}
\end{quote}

The importance of terminology for effective communication is well-known, being recognized by the International Organization for Standardization; it is worth recalling the central two here, as they are a result of substantial, long-term efforts to agree on what terminology may be, what it is useful for, how specific terminologies should be created, and by whom:
\begin{itemize}
    \item \textit{ISO 1087 Terminology work -- Vocabulary} provides ''a systemic description of the concepts in the field of terminology and to clarify the use of the terms in this field'', a terminology for terminologists, in a sense.
    \item \textit{ISO 704 Terminology work — Principles and
    methods} aims ''to standardize the essential elements for terminology work'', i.e., how to create and improve terminologies.
    \item A number of standards and recommendations on storing, organizing, and sharing terminologies in digital format\footnote{For an overview and historical developments, see the work of }.
\end{itemize}



, with ISO 704 \cite{iso704}, and consortia, such as Terminology for Large Organizations (TerminOrgs). Here is the latter's position on why terminology matters in large organizations:

\begin{quote}
''Effective communications is a goal of all organizations that deal with the public, commercially or otherwise. This includes businesses, enterprises, public institutions, NGOs, governments, and any other type of organization. When the organization operates in different linguistic communities, requiring different languages, the goal of effective communication requires a proactive approach that includes terminology management. These types of organizations are characterized as 'global.' [...]

At the research and development stage, the use of different terms for core features or functions can lead to misunderstanding among workers. Errors can occur, and some production tasks may even need to be repeated as a consequence, often at great cost. 

After a product or service has been developed, the informational and marketing content is produced, then translated. There is often a disconnect between the marketing department and the product development department. Each has its own team of writers. Inconsistent and conflicting terminology between marketing and development content concerning the same topic (product or service) is a common problem. A centralized termbase is a tool that helps to ensure consistent and appropriate use of language throughout the organization. Without a termbase, language problems are left to editors to detect based on their own internal knowledge. Many inconsistencies and problems are undetected, and are then repeated unknowingly by translators in the translated versions. Furthermore, the editing stage is the very end of the content production cycle, after nearly all the content for a product has been produced. At this stage, the problems are multiplied many times over and the cost of fixing them is substantial.
'' \cite{terminorgs2016}
\end{quote}

Clearly, we should look for principles and methods promoted for the design of terminologies, if we want to reduce risk from misunderstanding in communication.

\section{Terminology for Changing Terms?}
% (Cite Sowa on meaning changes, language games, find if Wittgenstein wrote on this in relation to meaning change...)

Risk of misunderstanding can be mitigated through precise, accurate, and clear definitions of ideas, in other words, the construction of a terminology which should, ideally, establish clear relationships between names, ideas, and objects. This has been a principle promoted in philosophy since the Greeks, has had many proponents continuously since, and is warmly adopted in contemporary science and engineering, having been converted into international standards for industry (ISO704, ISO12...). 

The economics of terminology are clear when definitions need to be made for ideas which stabilized in the relevant community: the investment in producing precise, accurate, and clear definitions of specialized stable ideas makes sense because these definitions will help reduce future misunderstanding, and critically, they will not have to change frequently: there will be no need to frequently make similar investments again.

What does this mean when ideas are new and expected to change frequently? Does it mean that we should not invest in terminology during innovation? Is the only reasonable implication that we should wait for ideas to stabilize, before investing in creating precise, accurate, and clear definitions thereof? 

The central idea in this book is that terminology of stable ideas and terminology of unstable ideas have fundamentally different purposes. 

The purpose of creating a terminology of stable ideas is to keep them stable and avoid disagreement. This cannot be the purpose of creating a terminology of unstable ideas: instability is due to our search for how to improve these ideas, and it is only if they prove their usefullness that they become stable, and we stop looking for how to improve them.




% Yes, but there are two twists:



\section{}
Unless you are working in over-the-road logistics, it is unlikely that you know what I meant above by ''truckload'' or ''less-than-truckload''. It wouldn't be hard to find by searching online. These are widely used terms in logistics. If we had to agree on their \textit{general} definitions, we could do that quickly. The less these definitions matter to what we will do next -- the less they matter in our decisions and for our actions -- the easier it will be to agree on them. If, in other words, it makes no difference what they are, then why disagree?

What if I had my own notion of ''truckload'', which is not the same as the general one? What if the general definition doesn't work for me? What if, in addition, it really did matter what truckload stands for? What if going along my idea of truckload meant a certain cost and time for our future software product, and yours implied different cost and time? Then the definition matters, and agreement will depend on how the definition we are looking into fits your goals and mine.

I bet you were not among the twenty people I worked with in 2016, when we got into that logistics startup. If not, then I'm sure that you and I do not have the same notion of ''load'' and ''shipment''. We had serious difficulties to align internally on what these two mean. It got worse when we started growing faster.

I made a mistake in that startup. It was a mistake I did not make in the ten years before that, while working for various investors to build other software startups, in Belgium, Denmark, Israel, Italy, and the USA.
 
The mistake was this: I did not insist that we create, maintain, improve, and adhere to a precise, clear, and accurate glossary of terms we invented as we built and grew the business, its people, and its technology. We had no written glossary about the innovations we were designing, making, and releasing to customers.

A glossary is a list of terms and their definitions. Most non-fiction books come with a glossary, and technical literature almost always has it, so you probably saw them many times.

The problem with glossaries in startups, is that startups -- at least those I was involved in -- had three characteristics: 
\begin{itemize}
    \item They were \textit{new companies}, neither spin-offs of existing ones, nor incubated by existing ones, nor academic spin-offs; they started off without a direct access to early adopting customers, without an already well-oiled team that had worked together in the past.
    \item Their investors wanted \textit{high returns over short periods of time}: for every dollar invested, the aim was to return 5 or more, ideally above 10, within 2 to 4 years.
    \item \textit{The only way to achieve these returns was through innovation}; we were tasked with coming up with new ways to solve problems which many of us were trying to understand for the first time.
\end{itemize}
 
You might be seeing the outlines of the paradox, with wanting to have a glossary in such a setting. This \textit{\ndparadox{}} is as follows.

To make the glossary, you have to define the new ideas that this new team is coming up with. But they are new ideas, and therefore, they are likely to change substantially as you continue your innovation process. 

Hence the paradox: \textit{you are trying to define new ideas in a precise, accurate, clear way, all the while knowing you will soon be throwing many of these definitions away}. You are, in other words, trying carefully craft something that you know will be short lived.

So, what do you do? Do you invest more time to make and remake the glossary? Or do you ignore it, and hope misunderstandings will be infrequent and quick to resolve. 

The added difficulty in a startup is that \textit{this has to be done by a team which is new}: they have not worked together before, do not use the same terms in similar ways, and they are now asked to do innovation together.

However, here's what makes it worthwhile to think about and try to live with the \ndparadox, instead of ignoring it: 
\begin{itemize}
    \item misunderstandings which are caused by the same terms being about different ideas\footnote{In linguistics, such terms are called plurisemantic terms.} are recurrent,
    \item they are proportional to the number of people involved, and 
    \item the longer they stay undetected, the more damage they make. 
\end{itemize}

The only way to stop them from repeating, is to have everyone agree on a definition, and have that definition written down. In other words, by having a glossary. This glossary will keep changing, but it has to be written down and available to all.

Why are they proportional to the number of people involved? 

When we invented the ''load'' and ''shipment'' concepts for the product we were creating, we unfortunately used common words for something that was in fact very specific, that is, different from everyday meaning people may think of when hearing or using these same words. It was not specific in the sense that we went to an existing glossary of, say, terms in logistics, but we had our own meaning for both of these -- and in fact many other common nouns and verbs. Every time we have someone new come to this, or someone who forgets the specifics of our ''load'' and ''shipment'' concepts, we had to go through the ritual of explaining what was the intended meaning of these terms. Each person was bringing, quite expectedly, their own thoughts about what a load or shipment are, in general, or often in their own prior companies and jobs. 

To see the extent to which a ''load'' can mean different things, consider the following sample of definitions. All are more or less for the same term, refer to related, seemingly overlapping ideas. But if you had to make sure there was a shared interpretation in a team, then they are about very different ideas.

\begin{itemize}
    \item Oxford English Dictionaries \cite{oed-load}: ''A heavy or bulky thing that is being carried or is about to be carried.'' 
    \item Iowa Department of Transportation's definition of ''cargo'' \cite{iowa-dot-cargo}, the closest concept to ''load'' among the terms defined there: ''Anything other than passengers, carried for hire, including both mail and freight.''
    \item Sample definition of ''truckload'' from a major logistics company \cite{chrobinson-truckload}; ''truckload'' is the closest to our idea of ''load'': ''Truckload is a mode of freight for larger shipments that typically occupy more than half and up to the full capacity of a 48’ or 53’ trailer. This method is commonly used when shippers decide they have enough items to fill a truck, want their shipment in a trailer by itself, the freight is time-sensitive or the shipper decides it’s more cost-effective than other options.''
    \item Another major company's definition of ''truckload'' \cite{xpolog-inv-pres-aug-2019}: ''The ground transportation of cargo provided by a single shipper in an amount that requires the full limit of the trailer, either by dimension or weight. Cargo typically remains on a single vehicle from the point of origin to the destination and is not handled en route.''
    \item One of many versions of our definition of ''load'': ''Data held about an actual load which needs to be transported; includes: origin location, destination location, pickup time window, delivery time window, load value, weight, length, height, width, load content, trailer requirements.''
\end{itemize}

\noindent Why does a misunderstanding create more damage the longer it stays unresolved? 

When we invented our ''load'' and ''shipment'', and as we kept changing these meanings through our innovation processes, we did this in order to eventually have a software system made, to record, manage, and do some computations on data about our ''loads'' and ''shipments''. If I had one understanding of what ''load'' and ''shipment'' are, an engineer had another, and others had their own, we might superficially agree on what data this system needs to work with, and how, but it would be likely that they would deliver something that I did not expect. How could they, if they are making the software according to their idea of ''load'' and ''shipment'', while I expect it to fit my idea of ''load'' and ''shipment''? If I think that a load can only several pickup locations and one delivery location, and she keeps thinking it can have any number of either of these locations, and the engineering team thinks it can have only one of each of the two locations, we are in trouble: it might take the team months to implement their idea of ''load'', and then more months to change it after they deliver a system which fails to match the varying expectations of its stakeholders.

It is, of course, quite nice to suggest that a team that does innovation should make and improve this fleeting glossary of terms which defines their new ideas. What's difficult is to do this and not waste time in the process. This book is about how to do that.

The book has three parts.
\begin{itemize}
    \item Part 1 is about how to make glossaries of new ideas, how to update them, and how to use them. If your innovation glossary is 20 terms or less, or thereabout, and you need a practical guide, then this Part 1 may be all you need to read. 
    \item Part 2 looks at how to make, change, and use bigger glossaries. The techniques I show there can be used with small glossaries too, but really make a difference as you move past 20 or so terms. We'll see there that it is not really the number of terms that matters, but a certain kind of dependency between them. 
    \item Making these glossaries actually begs many questions which you might want to consider, especially after making a few of them. When is a definition of a term good enough? How are definitions of new ideas different from definitions of established ideas? How much confidence can you have in definitions of new ideas? How can you be even more precise when creating the glossary of new ideas? These are questions that have received quite some attention in philosophy, especially ontology and epistemology, as well as computer science, namely in knowledge representation and reasoning, ontology engineering, natural language processing, requirements engineering, and formal specification. Part 3 gets technical at times, but should be interesting if you want to go further than Parts 1 and 2.
\end{itemize}

\end{comment}
%%%%%%%%%%%%%%%%%%%%%%%%%%%%%%%%%%%%%%%%%%%%%%%%%%%%%%%%%%%%%%%%%%%%%%%%%%%%%%%%
%%%%%%%%%%%%%%%%%%%%%%%%%%%%%%%%%%%%%%%%%%%%%%%%%%%%%%%%%%%%%%%%%%%%%%%%%%%%%%%%
%%%%%%%%%%%%%%%%%%%%%%%%%%%%%%%%%%%%%%%%%%%%%%%%%%%%%%%%%%%%%%%%%%%%%%%%%%%%%%%%
%%%%%%%%%%%%%%%%%%%%%%%%%%%%%%%%%%%%%%%%%%%%%%%%%%%%%%%%%%%%%%%%%%%%%%%%%%%%%%%%
%%%%%%%%%%%%%%%%%%%%%%%%%%%%%%%%%%%%%%%%%%%%%%%%%%%%%%%%%%%%%%%%%%%%%%%%%%%%%%%%




%% APPENDIX - EMPTY
%%%%%%%%%%%%%%%%%%%%%% appendix.tex %%%%%%%%%%%%%%%%%%%%%%%%%%%%%%%%%
%
% sample appendix
%
% Use this file as a template for your own input.
%
%%%%%%%%%%%%%%%%%%%%%%%% Springer-Verlag %%%%%%%%%%%%%%%%%%%%%%%%%%

\appendix
\motto{All's well that ends well}
\chapter{Chapter Heading}
\label{introA} % Always give a unique label
% use \chaptermark{}
% to alter or adjust the chapter heading in the running head

Use the template \emph{appendix.tex} together with the Springer document class SVMono (monograph-type books) or SVMult (edited books) to style appendix of your book.


\section{Section Heading}
\label{sec:A1}
% Always give a unique label
% and use \ref{<label>} for cross-references
% and \cite{<label>} for bibliographic references
% use \sectionmark{}
% to alter or adjust the section heading in the running head
Instead of simply listing headings of different levels we recommend to let every heading be followed by at least a short passage of text. Furtheron please use the \LaTeX\ automatism for all your cross-references and citations.


\subsection{Subsection Heading}
\label{sec:A2}
Instead of simply listing headings of different levels we recommend to let every heading be followed by at least a short passage of text. Furtheron please use the \LaTeX\ automatism for all your cross-references and citations as has already been described in Sect.~\ref{sec:A1}.

For multiline equations we recommend to use the \verb|eqnarray| environment.
\begin{eqnarray}
\vec{a}\times\vec{b}=\vec{c} \nonumber\\
\vec{a}\times\vec{b}=\vec{c}
\label{eq:A01}
\end{eqnarray}

\subsubsection{Subsubsection Heading}
Instead of simply listing headings of different levels we recommend to let every heading be followed by at least a short passage of text. Furtheron please use the \LaTeX\ automatism for all your cross-references and citations as has already been described in Sect.~\ref{sec:A2}.

Please note that the first line of text that follows a heading is not indented, whereas the first lines of all subsequent paragraphs are.

% For figures use
%
\begin{figure}[t]
\sidecaption[t]
%\centering
% Use the relevant command for your figure-insertion program
% to insert the figure file.
% For example, with the option graphics use
\includegraphics[scale=.65]{figure}
%
% If not, use
%\picplace{5cm}{2cm} % Give the correct figure height and width in cm
%
\caption{Please write your figure caption here}
\label{fig:A1}       % Give a unique label
\end{figure}

% For tables use
%
\begin{table}
\caption{Please write your table caption here}
\label{tab:A1}       % Give a unique label
%
% For LaTeX tables use
%
\begin{tabular}{p{2cm}p{2.4cm}p{2cm}p{4.9cm}}
\hline\noalign{\smallskip}
Classes & Subclass & Length & Action Mechanism  \\
\noalign{\smallskip}\hline\noalign{\smallskip}
Translation & mRNA$^a$  & 22 (19--25) & Translation repression, mRNA cleavage\\
Translation & mRNA cleavage & 21 & mRNA cleavage\\
Translation & mRNA  & 21--22 & mRNA cleavage\\
Translation & mRNA  & 24--26 & Histone and DNA Modification\\
\noalign{\smallskip}\hline\noalign{\smallskip}
\end{tabular}
$^a$ Table foot note (with superscript)
\end{table}
%


%%%%%%%%%%%%%%%%%%%%%%%%%%%%%%%%%%%%%%%%%%%%%%%%%%%%%%%
%%%%%%%%%%%%%%%%%%%%%%%%%%%%%%%%%%%%%%%%%%%%%%%%%%%%%%%
%%%%%%%%%%%%%%%%%%%%%%%%%%%%%%%%%%%%%%%%%%%%%%%%%%%%%%%
%% BACK MATTER
\backmatter

%% GLOSSARY - EMPTY
%%%%%%%%%%%%%%%%%%%%%%%acronym.tex%%%%%%%%%%%%%%%%%%%%%%%%%%%%%%%%%%%%%%%%%
% sample list of acronyms
%
% Use this file as a template for your own input.
%
%%%%%%%%%%%%%%%%%%%%%%%% Springer %%%%%%%%%%%%%%%%%%%%%%%%%%

\Extrachap{Glossary}


Use the template \emph{glossary.tex} together with the Springer document class SVMono (monograph-type books) or SVMult (edited books) to style your glossary\index{glossary} in the Springer layout.


\runinhead{glossary term} Write here the description of the glossary term. Write here the description of the glossary term. Write here the description of the glossary term.

\runinhead{glossary term} Write here the description of the glossary term. Write here the description of the glossary term. Write here the description of the glossary term.

\runinhead{glossary term} Write here the description of the glossary term. Write here the description of the glossary term. Write here the description of the glossary term.

\runinhead{glossary term} Write here the description of the glossary term. Write here the description of the glossary term. Write here the description of the glossary term.

\runinhead{glossary term} Write here the description of the glossary term. Write here the description of the glossary term. Write here the description of the glossary term.

%% EXCERCISE SOLUTIONS - EMPTY
%
\Extrachap{Solutions}

\section*{Problems of Chapter~\ref{intro}}

\begin{sol}{prob1}
The solution\index{problems}\index{solutions} is revealed here.
\end{sol}


\begin{sol}{prob2}
\textbf{Problem Heading}\\
(a) The solution of first part is revealed here.\\
(b) The solution of second part is revealed here.
\end{sol}



%%%%%%%%%%%%%%%%%%%%%%%%%%%%%%%%%%%%%%%%%%%%%%%%%%%%%%%
%% INDEX
\printindex

%%%%%%%%%%%%%%%%%%%%%%%%%%%%%%%%%%%%%%%%%%%%%%%%%%%%%%%
%%%%%%%%%%%%%%%%%%%%%%%%%%%%%%%%%%%%%%%%%%%%%%%%%%%%%%%
%%%%%%%%%%%%%%%%%%%%%%%%%%%%%%%%%%%%%%%%%%%%%%%%%%%%%%%
\end{document}





